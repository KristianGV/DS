%%%%%%%%%%%%%%%%%%%%%%%%%%%%%%

\section{Dark matter distributions -- theory}
\label{sec:halo}

All the dark matter detection rates depend in one way or another on
the properties of the Milky Way dark matter halo. We will here outline
the halo model that by default is included with \ds.

Observationally, the distribution of DM on scales relevant for DM searches is only poorly constrained. 
The situation is somewhat improved when instead referring to the results of large $N$-body simulations 
of gravitational clustering, which consistently find that DM halos {\it on average} are well described
by Einasto \cite{1965TrAlm...5...87E} or Navarro-Frenk-White profiles \cite{Navarro:1995iw}, with
the halo mass being essentially the only free parameter (after taking into account that the halo 
concentration strongly correlates with the halo mass \cite{Maccio:2008pcd}). On the other hand, there is 
a considerable halo-to-halo scatter associated to these findings, so that it remains challenging
to make concrete predictions for individual objects -- in particular if they are located in 
cosmologically somewhat `special' environments like in the case of the Milky Way and its
embedding in the Local Group. Even worse,
baryonic physics can have a large impact on the DM profiles, especially on their inner parts 
most relevant for indirect detection, and even though hydrodynamic simulations taking into
account such effect have made tremendous progress in recent 
years \cite{Schaye:2014tpa, Schaller:2014uwa, Wang:2015jpa, Tollet:2015gqa}, there is 
still a significant uncertainty related to the modelling of the underlying processes.
In light of this situation, there is a considerable degree of freedom concerning halo models
and the DM density profiles, and a code computing observables related to DM should be
able to fully explore this freedom. 


%%%%%%%%%%%%%%%%%%%%%%%%%%%%%%
\subsection{Rescaling of the WIMP density}
\label{sec:rescale}

While it is natural to assume that the DM particles described by a given
particle module implemented in \ds\ make up most of the DM 
in our galaxy, they may also just constitute a sub-dominant part
of a multi-component realization of DM. What is more, there might 
be both a thermal contribution to the cosmological DM abundance
-- as computed by the relic density routines in \ds -- and a non-thermal
contribution, e.g.~via out-of-equilibrium production or via the decay
of heavier particles.
%
In this context, it is important to remember that DM detection rates
only depend on the {\it local} DM density {\it of that particular DM candidate}. 
For the case of direct detection, e.g.,
the rate scales linearly with the local density {\it at earth}, while
for indirect detection it scales linearly or quadratically with the
local DM density {\it at the point of of decay or or annihilation}, respectively.

In previous versions of the code, the ratio of thermal relic abundance returned
by \code{dsrdomega} and observed cosmological DM abundance
was used to internally rescale the results from rate calculations. For the reasons
given above, this is not fully satisfactory and in any case obscures the origin
of this rescaling. Starting from \ds\ 6, this is therefore no longer the case. 
All rate routines now assume that the local DM density equals the local density
in the particular DM candidate realized in the particle module -- 
unless they explicitly take the local DM density as an input parameter 
(which in that case refers to the local density in the particles described by the
respective \ds\ module). An example for such an exception are the 
neutrino telescope routines, because the combined effect of DM capture and
annihilation makes the dependence on the local DM density more complicated.
In all other cases, e.g.~if DM rate routines just take a halo label as input,
the user has to make sure to rescale the rates, as described above, {\it by hand}
to reflect possible sub-dominant DM populations.




\section{Implementation in \ds}

The implementation of dark matter halo models and related quantities in the library 
\code{ds\_core} follows a new and highly flexible scheme, compared to earlier versions of the code,
avoiding pre-defined hardcoded functions. 
For convenience a few pre-defined options are provided, however these can be either complemented by other profiles
eventually needed, or the entire sample configuration can be simply replaced linking to a user-defined setup 
- in both cases without editing routines provided in this release of the code. 
A further improvement compared to previous versions of \ds\ is that different dark matter density profiles, possibly referring to 
different dark matter detection targets, can be defined at the same time: E.g., one can easily
switch back and forth from a computation of the local positron flux induced by dark matter annihilations or decays in the
Milky Way halo to the computation of the gamma-ray flux from an external halo or a Milky Way satellite within the same   
particle physics scenario. Finally the present implementation simplifies the task of keeping track of consistent definitions 
for related quantities, such as, e.g., a proper connection between the dark matter profile and the source function 
for a given dark matter yield (see Chapter \ref{ch:src/an_yield}), or calling within an axisymmetric coordinate system a spherically symmetric function 
(and preventing the opposite). 

At the basis of the implementation, there is the subroutine \code{dsdmsdriver} routine which acts as an interface to quantities related to
the dark matter density profiles. This routine must contain a complete set of instructions on how to retrieve the different observables: E.g.~it checks the scaling of the various DM source functions (Chapter \ref{ch:src/an_yield}
with the DM density $\rho_\chi$ -- namely $\rho$ for decaying dark matter and $\rho^2$
for dark matter pair annihilations (in case the effect of substructures is neglected) -- and passes this information
to the routines for propagations of charged cosmic rays in the Galaxy (Chapter \ref{ch:src/cr_axi}), 
%at a given position in the Galaxy, and 
assuming that such source function is axisymmetric; line of sight integration routines (needed, e.g. for the computation of gamma-ray fluxes, see Chapter \ref{ch:src/cr_gamma})
call this same subroutine, but may assume instead that the corresponding source function is spherically symmetric. The routine 
\code{dsdmsdriver} must contain the specification on whether the dark matter density profile is spherically symmetric 
or axisymmetric, and in the first case provide a consistent numerical output to both calls, in the latter return an error to the
second call (since a spherically symmetric profile was expected). It may also be useful to use the \code{dsdmsdriver} routine for 
initialization calls, for instance to set parameters for a given parametric density profile, and test calls, for instance to print
which dark matter profile is currently active within a set of available profiles. The input/output structure of the routine is rather
general, with the first entry being however fixed to an integer flag specifying the action of the routine; currently 
available values and relative action are implicitly defined (through integer variables 
%which are hopefully self-explanatory, 
such as `\code{idmddensph}' referring to the spherical dark matter density profile) in an include file and are global parameter. 
Such set of definitions should not be changed, but can be enlarged in case further profile-related quantities would be needed.

While a specific \code{dsdmsdriver} routine should match the user needs in the problem at hand, the present release provides
a sample version, illustrating the flexibility of the setup. In particular the version included in the library \code{ds\_core} assumes that the dark
matter density profile is spherically symmetric and does not include dark matter substructure; it allows to choose as dark matter 
profile one among three parametric profiles, namely the Einasto \cite{1965TrAlm...5...87E}, the Navarro-Frenk-White \cite{Navarro:1995iw}
and the Burkert \cite{XXX} profiles, or a profile interpolated from a table of values of the dark matter density at a given radius. 
Besides providing parameters as needed in case of parametric profiles, to complete the initialization of a profile one should also specify: 
{\sl i)} an inner truncation radius, namely some $r_{ic}$ fixing $\rho(r)=\rho(r_{ic})$ for any $r<r_{ic}$ (the choice has an impact on predictions
for dark matter rates only for very singular dark matter profiles; choosing a value which is not too small allows for a faster numerical 
convergence of some rate computations); {\sl ii)} an outer truncation radius, namely some $r_{oc}$ beyond which the profile is assumed to 
be zero; {\sl iii)} the distance from the observer of the center of the profile, corresponding to the Sun galactocentric distance only in case 
the profile refers to the Milky Way; {\sl iv)} whether it is a profile that refers to the Milky Way and hence for which rates that are Milky Way 
specific, such as the local contribution to antimatter fluxes, can be computed;  {\sl v)} whether it is a profile to be saved in a halo profile
database for later use. 

Regarding this last point, the code implements a procedure of associating the set of entries fully specifying a halo 
profile (namely the choice of the parametric profile, the corresponding parameters and and the entries {\sl i)}--{\sl v)} above) to a given {\it input 
label}, and this can be reloaded at any time when needed; in particular all indirect detection flux routines have the label among their 
input parameters, so that in case of several dark matter detection targets or several profiles for the same targets it becomes unambiguous 
which profile is being considered. On the other hand it may be the case that the user needs to loop over many different profiles without 
keeping track of all of them for later reuse (e.g., in a scan over parameter space in the estimate of line of sight integrals towards a 
dwarf satellite); in such case the profile can be defined as ``temporary", with only the latest set temporary profile available at any given time.
For halos that are stored in the halo profile database, one can save and/or read from disc tabulated quantities, such as, e.g., the Green 
function needed for the computation of the local positron flux, for temporary profiles tables can (or in some cases need) to be computed 
running the code but are overwritten any time the temporary profile is changed. While in the previous releases of \ds,  halo parameters 
were typically set via common blocks to be included in the main file, the default \code{dsdmsdriver.f} implements a procedure in which,
when initializing a given halo profile, parameters are given as an input in association with a corresponding parameter label, and profile 
settings are specified as character strings appearing within the profile label. 

Along with the default \code{dsdmsdriver.f} routine, which is unfortunately rather involved since it 
allows for several different options, in the present release we provide example main files which illustrate 
a few of the possible user needs.
Those are described in the `quick start' part of this manual, see Section \ref{sec:aux_ex}, and cover 
examples of how to use pre-defined halo profiles, read in tabulated profiles as well as how to create
completely new ones.

 

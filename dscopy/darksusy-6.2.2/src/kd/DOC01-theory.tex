%%%%%%%%%%%%%%%%%%%%%%%%%%%%%%%%%%%%%%%%%%%%%%%%%%%%%%%%%%%%%%%%%%%%%%
\section{Kinetic decoupling and microhalos (kd) -- theory}
\label{sec:kd}


Even after \emph{chemical decoupling}, which sets the DM relic density (see Section 
\ref{sec:Boltzmann}), DM  frequently scatters with the very abundant 
standard model particles and thereby stays in local thermal equilibrium with the 
heat bath until the temperature has dropped by another factor of between 10 and
 a few 1000; after \emph{kinetic decoupling}, even these scattering events cease and DM 
no longer interacts with standard model particles. Inhomogeneities 
in the DM density can only develop after this has happenened, and the DM particles have
sufficiently cooled down so that free streaming becomes negligible.
The scale of kinetic decoupling can therefore directly be translated 
into a cutoff in the power spectrum of (dark) matter fluctuations and thus the size of the 
smallest (at least when not taking into account primordial black holes) gravitationally 
bound objects in the universe.



\subsection{Kinetic decoupling}

The process of kinetic decoupling is governed by the full Boltzmann equation for the WIMP 
distribution function $f(\mathbf{x},\mathbf{p})$, which in 
a flat Friedmann-Robertson Walker spacetime reads
\begin{equation}
  \label{eq:fullboltz}
  E(\partial_t-H\,\mathbf{p}\cdot\nabla_\mathbf{p})\,f=C[f]\,,
\end{equation}
where $C[f]$ is known as the collision term.
The Boltzmann equation quoted in Eq.~(\ref{eq:Boltzmann}) for the description of 
the (chemical) DM freeze-out process is actually just the first moment of 
this expression, i.e.~obtained by integrating it over $\int d^3p$ (after dividing it by $E$). 
As was realized int \cite{Bringmann:2006mu,Bringmann:2009vf}, kinetic decoupling can be decribed to a very high precision 
by considering, instead, the \emph{second} moment of Eq.~(\ref{eq:fullboltz}). For this purpose,
one introduces the WIMP "temperature" $T_\chi$,
\begin{equation}
  \int\frac{d^3p}{(2\pi)^3}\mathbf{p}^2f(\mathbf{p})\equiv3\,m_\chi T_\chi n_\chi\,,
\end{equation}
as a parameter that characterizes the deviation from thermal equilibrium (for which 
$T_\chi=T$ holds). 
After kinetic
decoupling, the DM `temperature' will simply decrease due to the expansion of the universe and, 
for non-relativistic DM, scale like $T_\chi\propto a^{-2}$. It is thus natural
 to define the moment of decoupling as the transition between these two regimes \cite{Bringmann:2009vf}.
Allowing for the scattering partners to have a different temperature ($T_{\tilde\gamma}$) than the photons ($T$), this implies
\be
\label{TchiT}
\,T_\chi(T)=\left\{
\begin{array}{cc}
T_{\tilde\gamma}(T) & \mathrm{for}~T\gtrsim T_\mathrm{kd} \\
T_{\tilde\gamma} (T_\mathrm{kd})\left(\frac{a(T_\mathrm{kd})}{a(T)}\right)^2& \mathrm{for}~T\lesssim T_\mathrm{kd}
\end{array}
\right. 
\ee

For practical purposes, one may now further introduce
\begin{eqnarray}
   x &\equiv& m_\chi/T\,,\\
   y &\equiv& {m_\chi T_\chi}{s^{-2/3}}\,.
\end{eqnarray}
Multiplying Eq.(\ref{eq:fullboltz}) by $\mathbf{p}^2/E$, integrating it 
over $\mathbf{p}$ and keeping only the leading order terms\footnote{
For very early decoupling, next-order correction terms may become relevant \cite{Binder:2017rgn}.
}
 in $\mathbf{p}^2/m_\chi^2$ 
then results in \cite{Bringmann:2009vf,Bringmann:2016ilk}
\be
\label{dydx}
\frac{d\log y}{d\log x}=\left(1-\frac{1}{3}\frac{d \log g_{*S}}{d\log x}\right)
\frac{\gamma(T_{\tilde\gamma})}{H(T)}\left(\frac{y_\mathrm{eq}}{y}-1\right)\,.
\ee
Here, $g_{*S}$ is the number of effective entropy degrees of freedom, $y_\mathrm{eq}$ is the 
value of $y$ in thermal equilibrium and $\gamma$ denotes the momentum transfer rate, 
\be
\label{fT}
 \gamma (T_{\tilde\gamma})=\frac{1}{48\pi^3g_\chi T_{\tilde\gamma}m_\chi^3}
 \sum_i
 \int d\omega\,k^4
 \left(1\mp g_i^\pm\right)g_i^\pm(\omega)
  \mathop{\hspace{-12ex}\left|\mathcal{M}\right|^2_{t=0}}_{\hspace{4ex}s=m_\chi^2+2m_\chi\omega+m_\mathrm{\tilde\gamma}^2}\,,
\ee
where the sum runs over all DM scattering partners (counting, e.g., electrons and positrons separately), 
$k\equiv\left| \mathbf{k}\right|$ is their momentum and $\omega$ their energy. The $g_i$ denote the SM
distribution functions, which are assumed to be thermal (note, however, that no 
assumption has been made about the form of the WIMP distribution function $f$).
The scattering amplitude
squared in this expression, $|\mathcal{M}|^2$, is understood to be {\it summed} over all internal (spin or 
color) degrees of freedom, including initial ones. If it is not Taylor expandable around $t=0$, one has to 
make the replacement \cite{KasaharaPHD,Gondolo:2012vh} 
\be
\label{maverage}
 \mathop{\hspace{-12ex}\left|\mathcal{M}\right|^2_{t=0}}_{\hspace{4ex}s=m_\chi^2+2m_\chi\omega+m_\mathrm{\tilde\gamma}^2}
\!\!\! \longrightarrow
\left<\left|\mathcal{M}\right|^2\right>_t 
\!\!\equiv \frac{1}{8k^4}\int_{-4k^2_{CM}}^0
\!\!\!\! dt(-t)\left|\mathcal{M}\right|^2
\ee
in the above expression. We may easily check that the asymptotic behaviour 
for $T_\chi$ described by Eq.~(\ref{dydx}) is consistent with the expectation outlined above: 
At large $T$, we have $\gamma\gg H$, thus enforcing
$T_\chi=T$; when $T$ becomes small, the WIMPs completely decouple from the thermal 
bath and $y$ stays constant, i.e. $T_\chi\propto s^{2/3}\propto a^{-2}$. 
A practical way to determine the 
\emph{kinetic decoupling} temperature $T_{\rm kd}$ as defined in Eq.~(\ref{TchiT}) 
is by solving the above differential equation until $y$ stays constant (indicated by the `$x\to\infty$'):
\begin{equation}
  \label{eq:tkddef}
  x_{\rm kd}=\frac{m_\chi}{T_{\rm kd}}\equiv g_{\rm eff}(T_{\rm kd})\, \left.y\right|_{x\rightarrow\infty}\,.
\end{equation}





\subsection{The smallest protohalos}


Before kinetic decoupling, WIMPs are tightly coupled to the heat bath, so
any small-scale perturbations in the DM fluid would immediately be washed out.
For temperatures $T\lesssim T_{\rm kd}$, however, this is no longer the case and perturbations
 in the DM density start to devolop under the influence of gravity; however, the 
remaining viscous coupling between the two fluids and, subsequently, the free-streaming 
of the WIMPs generate an exponential cutoff in the power spectrum \cite{Green:2005fa}, 
with a characteristic comoving damping scale $k_{\rm fs}$. The WIMP mass contained in a 
sphere of the corresponding size is thus given by \cite{Bringmann:2009vf}
\begin{equation}
   M_{\rm fs}\approx\frac{4\pi}{3}\rho_\chi\left(\frac{\pi}{k_{\rm fs}}\right)^3
=4.0\times 10^{-6}\left(\frac{1+{\rm ln}\left(g_{\rm eff}^{1/4}T_{\rm kd}/30\;{\rm MeV}\right)/18.6}{\left(m_\chi/100\; {\rm GeV}\right)^{1/2} g_{\rm eff}^{1/4}\left(T_{\rm kd}/30\;{\rm MeV}\right)^{1/2}}\right)^3M_\odot\,.
\end{equation}
Acoustic oscillations also have to be taken into account as a damping mechanism 
and lead to a similar exponential cutoff in the power spectrum
\cite{Loeb:2005pm,Bertschinger:2006nq}. In this case, the characteristic damping mass 
 is given by the total amount of DM inside the horizon at the time of kinetic decoupling:
\begin{equation}
  M_{\rm ao}\approx\frac{4\pi}{3}\left.\frac{\rho_\chi}{H^3}\right|_{T=T_{\rm{kd}}}
  =3.4\times10^{-6}\left(\frac{T_{\rm kd}g_{\rm eff}^{1/4}}{50\,{\rm MeV}}\right)^{-3}M_\odot\,.
\end{equation}
Note that $T_{\rm kd}$ in the above expressions only holds when using the definition given in Eq. (\ref{TchiT}); 
for an alternative definition, the expected magnitude of the cutoff mass has to be correspondingly re-scaled.


In general, the actual cutoff in the power spectrum is  given by
$M_{\rm cut}=\max\left[M_{\rm fs},M_{\rm ao}\right]$; which of the two
physically independent damping mechanisms is more efficient depends on the 
particle nature of the WIMP. 
The expected mass for the smallest gravitationally bound objects in the 
universe is then also simply given by $M_{\rm cut}$. Numerically, the formation of 
protohalos with masses down to the cutoff scale has been confirmed
and their evolution could be 
followed until a redshift of $z\sim26$ \cite{Diemand:2005vz}; the further survival 
probabilities of these objects, however, as well as the resulting consequences for the indirect 
detection of DM, are subject to a presently still ongoing discussion.



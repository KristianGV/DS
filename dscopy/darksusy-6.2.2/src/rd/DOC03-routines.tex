%%%%%%%%%%%%%%%%%%%%%%%%%%%%%%%%%%%%%%%%%%%%%%%%%%%%%%%%%%%%%%%%%%%%
\section{Relic density -- routines}

In \ft{src/rd}, the general relic density routines are found. These
routines can be used for any dark matter candidate and e.g.\ the interface
to neutralino dark matter is in \ft{src\_models/mssm/rd}. 

The main routine for relic density calculations is \ft{dsrdomega}. It calculates and returns the relic density as well as the approximate temperature of freeze-out. It taks a few arguments: i) one \code{option} argument that typically (depending on the particle physics module) determines how co-annihilations are included, ii) one \code{fast} argument that determines how careful the Boltzmann solver should be in calculating the relic density. See the header of \ft{dsrdomega} for more details.
%\index{relic density}


%%%%%%%%%%%%%%%%%%%%%%%%%%%%%%%%%%%%%%%%%%%%%%%%%%
\subsection{Global parameters}

All internal settings of the relic density routines are set in common
blocks in \ft{dsrdcom.h}. The most important parameters that can be
changed by the user are

\begin{sub}{Important parameters in \ftb{dsrdcom.h}}
  \itit{Purpose:} Provide a set of parameters, with which the internal
  behaviour of the relic density routines can be changed.
  \itit{Parameters}
  \itv{tharsi}{i} Size of the coannihilation, resonance and threshold
    arrays (default=50). Increase this size if you have more than 50
    coannihilating particles, more than 50 resonances or more than 50
    thresholds.
  \itv{rdluerr}{i} Logical unit number where error messages are
    printed.
  \itv{rdtag}{c*12} Idtag that is printed in case of errors.
  \itv{cosmin}{r8} \ldots
  \itv{waccd}{r8} \ldots
  \itv{dpminr}{r8} \ldots
  \itv{dpthr}{r8} \ldots
  \itv{wdiffr}{r8} \ldots
  \itv{wdifft}{r8} \ldots
  \itv{hstep}{r8} \ldots
  \itv{rdt\_max}{r8} Maximum number of seconds to spend on relic density calculation. If the time limit is exceeded, \ft{dsrdomega} returns with an error flag and the result 0. The time is the total CPU time (i.e.\ summed up over all cores/threads) and the limit is approximate as it is only checked before a new point is added to the $W_{\rm eff}$ tabulation.
\end{sub}


%%%%%%%%%%%%%%%%%%%%%%%%%%%%%%%%%%%%%%%%%%%%%%%%%%
\subsection{Brief description of the internal routines}

Below, the remaining routines related to the relic density calculation
are briefly mentioned. For more details, we refer to the routines
themselves.

\begin{brief-subs}
\bsub{dsrdaddpt}
  To add one point in the $W_{\rm eff}$-$p_{\rm eff}$ table.
\bsub{dsrdcom}
  To initialize parameters in the common blocks in \ft{dsrdcom.h}. If
  you want to change these parameters yourself, include \ft{dsrdcom.h}
  in your code and change the parameters you want.
\bsub{dsrddof}
  Returns the degrees of freedom as a
  function of the temperature in the early Universe.
\bsub{dsrddpmin}
  To return the allowed minimal distance in $p_{\rm
  eff}$ between two points in the $W_{\rm eff}$-$p_{\rm eff}$ plane.
  The returned value depends on if there is a resonance present or not
  at the given $p_{\rm eff}$.
\bsub{dsrdens} Routine to solve the Boltzmann equation and return the relic density. Called by \ft{dsrdomega}.
\bsub{dsrdeqn}
  To solve the relic density equation by means of an
  implicit trapezoidal method with adaptive stepsize and termination.
\bsub{dsrdfunc}
  To return the invariant annihilation rate times the
  thermal distribution.
\bsub{dsrdfuncs}
  To provide \ft{dsrdfunc} in a form suitable for
  numerical integration.
\bsub{dsrdlny}
  To return $\ln W_{\rm eff}$ for a given $p_{\rm
  eff}$.
\bsub{dsrdnormlz}
  To return a unit vector in a given direction.
\bsub{dsrdqad}
  To calculate the relic density with a
  quick-and-dirty method. It uses the approximative expressions in
  Kolb \& Turner with the cross section expaned in $v$.
\bsub{dsrdqrkck}
  To numerically integrate a function with a
  Runge-Kutta method
\bsub{dsrdrhs}
  To calculate terms on the right-hand side in the
  Boltzmann equation.
\bsub{dsrdset}
  To set the control parameters for the relic density calculation. Currently, only the choice of effective degrees of freedom is implemented through \ft{dsrdset}; the other parameters are passed as arguments to \ft{dsrdomega}.
\bsub{dsrdspline}
  To set up the table $W_{\rm eff}$-$p_{\rm eff}$ for
  spline interpolation.
\bsub{dsrdstart}
  To sort and store information about coannihilations,
  resonances and thresholds in common blocks.
\bsub{dsrdtab}
  To set up the table $W_{\rm eff}$-$p_{\rm eff}$.
\bsub{dsrdthav}
  To calculate the thermally averaged annihilation
  cross section at a given temperature.
\bsub{dsrdthclose}
  ...
\bsub{dsrdthlim}
  To determine the end-points for the thermal
  average integration.
\bsub{dsrdthtest}
  To check if a given entry in the $W_{\rm
  eff}$-$p_{\rm eff}$ table is at a threshold.
%\bsub{dsrdwdwdcos}
%  To write out a table of $dW_{\rm eff}/d\cos \theta$
%  as a function of $\cos \theta$ for a given $p_{\rm eff}$.
\bsub{dsrdwfunc}
  To write out \ft{dsrdfunc} for a given $x=m_\chi/T$.
\bsub{dsrdwintp}
  To return the invariant rate $W_{\rm eff}$ for any
  given $p_{\rm eff}$ by performing a spline interpolation in the
  $W_{\rm eff}$-$p_{\rm eff}$ table.
\bsub{dsrdwintpch}
  To check the spline interpolation in the $W_{\rm
  eff}$-$p_{\rm eff}$ table and compare with a linear interpolation.
\bsub{dsrdwintrp}
  To write out a table of the invariant rate $W_{\rm
  eff}$ and some internal integration variables and expressions.
\bsub{dsrdwres}
  To write out the table $W_{\rm eff}$-$p_{\rm eff}$.
\end{brief-subs}

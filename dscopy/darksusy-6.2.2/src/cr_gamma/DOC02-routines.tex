%%%%%%%%%%%%%%%%%%%%%%%%%%%%%%%%%%%%%%%%%%%%%%%%%%%%%%%%%%%%%%%%%%%%

\section{Gamma rays from the halo -- routines}

\ds\  provides the functions \code{dscrgaflux\_dec} and \code{dscrgaflux\_v0ann} that take $J^{\rm dec}$ 
(or $J^{\rm ann}$)  as input and return the fluxes given in  Eqs.~(\ref{eq:phidec}) and (\ref{eq:phiann}). Here, the subscript \code{\_v0ann} refers to the fact that, for the
purpose of those routines, $S_2$ is evaluated
in the limit of vanishing relative velocity of the annihilating DM pair. While this is the only situation of practical 
interest in many DM models, future \ds\ versions will offer support for a full velocity dependence of this quantity.
In analogy with \code{dscrsource\_line}, the \code{core} library furthermore provides
routines \code{dscrgaflux\_line\_dec} and \code{dscrgaflux\_line\_v0ann} (and correspondingly for neutrinos) 
that return strength, width and location of {\it monochromatic} (`line') signals. The latter is a convenient
change with respect to previous versions of the code,  as one cannot generally know how many lines there
are (this depends on the particle physics module).

If a halo label is available, one can also 
more conveniently call \code{dsgafluxsph} instead. This function  directly returns the gamma-ray flux,
for that halo, 
automatically calculating the required line-of-sight integrals and adding decaying and annihilating DM
components depending on which particle model is initialized. 

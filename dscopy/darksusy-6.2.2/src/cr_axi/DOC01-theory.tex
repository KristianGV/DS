%%%%%%%%%%%%%%%%%%%%%%%%%%%%%%%%%%%%%%%%%%%%%%%%%%%%%%%%%%%%%%%%%%%%%%
\section{Cosmic ray propagation in axially symmetric halos}

There is a clean asymmetry between particles and antiparticles in the standard cosmic ray picture:
The bulk of cosmic rays -- protons, nuclei and electrons -- are mainly ``primary" species, i.e. particles
accelerated in astrophysical sources and then copiously injected in the interstellar medium;
``secondary" components, including antimatter, are instead produced in the interaction of primaries 
with the interstellar
medium during the propagation. It follows that there is a pronounced deficit of antimatter compared to matter 
in the locally measured cosmic ray flux (about 1 antiproton in $10^4$ protons). When considering instead a
source term due to DM annihilations or decays, a significant particle-antiparticle asymmetry is in general
not expected, and antiprotons, positrons and antideuterons turn out to be competitive indirect DM probes.

Charged particles propagate diffusively through the regular and turbulent components of Galactic magnetic 
fields. This makes it more involved for local measurements to track spectral and morphological imprints of 
DM sources than, e.g., for the gamma-ray and neutrino channels (though searches for spectral features
in CR positron fluxes still lead to very competitive limits \cite{Bergstrom:2013jra}).
In fact the transport of cosmic rays in the Galaxy is still a debated
subject: Most often one refers to the quasi-linear theory picture (with magnetic inhomogeneities as a perturbation 
compared to regular field lines) in which propagation can be described in terms of a (set of) equation(s) linear
in the density of a given species, containing terms describing diffusion in real space, diffusion in momentum
space (the so-called reacceleration), convective effects due to Galactic winds, energy or fragmentation
losses and primary and secondary sources (see, e.g., Ref.~\cite{Strong:2007nh} for a review).
Dedicated codes have been developed to solve numerically this transport equation, including GALPROP~\cite{Strong:1998pw}, DRAGON~\cite{Evoli:2016xgn} and PICARD~\cite{Kissmann:2014sia}. 

Here we follow instead a semi-analytical approach, analogous to that developed for the USINE 
code~\cite{Maurin:2001sj}. In particular, we model
the propagation of antiprotons and antideuterons by considering the steady state equation~\cite{Bergstrom:1999jc}
\begin{equation}
 \frac{\partial{N}}{\partial{t}} = 0 = \nabla \cdot
 \left(D\,\nabla N\right)
 - \nabla \cdot \left( \vec{u}\,N \right)
 - \frac{N}{\tau_N} + Q\,.
\label{eq:pbardiff}
\end{equation}
We solve it for situations where {\it i)} the diffusion coefficient $D$ can have an arbitrary dependence on the particle 
rigidity but can at most take two different values in the Galactic disc and in the diffusive halo, {\it ii)} the 
convective velocity $\vec{u}$ has a given fixed modulus and is oriented outwards and perpendicular to the disc,
{\it iii)} the loss term due to inelastic collisions has an interaction time $\tau_N$ which is energy dependent 
but spatially constant in the disc and going to infinity in the halo (corresponding to a constant target gas 
density in the disc and no gas in the halo), {\it iv)} the DM
source $Q$ is spatially axisymmetric and has a generic energy dendence. Under these
approximations and assuming, as is usually done, that the propagation volume is a cylinder centred at the disc 
and that particles can freely
escape at the boundaries of the diffusion region, Eq.~(\ref{eq:pbardiff}) can be solved analytically by expanding $N$
in a Fourier-Bessel series; the computation of the flux involves, at each energy, a sum over the series of 
zeros of a Bessel function of first kind and order zero, and a volume integral of the spatially dependent part in the axisymmetric
source term $Q$
 (basically the DM density $\rho_\chi$ for decaying DM and its square for pair annihilating DM) 
times a
weight function depending on the given zero in the series (see~\cite{Bergstrom:1999jc} for further details).

Since the path lengths for antiprotons and antideuterons are rather large, of the order of a few kpc, taking average
values for parameters in the transport equation rather than the more realistic modelling that can be implemented
in full numerical solutions, has no large impact in case of extended and rather 
smooth sources such as for DM. 
Eq.~(\ref{eq:pbardiff}) neglects reacceleration effects, which may in general be relevant at low energies
(rigidities below a few GV); however even this does not have a large impact in case of the species at hand,
see, e.g.,~\cite{Evoli:2011id} for a comparison of results with numerical and semi-analytical solutions for cosmic-ray
antiprotons. The power of our semi-analytic approach is that one can store values of 
the solution of the transport equation obtained by assuming 
a given mono-energetic source -- provided by 
the functions \code{dspbtdaxi} and \code{dsdbtdaxi} for, respectively, antiprotons and antideuterons -- and 
then apply these as weights to any particle source term $\mathcal{S}_n(E_f)$ as introduced above.
For antiprotons and antideuterons, this latter step is done in the functions that compute the local galactic differential 
fluxes from DM annihilation and decay, \code{dspbdphidtaxi} and \code{dsdbdphidtaxi}, respectively.
Note that while the outputs of \code{dspbtdaxi} and \code{dsdbtdaxi}  are labelled ``confinement time" in the code,
since they do have a dimension of time and scale the dependence between source and flux, one cannot trade
them for what is usually intended as confinement time for standard cosmic ray components, given that the
morphology of the DM source is totally different from supernova remnant distributions usually implemented
for describing ordinary primary components.

The structure we implemented gives a particularly clear advantage when the code is used to scan over many 
particle physics DM models, but only over a limited number of propagation parameters and DM density profiles.
For such an application,  it is useful to tabulate the 'confinement times' (returned by \code{dspbtdaxi} and 
\code{dsdbtdaxi}) over a predefined range of energy; this is
done in the functions \code{dspbtdaxitab} and \code{dsdbtdaxitab} when calling the flux routines with an appropriate
option (and only in case the DM halo profile currently active is within the halo profile database). 
Such tabulations  can be saved and re-loaded for later use; 
here the proper table association is ensured by 
a propagation parameter label setting system in analogy to the one implemented for the halo profile database. 
Computing such a table on the first call is rather CPU  consuming, especially for DM profiles that are 
singular towards the Galactic center, so in case only a few flux
computations are needed it may be better to switch off the tabulation option; this is true also in case the flux is
needed at a small number of energy values, since the latest 100 (non-equivalent) calls to  
\code{dspbtdaxi} and \code{dsdbtdaxi}
are stored in memory (with the corresponding propagation parameters and halo model correctly associated).

The case for positrons is treated analogously, except that energy losses and spatial diffusion are
the most important effects for propagating cosmic ray leptons. The transport equation we solve 
semi-analytically therefore has the form~\cite{Baltz:1998xv}
\begin{equation}
 \frac{\partial{N}}{\partial{t}} = 0 = \nabla \cdot
 \left(D\,\nabla N\right)
 + \frac{\partial}{\partial{p}} \left( \frac{dp}{dt} N \right) + Q\,,
\label{eq:eplusdiff}
\end{equation}
where the functional form of $D$ and $Q$ can be chosen as for antiprotons and antideuterons, 
and the energy loss rate $dp/dt$ can have a generic
momentum dependence but needs again to be spatially constant. Assuming the same topology for
the propagation volume and free escape conditions at the vertical boundaries (to compute the local positron flux the radial boundary turns out to be irrelevant) ,
the solution of the propagation equation is given in terms of
a Green's function in energy (the function  \code{dsepgreenaxi} in the code) to be convoluted over the
source energy spectrum at emission for a given particle DM candidate. This last step is performed
by the function \code{dsepdndpaxi} returning the local positron number density,
while \code{dsepdphidpaxi} converts it to a flux and is the function which should be called from the main file.
The method to implement this solution is a slight generalization of the one described~\cite{Baltz:1998xv} and
generalizes the one introduced in~\cite{Colafrancesco:2005ji}
for a spherically symmetric configuration to an axisymmetric system.

The computation of the Green's function involves a volume integral over the spatially dependent part of
the DM source function $Q$ (again basically the DM density $\rho_\chi$ for decaying DM and its square for pair
 annihilating DM) 
and   the implementation via the so-called method of image charges (again a sum over a series)
of the free escape boundary condition. 
It can again be CPU expensive for singular halo profiles, but its
tabulation is always needed since the Green function appears in a convolution. The main limiting factor with
respect to full numerical solutions is that one is forced to assume an (spatially) average value for $dp/dt$. 
However this has not a severe
impact on our results for the local DM-induced positron flux since, especially for energies above 10 GeV, 
the bulk of the DM contribution to the local flux stems from a rather
close-by emission volume; one thus just has to make sure to normalize $dp/dt$ to the mean value for 
{\it local} energy losses, as opposed to the mean value in the Galaxy, which are mainly due to the 
synchrotron and inverse Compton processes.

While the transport equations (\ref{eq:pbardiff}) and (\ref{eq:eplusdiff}) are essentially the same as considered
in previous releases of the code, their implementation in the present release is completely new and appears to be
numerically more stable. In particular cases with very singular DM profiles still give numerically accurate
results and converge faster
(in case of antiprotons and antideuterons implementing a procedure which applies to point sources).
Note however that the case of very singular DM profiles is also the one in which
 our models or 
{\it any} propagation model is subject to a significant uncertainty related to the underlying
physics,
since propagation in the
Galactic center region is difficult to model and probably rather different from what can be tested in the
local neighbourhood by measurements of primary and secondary cosmic rays. Finally, the 
implementation starting from \ds\ 6 is more flexible regarding parameter choices, such as for the 
rigidity scaling of the diffusion
coefficient and energy scaling of energy losses, in a framework which is now fully consistent for
antiprotons, antideuterons and positrons.

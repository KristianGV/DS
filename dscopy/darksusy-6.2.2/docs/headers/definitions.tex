%Mathematics
%\def\lrpartial{\raise1pt\hbox{$\mathord{\stackrel{\leftrightarrow}{\partial}}$}}

%New margins
%\setlength{\oddsidemargin}{0 cm}
%\setlength{\evensidemargin}{0 cm}
%\setlength{\topmargin}{-1.0 cm}
%\setlength{\textwidth}{16.0cm}
%\setlength{\textwidth}{15.0cm}
%\setlength{\textheight}{23.0cm}
%\setlength{\textheight}{21.0cm}
\setlength{\unitlength}{1cm}


\newcommand{\paolo}[1]{{\color{blue}\bf (PG: #1)}}
\newcommand{\joakim}[1]{{\color{red}\bf (JE: #1)}}
\newcommand{\lars}[1]{{\color{green}\bf (LBE: #1)}}
\newcommand{\piero}[1]{{\color{cyan}\bf (PU: #1)}}
\newcommand{\tb}[1]{{\color{magenta}\bf (TB: #1)}}



\newcommand{\beq}{\begin{equation}}
\newcommand{\eeq}{\end{equation}}
\newcommand{\bea}{\begin{eqnarray}}
\newcommand{\eea}{\end{eqnarray}}
\newcommand{\beqa}{\begin{eqnarray}}
\newcommand{\eeqa}{\end{eqnarray}}
\newcommand{\app}[3]{Astropart.\ Phys.\ {\bf #1} (#2) #3}
\newcommand{\hepex}[1]{{\tt hep-ex/#1}}
\newcommand{\hepph}[1]{{\tt hep-ph/#1}}
\newcommand{\astroph}[1]{{\tt astro-ph/#1}}
\newcommand{\prep}[3]{Phys.\ Rep.\ {\bf #1} (#2) #3}
\newcommand{\plb}[3]{Phys.\ Lett.\ {\bf B#1} (#2) #3}
\newcommand{\npb}[3]{Nucl.\ Phys.\ {\bf B#1} (#2) #3}
\newcommand{\ibid}[3]{{\em ibid.}\ {\bf B#1} (#2) #3}
\newcommand{\cpc}[3]{Comm.\ Phys.\ Comm.\ {\bf #1} (#2) #3}
\newcommand{\prl}[3]{Phys.\ Rev.\ Lett. {\bf #1} (#2) #3}
\newcommand{\apj}[3]{Astrophys.\ J.\ {\bf #1} (#2) #3}
\newcommand{\prd}[3]{Phys.\ Rev.\ {\bf D#1} (#2) #3}
\newcommand{\rmp}[3]{Rev.\ Mod.\ Phys.\ {\bf #1} (#2) #3}
%\newcommand{\href}[2]{#1}
\newcommand{\email}[1]{\tt #1}
\newcommand{\sigv}{\langle\sigma v\rangle_{\rm tot}}
\newcommand{\taue}{\tau_E}
\newcommand{\taud}{\tau_D}
\newcommand{\deltav}{\Delta v}
\newcommand{\mct}{{\tilde{m}_\chi}}
\newcommand{\anl}{\\[1ex]}
\newcommand{\tabspace}{\\[-2.5ex]}

\def\rn{\noindent\parshape 2 0truecm 8.5truecm 0.3truecm 8.2truecm}
\def\rn{}% NAME STYLE: A. E. Neumann
\def\nn#1 #2{#2. #1}                            % Name with 1 initial
\def\nnn#1 #2 #3{#2. #3. #1}                    % Name with 2 initials
\def\nnnn#1 #2 #3 #4{#2. #3. #4. #1}            % Name with 3 initials
\def\nnnnn#1 #2 #3 #4 #5{#2. #3. #4. #5. #1}    % Name with 4 initials
% AUTHOR SEPARATION STYLE: "first and second", "first, second, and third"
\def\dualand{ and\hbox{ }}
\def\multiand{, and\hbox{ }}
%\def\multiand{ and,\hbox{ }}
% JOURNAL ARTICLE STYLE:
\def\rf#1;#2;#3;#4;#5 {{\frenchspacing\par\rn#1, #3 {\bf #4}, #5 (#2). \par}}
% BOOK STYLE:
\def\rfbook#1;#2;#3;#4;#5 {{\frenchspacing\par\rn#1, {\it #3} (#5, 
#4, #2).\par}}
% PREPRINT STYLE:
\def\rfprep#1;#2;#3 {{\frenchspacing\rn#1, #3 (#2);\ }}
\def\rfprepend#1;#2;#3 {{\frenchspacing\rn#1, #3 (#2).}}
%\def\rfprep#1;#2;#3 {{\par\frenchspacing\rn#1, Report  No. #3, #2 
%(unpublished).\par}}

%Front page
\def\frontmatter{}
\def\preprint#1{{\flushright #1 \endflushright}}
\def\title#1{{\center\LARGE\bf{#1}\endcenter}}
\def\author#1{\vskip2\baselineskip{\center\Large#1\endcenter}}
\def\affil#1{\vskip\baselineskip{\center{\normalsize\em #1}\endcenter}}
\def\abstract{\vskip\baselineskip\hrule\section*{Abstract}}
\def\endabstract{\relax}
\def\keywords{\vskip\baselineskip\noindent\emph{Key words: }}
\def\endkeywords{}
\def\endfrontmatter{\vskip\baselineskip\hrule}

\newcounter{comnum}
\setcounter{comnum}{1}
\renewcommand{\thecomnum}{\arabic{comnum}}
\newcommand{\comment}[1]{{\sffamily \bfseries \color{red} COMMENT \#\thecomnum: #1
\addtocounter{comnum}{1}}}
\newcommand{\mcomment}[1]{\marginpar{\small \sf COMMENT \#\thecomnum:\\
#1}\addtocounter{comnum}{1}}


\newcommand{\code}[1]{\ft{#1}}
\newcommand{\codeb}[1]{\ftb{#1}}

\newcommand{\ds}{{\sffamily DarkSUSY}}
% Todo text, prio 1-3, where one is most important
\newcommand{\todoone}[1]{{\color{red}\sffamily \bfseries TODO (prio 1): #1}}
\newcommand{\todotwo}[1]{{\color{magenta}\sffamily \bfseries TODO (prio 2): #1}}
\newcommand{\todothree}[1]{{\color{blue}\sffamily \bfseries TODO (prio 3): #1}}

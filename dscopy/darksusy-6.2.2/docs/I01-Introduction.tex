%%%%%%%%%%%%%%%%%%%%%%%%%%%%%%%%%%%%%%%%%%%%%%%%%%%%%%%%%%%%%%%%%%%%
%%%%%%%%%%%%%%%%%%%%%%%%%%%%%%%%%%%%%%%%%%%%%%%%%%%%%%%%%%%%%%%%%%%%
%%%%%%%%%%%%%%%%%%%%%%%%%%%%%%%%%%%%%%%%%%%%%%%%%%%%%%%%%%%%%%%%%%%%
\chapter{Introduction}

\ds\ is a set of Fortran routines to allow advanced numerical calculations
connected to dark matter physics. In part I of this Manual we explain the
basic structure of the code, and provide an introduction on how to get
quickly started and use it. Part II is devoted to a description of the 
\ds\ {\it master library}, which contains all routines that are completely independent
of the underlying DM particle physics.
In part III, we describe the additional libraries, or {\it particle modules}, for 
specific DM models that are currently implemented (including
supersymmetric dark matter in the Minimal Supersymmetric Standard
Model, the MSSM). The modular structure of the code allows to easily
link the master library to any of those particle modules, as well as to add
user-defined new modules in a straight-forward way.

The physics involved is covered in the \ds\ papers
\cite{ds4,ds6}. In this manual we will mainly cover the more techincal
aspects of \ds, i.e.~how to call different subroutines, both particle-physics 
dependent and not, and how to change various switches and options in more
advanced routines. We will only briefly review the necessary
physics involved when needed and refer the reader to \cite{ds4,ds6}
and the original papers behind \ds\ (see Section \ref{sec:DS_papers}) for more
details. If you use \ds\, please consider the original physics work
behind and give proper credit to \cite{ds6} and the relevant
references in Section \ref{sec:DS_papers}. If you use non-standard options,
e.g.\ a different propagation model for antiprotons, please remember
to give proper credit to that model.

\comment{We don't follow the style conventions, so I have simply commented
them out for the moment}

%\newpage
%%%%%%%%%%%%%%%%%%%%%%%%%%%%%%%%%%%%%%%%%%%%%%%%%%%%%%%%%%%%%%%%%%%%
%\subsection*{Style conventions}
%%%%%%%%%%%%%%%%%%%%%%%%%%%%%%%%%%%%%%%%%%%%%%%%%%%%%%%%%%%%%%%%%%%%

%In an attempt to keep this manual reasonably easy to follow we will
%need to specify our notation.  We will use the following convention
%for fonts,
%\begin{sub}{Convention for fonts}
%  \itv{\rmfamily text}{} This font is used for normal text.
%  \itv{variable}{} This font is used for variables.
%  \itv{\ftb{routine}}{} This font is used for subroutine or function
%  names or for header file names.
%  \itv{\tt dump}{} This font will be used for screen dumps of outputs.
%  \itv{\ttfamily \em input}{} This font will be used for user
%  input, i.e.\ where you are supposed to write something.
%\end{sub}

%\medskip

%\emph{Subroutines} and \emph{functions} always reside in a file with the
%same name as the subroutine/function. Routines that belong together
%are put in separate subdirectories in the \ft{src} or \ft{src\_models} directory. 
%An individual description of each routine is provided in the long version of the manual
%(by extracting the routine headers automatically from the code).

%Subroutines and functions will be described with the following structure
%\begin{sub}{subroutine \ftb{example}(in1,in2,in3,in4,in5,in6,in7,out1)}
%  \itit{Purpose:} Here the routine will be explained.
%  \itit{Inputs:}
%  \itv{in1}{i} This is an input argument, declared as \ft{integer}.
%  \itv{in2}{r} This is an input argument, declared as \ft{real}.
%  \itv{in3}{r8} This is an input argument, declared as \ft{real*8}.
%  \itv{in4}{c} This is an input argument, declared as \ft{complex}.
%  \itv{in5}{c16} This is an input argument, declared as \ft{complex*16}.
%  \itv{in6}{ch2} This is an input argument, declared as \ft{character*2}.
%  \itv{in7}{ch*} This is an input argument, declared as \ft{character*(*)}.
%  \itit{Outputs}
%  \itv{out1}{r8} This is an output argument, declared as \ft{real*8}
%\end{sub}
%where the shorthand notation for the type of the arguments is
%indicated. For functions, the type is indicated on the first line,
%\begin{sub}{function \ftb{fun}(arg) \hfill r8}
%  \itit{Purpose:} Here the function will be explained.
%  \itit{Inputs:}
%  \itv{arg}{i} This is an input argument, declared as \ft{integer}.
%\end{sub}
%i.e., in this case the function is declared as \ft{real*8}.

%\medskip

%\emph{Common blocks} are declared in header files in the \ft{src/include}
%subdirectory, as well as in several individual \ft{include} subdirectories in
%\ft{src\_models}. When discussing switches and parameters in common blocks we
%will, instead of describing the common blocks in detail, 
%mention which header file they reside in. If you want to access these
%variables, you should then include the corresponding header
%file. E.g., it can look like this
%\begin{sub}{Example parameters in \ftb{headerfile.h}}
%  \itit{Purpose:} Description of this set of variables.
%  \itv{par1}{r8} Description of a \ft{real*8} parameter.
%\end{sub}

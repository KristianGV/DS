%%%%%%%%%%%%%%%%%%%%%%%%%%%%%%%%%%%%%%%%%%%%%%%%%%%%%%%%%%%%%%%%%%%%
\chapter{Comparison to previous \ds\ versions}
\label{ch:translation}

For those \ds\ users who are familiar with previous version of the code, the main 
structural difference introduced with \ds\ 6.0 is a highly modular setup that allows to 
fully disentangle the astrophysics-related parts from those that rely on a specific particle
physics model (see Chapter 
\ref{ch:philosophy} for a detailed description). There are also many new or significantly 
improved physics capabilities introduced after version 5 and 4 \cite{ds4} -- including 
electroweak and strong corrections to DM annihilation, improved routines to solve
the Boltzmann equation for chemical and kinetic decoupling, and a new framework to handle 
the propagation of cosmic rays. For a detailed list of those new features, we refer to the 
publication describing this release of the code \cite{ds6}.


One technical aspect that has changed are the particle codes ($k$-variables), which are 
now treated as (module-)internal codes. They can be used by the particle physics module if the 
module so wishes, but the interface functions and routines in \code{ds\_core} instead 
use PDG codes when referring to particles.

In the course of re-organizing the code, it also became necessary to re-name some of the 
basic functions and subroutines that existed in earlier versions and which users may have
become familiar with. For convenience, we therefore list below, in Table \ref{tab:translation},
 the most commonly used functionalities in version 4 and 5 that have changed in name or 
 usage with the present version of the code.


%%%%%%%%%%%%%%%%%%%%%%%%%%
%\begin{table}[t!]
%\centering
%\begin{tabular}[l]{lll}
\captionsetup{width=\textwidth}
\begin{longtable}{ p{.33\textwidth}  p{.22\textwidth}  p{.35\textwidth} } 
\\[1ex]
routine name up to \ds\ 5 & new routine name & comments\\
\hline
\code{dshmset} & \code{dsdmdset\_halomodel} &  \parbox[t]{5.5cm}{Setting up and refering to DM
halos has fundamentally changed. See Section \ref{sec:halo_general} for an overview and all Chapters with names
containing  \code{src\_dmd} in Part II of the manual for more details.} \vspace*{1.5ex}\\
%
\code{dshmj} & \code{dsjfactor} &  \parbox[t]{5.5cm}{Line-of-sight integrals now take a halo label as input, and can be
computed for various objects at the same time.} \vspace*{1.5ex}\\
%
\code{dssusy} & [\code{dsmodelsetup}] &  \parbox[t]{5.5cm}{Setting up a model (calculating the mass spectrum,  
relevant 3-particle vertices etc.) now depends on the particle module implementation.} \vspace*{1.5ex}\\
%
\parbox[t]{4cm}{\code{dsmhtkd}\\\code{dsmhmcut}} & \parbox[t]{4cm}{\code{dskdtkd}\\\code{dskdmcut}} 
&  \parbox[t]{5.5cm}{The 'microhalo' routines are now more properly referred to as 'kinetic decoupling' routines.} \vspace*{1.5ex}\\
%
\code{dshmrescale\_rho} & ---& \parbox[t]{5.5cm}{A mismatch between local halo density and DM abundance 
for a given \ds\ module is no longer hidden as a rescaling factor in a common block. Instead, such factors 
typically enter as explicit parameters in direct and indirect detection routines.} \vspace*{1.5ex}\\
%
\code{dshaloyield} & [\code{dsanyield}] & \parbox[t]{5.5cm}{total cosmic ray yield from {\it neutralino} annihilation}
\vspace*{1.5ex}\\
%
\code{dshayield} & \code{dsanyield\_sim} &  \parbox[t]{5.5cm}{simulated cosmic ray yields from individual annihilation/decay channels to SM particles. 
Note that internal channel codes are now replaced with PDG codes of the final state particles as input.}
\vspace*{1.5ex}\\
%
\code{dshrgacontdiff} & \code{dsgafluxsph} &  \parbox[t]{5.5cm}{Gamma-ray flux routines now work seamlessly together with both the new halo setup and the modular particle physics structure.}
\vspace*{1.5ex}\\
%
\code{dshrgaline} & \parbox[t]{4cm}{\code{dscrgaflux\_line\_v0ann}\\\code{dscrgaflux\_dec}} & \parbox[t]{5.5cm}{Gamma-ray line routines now return both number, energies and widths of all such signals that are present in the current particle setup.}
\vspace*{1.5ex}\\
%
\parbox[t]{4cm}{\code{dshrpbardiff}\\\code{dshrdbardiff}\\\code{dsepdiff}} 
& \parbox[t]{4cm}{\code{dspbdphidtaxi}\\\code{dsdbdphidtaxi}\\\code{dsepdphidpaxi}} 
&  \parbox[t]{5.5cm}{Cosmic-ray propagation routines have been re-written 
from scratch. They are now much more flexible and can be used for any axisymmetric halo/diffusion model.}
\vspace*{1.5ex}\\
%
\code{dsntrates} & \code{dssenu\_rates} &  \parbox[t]{5.5cm}{Neutrino rates from inside the sun or earth}
\vspace*{1.5ex}\\
%
\code{dshrmuhalo} &  \parbox[t]{4cm}{\code{dscrmuflux\_v0ann}\\\code{dscrmuflux\_v0ann}} 
&  \parbox[t]{5.5cm}{Neutrino-induced muon flux from the halo
(for annihilating and decaying DM, respectively).}
\vspace*{1.5ex}\\
%

\hline
%\end{tabular}
\caption{`Translation table' for how the most commonly used functionalities in \ds\ version 5 and earlier have changed with the
present version of the code. Routines in parentheses, e.g.  [\code{dsanyield}], are no longer part of the \ds\ main library and only 
provided by specific particle physics modules. For more detailed descriptions of the new routines and functions, see the headers of the respective 
files. 
}
\label{tab:translation}
\end{longtable}
%\end{table}
%%%%%%%%%%%%%%%%%%%%%%%%%%		


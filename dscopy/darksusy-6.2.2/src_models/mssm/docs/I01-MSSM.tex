%%%%%%%%%%%%%%%%%%%%%%%%%%%%%%%%%%%%%%%%%%%%%%%%%%%%%%%%%%%%%%%%%%%%
\chapter{The Minimal Supersymmetric Standard Model}

The implementation of the minimal supersymmetric standard model (MSSM)
in the module \code{mssm} mainly follows that of \ds versions 4 \cite{ds4}
(see also Section \ref{sec:ge} in this manual). 
In particular, the conventions 
for the superpotential and soft supersymmetry-breaking potential are the same as 
implemented in \cite{Bergstrom:1995cz}, and thus similar to \cite{Haber:1984rc,Gunion:1984yn}.
The full set of input parameters to be provided at the weak scale thus consists of the 
pseudoscalar mass ($m_A$), 
the ratio of Higgs vacuum expectation values ($\tan\beta$), the Higgsino ($\mu$) and 
gaugino ($M_1$, $M_2$, $M_3$) mass parameters, trilinear couplings ($A_{Eaa}$,
$A_{Uaa}$, $A_{Daa}$, with $a=1,2,3$) as well as  soft sfermion masses ($M^2_{Qaa}$, 
$M^2_{Laa}$, $M^2_{Uaa}$, $M^2_{Daa}$, $M^2_{Eaa}$, with $a=1,2,3$).\footnote{
Note that currently only diagonal matrices are allowed.  While not being the most general
ansatz possible, this implies the absence of flavour changing neutral currents at tree-level  
in all sectors of the model.
}
Internally, those values are stored in \code{mssm} common blocks. The user may either 
provide them directly or by setting up pre-defined phenomenological MSSM models
with a reduced number of parameters
through a call to a routine like \code{dsgive\_model} or  \code{dsgive\_model25}
(followed by a call to \code{dsmodelsetup}). The 
former sets up the simplest of those models, defined by the input parameters  
$\mu$, $M_2$, $m_A$, $\tan\beta$, a common scalar mass $m_0$, and trilinear 
parameters $A_t$ and $A_b$; $M_1$ and $M_3$ are then calculated by assuming the 
GUT condition, and the remaining MSSM parameters are 
 given by ${\bf M}_Q = {\bf M}_U = {\bf M}_D = {\bf M}_E = {\bf M}_L = m_0{\bf 1}$, 
 ${\bf A}_U = {\rm diag}(0,0,A_t)$, ${\bf A}_D = {\rm diag}(0,0,A_b)$, ${\bf A}_E = {\bf 0}$. 
 Similarly, \code{dsgive\_model25} sets up a pMSSM model with 25 free parameters
 (see the header of that file for details). Alternatively, all those values can be set by 
 reading in an SLHA file, or providing GUT scale parameters in the case of cMSSM
 models (via an interface to the \code{ISASUGRA} code, as included in ISAJET \cite{Paige:2003mg,isajet_www}).
 
Compared to previous versions of the code, \ds\ 6 has a new interface to read and write SUSY 
Les Houches Accord (SLHA) files \cite{Skands:2003cj,Allanach:2008qq}. 
\comment{would be good to give some details here, also a warning that not everything is perfect
yet!}
 

 All particle and sparticle masses are stored in a common block array \code{mass()}. For 
 neutralino masses, we include the leading loop corrections \cite{Drees:1996pk,
 Pierce:1993gj,Lahanas:1993ib}
 but neglect the relatively small corrections for charginos \cite{Drees:1996pk} 
 (in both cases, masses cannot be negative in our convention).\footnote{
 Unless, of course, those values are provided by an SLHA file. This comment also applies to the following 
 simplifications concerning both sparticle masses and widths.
 }
 Likewise, all mixing matrices and decay widths are available as common block arrays. 
 The latter are currently only computed for the Higgs particles (via an interface to the
 \code{FeynHiggs} \cite{Heinemeyer:1998yj,feynhiggs_www,Heinemeyer:1998jw,
 Heinemeyer:1998np,Heinemeyer:1999be} package),
 while the other sparticles have fictitious widths of 0.5\% of the sparticle mass (for the 
 sole purpose of regularizing annihilation amplitudes close to poles). Again, the 
 conventions for masses and mixings follow exactly those of Ref.~\cite{ds4}, to which we 
 refer for further details.
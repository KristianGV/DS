%%%%%%%%%%%%%%%%%%%%%%%%%%%%%%%%%%%%%%%%%%%%%%%%%%%%%%%%%%%%%%%%%%%%
\section{Relic density -- more details on routines}

%%%%%%%%%%%%%%%%%%%%%%%%%%%%%%%%%%%%%%%%%%%%%%%%%%
\subsection{Global parameters}

When the relic density has been calculated, the integer variable \code{copart} in \code{dsandwcom.h} is set to indicate which coannihilating particles that have been included in the calculation. In Table~\ref{tab:copart}, the meaning of this variable is shown for the case of supersymmetric particles.

\begin{table}[!h]
\centering
\begin{tabular}{rrrcrrl} \hline
\multicolumn{3}{c}{\code{copart}} & & \multicolumn{2}{c}{PAW variables} \\ \cline{1-3} \cline{5-6}
Bit set & Octal value & Decimal value && \code{cop1} bit & \code{cop2} bit & Particle \\ \hline
 0 &             1 &            1 &&   0 &  -- & $\tilde{\chi}_1^0$ \\
 1 &             2 &            2 &&   1 &  -- & $\tilde{\chi}_2^0$ \\
 2 &             4 &            4 &&   2 &  -- & $\tilde{\chi}_3^0$ \\
 3 &            10 &            8 &&   3 &  -- & $\tilde{\chi}_4^0$ \\ \hline
 4 &            20 &           16 &&   4 &  -- & $\tilde{\chi}_1^\pm$ \\
 5 &            40 &           32 &&   5 &  -- & $\tilde{\chi}_2^\pm$ \\ \hline
 6 &           100 &           64 &&   6 &  -- & $\tilde{e}_1$ \\
 7 &           200 &          128 &&   7 &  -- & $\tilde{\mu}_1$ \\
 8 &           400 &          256 &&   8 &  -- & $\tilde{\tau}_1$ \\
 9 &        1\,000 &          512 &&   9 &  -- & $\tilde{e}_2$ \\
10 &        2\,000 &       1\,024 &&  10 &  -- & $\tilde{\mu}_2$ \\
11 &        4\,000 &       2\,048 &&  11 &  -- & $\tilde{\tau}_2$ \\ \hline
12 &       10\,000 &       4\,096 &&  12 &  -- & $\tilde{\nu}_e$ \\
13 &       20\,000 &       8\,192 &&  13 &  -- & $\tilde{\nu}_\mu$ \\
14 &       40\,000 &      16\,384 &&  14 &  -- & $\tilde{\nu}_\tau$ \\ \hline
15 &      100\,000 &      32\,768 &&  -- &   0 & $\tilde{u}_1$ \\
16 &      200\,000 &      65\,536 &&  -- &   1 & $\tilde{c}_1$ \\
17 &      400\,000 &     131\,072 &&  -- &   2 & $\tilde{t}_1$ \\
18 &   1\,000\,000 &     262\,144 &&  -- &   3 & $\tilde{u}_2$ \\
19 &   2\,000\,000 &     524\,288 &&  -- &   4 & $\tilde{c}_2$ \\
20 &   4\,000\,000 &  1\,048\,576 &&  -- &   5 & $\tilde{t}_2$ \\ \hline
21 &  10\,000\,000 &  2\,097\,152 &&  -- &   6 & $\tilde{d}_1$ \\
22 &  20\,000\,000 &  4\,197\,304 &&  -- &   7 & $\tilde{s}_1$ \\
23 &  40\,000\,000 &  8\,388\,608 &&  -- &   8 & $\tilde{b}_1$ \\
24 & 100\,000\,000 & 16\,777\,216 &&  -- &   9 & $\tilde{d}_2$ \\
25 & 200\,000\,000 & 33\,554\,432 &&  -- &  10 & $\tilde{s}_2$ \\
26 & 400\,000\,000 & 67\,108\,864 &&  -- &  11 & $\tilde{b}_2$ \\ \hline
\end{tabular}
\caption{The bits of \code{copart} are set to indicate which initial states that
are included in the coannihilation calculation. In the output file \code{*.omegaco}, the value of \code{copart} is written in octal format. In PAW \code{cop1} and \code{cop2} are available. Check if a bit is set with \code{btest(cop1,bit)}.}
\label{tab:copart}
\end{table}

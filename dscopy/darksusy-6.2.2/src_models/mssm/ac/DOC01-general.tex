%%%%%%%%%%%%%%%%%%%%%%%%%%%%%%%%%%%%%%%%%%%%%%%%%%%%%%%%%%%%%%%%%%%%

\subsection{Accelerator bounds}

\ds\ contains a set of routines for a rough check whether a given model is excluded by
accelerator constraints. These routines are called \ftb{dsacbnd[number]}. The
policy is that when we update \ds\ with new accelerator constraints, we keep
the old routine, and add a new routine with the last number incremented by one.
Which routine that is called is determined by calling \ftb{dsacset} with a
tag determining which routine to call. To check the accelerator constraints, 
then call \ftb{dsacbnd} which calls the right routine for you. Upon return,
\ftb{dsacbnd} returns an exclusion flag, \code{warning}. If zero, the model
is OK, if non-zero, the model is likely excluded. The cause for the exclusion is coded
in the bits of \code{excl} according to table \ref{tab:acexcl}. 

We stress that these bounds are in most cases only {\it approximate} limits: \ds\ generally 
focusses on theoretical predictions for such observables, given a DM model realization,
rather than on the implementation of experimental likelihoods and the possibility to derive
statistically well-defined limits from those. For the latter, we instead refer to packages like 
{\sf DarkBit} \cite{Workgroup:2017lvb} (or  {\sf ColliderBit} \cite{Balazs:2017moi} for 
accelerator-based constraints)

\begin{table}[!h]
\centering
\begin{tabular}{rrrcl} \hline
\multicolumn{3}{c}{\code{excl}} && \\ \cline{1-3}
Bit set & Octal value & Decimal value && Reason for exclusion \\ \hline
 0 &             1 &            1 && Chargino mass \\
 1 &             2 &            2 && Gluino mass \\
 2 &             4 &            4 && Squark mass \\
 3 &            10 &            8 && Slepton mass \\
 4 &            20 &           16 && Invisible $Z$ width \\
 5 &            40 &           32 && Higgs mass in excluded region \\
 6 &           100 &           64 && Neutralino mass \\
 7 &           200 &          128 && $b \rightarrow s \gamma$ \\
 8 &           400 &          256 && $\rho$ parameter \\ 
 9 &		 1000 &	512 &&  $(g-2)_\mu$ \\  
10 &		 2000  &	1024 && $B_s \to \mu^+ \mu^-$ \\
11  &		4000   & 	2048 && squark-gluino\\
12  &		10000  &  4096 &&  Higgs mass does not fit observed Higgs  \\  
 \hline
\end{tabular}
\caption{The bits of \code{excl} are set to indicate by which process this
particular model is excluded. Check if a bit is set with 
\code{btest(excl,bit)}.}
\label{tab:acexcl}
\end{table}

Compared to previous \ds\ versions, we use in particular updated limits from 
{\sf HiggsBounds} \cite{Bechtle:2008jh} on the mass of the MSSM Higgs bosons, as well as
approximate bounds on squark and gluino masses from LHC 8 TeV data \cite{Aad:2014wea}.
For $b \rightarrow s \gamma$, we keep our genuine routines in \code{mssm/ac\_bsg} 
for this rare decay (see Ref.~\cite{ds4} for
a more detailed description) but now use as a default the result from {\sf SuperIso} \cite{Mahmoudi:2007vz};
we compare this to the current limit of $\mathcal{B}(B \to X_s\gamma) = (3.27 \pm 0.14) \times 10^{-4}$ 
as adopted in {\sf FlavBit} \cite{Workgroup:2017myk}, based on data from BarBar and Belle 
\cite{Lees:2012wg,Lees:2012ym,Belle:2016ufb}. {\sf SuperIso} also computes the rate for the
rare leptonic decay  $B_s^0\to\mu^+\mu^-$, which we compare to the LHCb measurement of 
$\mathcal{B}(B_s^0 \to \mu^+\mu^-) = (3.0 \pm 0.6^{+0.3}_{-0.2}) \times 10^{-9}$  \cite{Aaij:2017vad}.
Finally, $a_\mu\equiv (g-2)_\mu/2$ is calculated by \code{dsgm2muon}, based on \cite{Moroi:1995yh};
in  \code{dsacbnd}, this is compared to the observed
valued of $a_{\mu,\,{\rm obs}} = (11659208.9 \pm 6.3) \times 10^{-10}$ \cite{Bennett:2006fi} 
after subtracting the SM expectation as specified in {\sf PrecisionBit} \cite{Workgroup:2017bkh}.

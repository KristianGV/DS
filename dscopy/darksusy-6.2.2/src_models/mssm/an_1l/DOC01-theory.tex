%%%%%%%%%%%%%%%%%%%%%%%%%%%%%%%%%%%%%%%%%%%%%%%%%%%%%%%%%%%%%%%%%%%%

\subsection{Annihilation cross sections at 1-loop -- general }

The annihilation cross sections at 1-loop that we have implemented in
\ds\ are those to $\gamma \gamma$, $Z \gamma$ and $g g$. The
derivation of these is described in the works \cite{Bergstrom:1997fh,Ullio:1997ke}.
Let us mention that these one-loop expressions formally violate unitarity for diagrams
with electroweak gauge-boson exchange because they do not take into account the 
nonperturbative effects related to multiple electroweak gauge boson exchange (`Sommerfeld
effect') as described in \cite{Hisano:2003ec,Hisano:2004ds}. In practice, one can still trust 
the result for sub-TeV neutralinos -- but should keep in mind that the implemented cross 
sections become unphysical in the limit $m_W/m_\chi \to 0$.

To see how these routines are called, see the file
\codeb{src\_models/mssm/an/dsandwdcosnn.f} where the $\gamma \gamma$, $Z \gamma$
and $g g$ contributions are added to the annihilation cross section at
the end.

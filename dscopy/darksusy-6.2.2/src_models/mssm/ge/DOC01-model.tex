%%%%%%%%%%%%%%%%%%%%%%%%%%%%%%%%%%%%%%%%%%%%%%%%%%%%%%%%%%%%%%%%%%%%
\label{sec:ge}

\subsection{Supersymmetric model}

We will here review the definition of the MSSM as given in \cite{ds4}.

\subsubsection{Parameters}

In our notation, the superpotential and the soft supersymmetry-breaking scalar
potential minimal supersymmetric standard model (MSSM) with R-parity
conservation \cite{Haber:1984rc,Gunion:1984yn,Haber:1988px} read respectively
\begin{eqnarray}
  W &=& \epsilon_{ij} \left(
  - {\bf \hat{e}}_{R}^{*} {\bf h}_E {\bf \hat{l}}^i_{L} {\hat H}^j_1 
  - {\bf \hat{d}}_{R}^{*} {\bf h}_D {\bf \hat{q}}^i_{L} {\hat H}^j_1 
  + {\bf \hat{u}}_{R}^{*} {\bf h}_U {\bf \hat{q}}^i_{L} {\hat H}^j_2 
  - \mu {\hat H}^i_1 {\hat H}^j_2 
  \right),
\\
  V_{{\rm soft}} & = & 
  \epsilon_{ij} \left(
    -{\bf \tilde{e}}_{R}^{*} {\bf A}_E {\bf h}_E {\bf \tilde{l}}^i_{L} H^j_1 
  - {\bf \tilde{d}}_{R}^{*} {\bf A}_D {\bf h}_D {\bf \tilde{q}}^i_{L} H^j_1 
  + {\bf \tilde{u}}_{R}^{*} {\bf A}_U {\bf h}_U {\bf \tilde{q}}^i_{L} H^j_2
  - B \mu H^i_1 H^j_2 
  \right. \nonumber \\ && \phantom{\epsilon_{ij} \Bigl(} \left.
  + {\rm h.c.} 
  \right) \nonumber \\ &&
  + H^{i*}_1 m_1^2 H^i_1 + H^{i*}_2 m_2^2 H^i_2
  \nonumber \\ && +
  {\bf \tilde{q}}_{L}^{i*} {\bf M}_{Q}^{2} {\bf \tilde{q}}^i_{L} + 
  {\bf \tilde{l}}_{L}^{i*} {\bf M}_{L}^{2} {\bf \tilde{l}}^i_{L} + 
  {\bf \tilde{u}}_{R}^{*} {\bf M}_{U}^{2} {\bf \tilde{u}}_{R} + 
  {\bf \tilde{d}}_{R}^{*} {\bf M}_{D}^{2} {\bf \tilde{d}}_{R} + 
  {\bf \tilde{e}}_{R}^{*} {\bf M}_{E}^{2} {\bf \tilde{e}}_{R} .
\end{eqnarray}
Here $i$ and $j$ are SU(2) indices ($\epsilon_{12} = +1$), ${\bf h}$'s, ${\bf
  A}$'s and ${\bf M}$'s are $3\times3$ matrices in generation space,
and the other boldface letters are vectors in generation space.
  
The current version of \ds\ uses only a restricted set of parameters.
Namely the number of free parameters (a grand total of 124 \cite{Dimopoulos:1995kn})
is reduced by setting the off-diagonal elements of the ${\bf A}$'s and ${\bf
  M}$'s to zero and imposing CP conservation (except in the CKM matrix). 

\subsubsection{Mass spectrum}

For easy reference, we now give the particle mass matrices, together with our
convention for the mixing matrices.

Concerning the Higgs sector, we choose as independent parameters $\tan\beta$
and the mass $m_A$ of the CP-odd Higgs boson. The code provides five options
for the calculation of the Higgs masses: 


\ft{higloop}=0:  tree level formulas;
\ft{higloop}=1: the effective potential approach in
\cite{Ellis:1990nz,Ellis:1991zd,Brignole:1991pq} (correcting the sign of $\mu$ in eq.~(4) of
\cite{Brignole:1991pq}); 
\ft{higloop}=2: the effective potential approach in \cite{Drees:1991mx}
  with addition of D-terms and correction of some signs and numerical factors;
\ft{higloop}=3: the analytical approximations to the
  RGE-improved effective potential in \cite{Carena:1995bx};
\ft{higloop}=4: the pole mass calculation in \cite{Carena:1995wu};
\ft{higloop}=5: FeynHiggs (requires FeynHiggs to be
installed, default) \cite{Hahn:2010te}

The  masses of the Higgs bosons are obtained from
\begin{eqnarray}
&&  {\cal M}^2_{H} = 
  \left( %\matrix{
      \begin{matrix} 
      {m_Z^2 \cos^2\beta + m_A^2 \sin^2\beta + \Delta_{11} } &
      {-\sin\beta\cos\beta(m_Z^2+m_A^2) + \Delta_{12} }
      \cr
      {-\sin\beta\cos\beta(m_Z^2+m_A^2) + \Delta_{21} } &
      {m_Z^2 \sin^2\beta + m_A^2 \cos^2\beta + \Delta_{22} }
      \end{matrix} %} 
      \right) 
  \\
&&  m^2_{H^{\pm}} = m_{A}^2+m_W^2 + \Delta_{\pm}.
\end{eqnarray}
The quantities $\Delta_{ij}$ and $\Delta_{\pm}$ are the one-loop
radiative corrections, calculated acording to the value of \ft{higloop} as
described above. Diagonalization of $ {\cal M}^2_{H} $ gives the two CP-even
Higgs boson masses, $ m_{H_{1,2}} $, and their mixing angle $\alpha$ ($ -\pi/2
< \alpha < 0$). For \ft{higloop}=4, the pole masses are then obtained solving
$ m^{2\rm pole}_{H_i} = m^2_{H_i} + \Pi_{ii}(m^{2\rm pole}_{H_i}) - \Pi_{ii}(0)
$, where $\Pi_{ii}(p^2)$ is $H_iH_i$ the self-energy. In this case, $m_{H_3}$
is the pole mass and $m_A$ is the running mass.

\begin{table}
\centering
\begin{tabular}{lllll} \hline
 & \multicolumn{4}{c}{Higgs boson} \\ \cline{2-5}
Channel & $H_1^0$ & $H_2^0$ & $H_3^0$ & $H^+$ \\
\code{i=} & \code{j=1} & \code{j=2} & \code{j=3} & \code{j=4} \\ \hline
1  & $c\bar{c}$          & $c\bar{c}$          & $c\bar{c}$          & $u \bar{d}$ \\
2  & $b \bar{b}$         & $b \bar{b}$         & $b \bar{b}$         & $u \bar{s}$ \\
3  & $t\bar{t}$          & $t\bar{t}$          & $t\bar{t}$          & $u \bar{b}$ \\
4  & $\tau^+ \tau^-$     & $\tau^+ \tau^-$     & $\tau^+ \tau^-$     & $c \bar{d}$ \\
5  & $W^+W^-$            & $W^+W^-$            & --                  & $c \bar{s}$ \\ \hline
6  & $Z^0 Z^0$           & $Z^0 Z^0$           & --                  & $c \bar{b}$ \\
7  & --                  & $H_1^0 H_1^0$       & --                  & $t \bar{d}$ \\
8  & $H_2^0 H_2^0$       & --                  & --                  & $t \bar{s}$ \\
9  & $H_3^0 H_3^0$       & $H_3^0 H_3^0$       & --                  & $t \bar{b}$ \\
10 & $H^+ H^-$           & $H^+ H^-$           & --                  & $\nu_e e^+$ \\ \hline
11 & --                  & --                  & $Z H_1^0$           & $\nu_\mu \mu^+$ \\       
12 & --                  & --                  & $Z H_2^0$           & $\nu_\tau\tau^+$\\
13 & $Z H_3^0$           & $Z H_3^0$           & --                  & $W^+ H_1^0$ \\
14 & $W^+ H^- / W^- H^+$ & $W^+ H^- / W^- H^+$ & $W^+ H^- / W^- H^+$ & $W^+ H_2^0$ \\
15 & $\mu^+ \mu^-$       & $\mu^+ \mu^-$       & $\mu^+ \mu^-$       & $W^+ H_3^0$ \\ \hline
16 & $s\bar{s}$          & $s\bar{s}$          & $s\bar{s}$          & --\\
17 & $gg$                & $gg$                & $gg$                & --\\
18 & $\gamma \gamma$     & $\gamma \gamma$     & $\gamma \gamma$     & --\\
19 & $Z^0 \gamma$        & $Z^0 \gamma$        & $Z^0 \gamma$        & --\\
20 & $\tilde{f} \tilde{f}'$ & $\tilde{f} \tilde{f}'$ & $\tilde{f} \tilde{f}'$ & $\tilde{f} \tilde{f}'$\\ \hline
\end{tabular}
\caption{Higgs partial widths \code{hdwidth(i,j)}. Index \code{i} refers to the decay channel and index \code{j} to the Higgs boson. All widths are given in GeV\@. Note that typically we have that $m_{H_2}< m_{H_3}<m_{H^+}<m_{H_1}$ so many of these decay channels are not kinematically allowed, but included for completeness. If the \code{HDECAY} interface is used, the channels where $m_{H_2}< m_{H_3}<m_{H^+}<m_{H_1}$ is not satisfied are not included. Channels 16--19 are only included if HDECAY is used.}
\label{tab:hwidth}
\end{table}

The Higgs widths are calculated at tree level, but with QCD corrections. The decays to supersymmetric particles are also included in the total width, so the sum of the partial widths in Table \ref{tab:hwidth} does not necessarily sum up to the total width given in \code{width(k)}. The loop corrections are also available via an interface to \code{HDECAY}.

  
The neutralinos $ \tilde{\chi}^0_i$ are linear combinations of the neutral
gauginos ${\tilde B}$, ${\tilde W_3}$ and of the neutral higgsinos ${\tilde
  H_1^0}$, ${\tilde H_2^0}$.  In this basis, we write their mass matrix as
\begin{eqnarray}
  {{\cal M}}_{\tilde \chi^0_{1,2,3,4}} = 
  \left( \begin{matrix} %\matrix{
  {M_1} & 0 & -m_Z s_W c_\beta & +m_Z s_W s_\beta \cr
  0 & {M_2} & +m_Z c_W c_\beta & -m_Z c_W s_\beta \cr
  -m_Z s_W c_\beta & +m_Z c_W c_\beta & \delta_{33} & -\mu \cr
  +m_Z s_W s_\beta & -m_Z c_W s_\beta & -\mu & \delta_{44} \cr
  \end{matrix} %} 
  \right) ,
\end{eqnarray}
with $c_W=\cos\theta_W$, $s_W=\sin\theta_W$, $c_\beta=\cos\beta$, and
$s_\beta=\sin\beta$.  Here $\delta_{33}$ and $\delta_{44}$ are radiative
corrections important when two higgsinos are close in mass. Their explicit
expressions are from ref.~\cite{Drees:1996pk}. To neglect these radiative
corrections set \ft{neuloop}=0 instead of \ft{neuloop}=1 (default). The
neutralino mass eigenstates are written as
\begin{equation}
  \tilde{\chi}^0_i = 
  N_{i1} \tilde{B} + N_{i2} \tilde{W}^3 + 
  N_{i3} \tilde{H}^0_1 + N_{i4} \tilde{H}^0_2 .
\end{equation}
The phases of $N_{ij}$ are chosen so that the neutralino masses
$m_{\tilde{\chi}^0_i} \ge 0$.


The charginos are linear combinations of the charged gauge bosons ${\tilde
  W^\pm}$ and of the charged higgsinos ${\tilde H_1^-}$, ${\tilde H_2^+}$.
Their mass matrix,
\begin{eqnarray}
  {\cal M}_{\tilde{\chi}^\pm} = 
  \left( \begin{matrix} %\matrix{
  {M_2} & \sqrt{2} m_W \sin\beta \cr 
  \sqrt{2} m_W \cos\beta & \mu 
  \end{matrix} %} 
  \right) ,
\end{eqnarray}
is diagonalized by the following linear combinations
\begin{eqnarray}
  \tilde{\chi}^-_i & = & U_{i1} \tilde{W}^- + U_{i2} \tilde{H}_1^- , \\
  \tilde{\chi}^+_i & = & V_{i1} \tilde{W}^+ + V_{i2} \tilde{H}_1^+ .
\end{eqnarray}
We choose ${\rm det}(U)=1$ and $U^* {\cal M}_{\tilde{\chi}^\pm}
V^\dagger = {\rm diag} ( m_{\tilde{\chi}^\pm_1},
m_{\tilde{\chi}^\pm_2} )$ with non-negative chargino masses $
m_{\tilde{\chi}^\pm_i} \ge 0$.

When discussing the squark mass matrix including mixing, it is
convenient to choose a basis where the squarks are rotated in the same
way as the corresponding quarks in the standard model.  We follow the
conventions of the particle data group \cite{Olive:2016xmw} and put the mixing
in the left-handed $d$-quark fields, so that the definition of the
Cabibbo-Kobayashi-Maskawa matrix is $\mbox{\bf K}= \mbox{\bf V}_1
\mbox{\bf V}_2^\dagger$, where $\mbox{\bf V}_1$ ($\mbox{\bf V}_2$)
rotates the interaction left-handed $u$-quark ($d$-quark) fields to
mass eigenstates.  For sleptons we choose an analogous basis, but due
to the masslessness of neutrinos no analog of the CKM matrix appears.
 
We then obtain the general $6\times6$ $\tilde{u}$- and
$\tilde{d}$-squark mass matrices:
\begin{equation}
  {\cal M}_{\tilde u}^2 = \left( 
     \begin{matrix} %\matrix{
      \mbox{\bf M}_Q^2 + \mbox{\bf m}_u^\dagger \mbox{\bf m}_u +
      D_{LL}^{u} \mbox{\bf 1} &
      \mbox{\bf m}_u^\dagger 
      ( {\bf A}_U^\dagger - \mu^* \cot\beta ) \cr
      ( {\bf A}_U - \mu \cot\beta ) \mbox{\bf m}_u &
      \mbox{\bf M}_U^2 + \mbox{\bf m}_u \mbox{\bf m}_u^\dagger +
      D_{RR}^{u} \mbox{\bf 1} \cr
      \end{matrix} %} 
      \right),
  \label{mutilde}
\end{equation}
\begin{equation}
  {\cal M}_{\tilde d}^2=\left( {
        \begin{matrix} %\matrix{
        {\mbox{\bf K}^\dagger \mbox{\bf M}_Q^2 \mbox{\bf K}+
          \mbox{\bf m}_d\mbox{\bf m}_d^\dagger+D_{LL}^{d}\mbox{\bf 1}}&
        {\mbox{\bf m}_d^\dagger ( {\bf A}_D^\dagger-\mu^*\tan\beta )}\cr
        {( {\bf A}_D-\mu\tan\beta ) \mbox{\bf m}_d}&
        {\mbox{\bf M}_D^2+\mbox{\bf m}_d^\dagger\mbox{\bf m}_d+
          D_{RR}^{d}\mbox{\bf 1}}\cr
        \end{matrix} %}
        } \right),
  \label{mdtilde}
\end{equation}
and the general sneutrino and charged slepton mass matrices
\begin{equation}
  {\cal M}^2_{\tilde\nu} = \mbox{\bf M}_L^2 + D^\nu_{LL} \mbox{\bf 1}
\end{equation}
\begin{equation}
  {\cal M}^2_{\tilde e} =\left( {
       \begin{matrix} %\matrix{
        {\mbox{\bf M}_L^2+\mbox{\bf m}_e\mbox{\bf m}_e^\dagger+
          D_{LL}^{e}\mbox{\bf 1}}&
        {\mbox{\bf m}_e^\dagger ( {\bf A}_E^\dagger-\mu^*\tan\beta )}\cr
        {( {\bf A}_E-\mu\tan\beta ) \mbox{\bf m}_e}&
        {\mbox{\bf M}_E^2+\mbox{\bf m}_e^\dagger\mbox{\bf m}_e+
          D_{RR}^{e}\mbox{\bf 1}}\cr
        \end{matrix} %}
        } \right).
  \label{metilde}
\end{equation}
Here
\begin{equation}
  D^f_{LL}=m_Z^2\cos 2\beta(T_{3f}-e_f\sin^2\theta_w),
\end{equation}
\begin{equation}
  D^f_{RR}=m_Z^2\cos 2\beta e_f\sin^2\theta_w.
\end{equation}
In the chosen basis, $\mbox{\bf m}_u$ = $\mbox{\rm diag} ( m_{\rm
u}, m_{\rm c}, m_{\rm t} )$, $\mbox{\bf m}_d $ = $\mbox{\rm diag}
(m_{\rm d}, m_{\rm s}, m_{\rm b} )$ and $\mbox{\bf m}_e $ = $
\mbox{\rm diag} (m_e,$ $ m_\mu, m_\tau )$.

The slepton and squark mass eigenstates $\tilde{f}_k$ ($\tilde{\nu}_k$
with $k=1,2,3$ and $\tilde{e}_k$, $\tilde{u}_k$ and $\tilde{d}_k$ with
$k=1,\dots,6$) diagonalize the previous mass matrices and are related
to the current sfermion eigenstates ${\bf \tilde{f}}_{L}$ and ${\bf
\tilde{f}}_{R}$ via ($a=1,2,3$)
\begin{eqnarray}
  \tilde{f}_{La} & = & \sum_{k=1}^6 \tilde{f}_k {\bf \Gamma}_{FL}^{*ka} , \\
  \tilde{f}_{Ra} & = & \sum_{k=1}^6 \tilde{f}_k {\bf \Gamma}_{FR}^{*ka} .
\end{eqnarray} 
The squark and charged slepton mixing matrices ${\bf \Gamma}_{UL,R}$,
${\bf \Gamma}_{DL,R}$ and ${\bf \Gamma}_{EL,R}$ have dimension
$6\times 3$, while the sneutrino mixing matrix ${\bf \Gamma}_{\nu L}$
has dimension $3\times3$.

This version of \ds\ allows only for diagonal matrices ${\bf A}_U$,
${\bf A}_D$, ${\bf A}_E$, ${\bf M}_Q$, ${\bf M}_U$, ${\bf M}_D$, ${\bf M}_E$,
and ${\bf M}_L$. This ansatz, while not being the most general one, implies the
absence of tree-level flavor changing neutral currents in all sectors of the
model. In this case, the squark mass matrices can be diagonalized analytically.
For example, for the top squark one has, in terms of the top squark mixing
angle $\theta_{\tilde{t}}$,
\begin{equation}
  \Gamma_{UL}^{\tilde{t}_1\tilde{t}} =
  \Gamma_{UR}^{\tilde{t}_2\tilde{t}} = \cos \theta_{\tilde{t}} ,
  \qquad
  \Gamma_{UL}^{\tilde{t}_2\tilde{t}} =
  - \Gamma_{UR}^{\tilde{t}_1\tilde{t}} = \sin \theta_{\tilde{t}} .
\end{equation}

Special values of the sfermion masses can be set with the parameters {\tt
  msquarks}, and \ft{msleptons}.  If \ft{msquarks}=\ft{msleptons}=0, the
sfermion masses are obtained with the diagonalization described above. If {\tt
  msquarks}$>$0 (or \ft{msleptons}$>$0), all squark masses are set to {\tt
  msquarks} (or all slepton masses to \ft{msleptons}).  Finally, if {\tt
  msquarks}$<$0 (or \ft{msleptons}$<$0), the squark (or slepton) masses are
set equal to the neutralino mass but never less than $|\ft{msquarks}|$ (or
$|\ft{msleptons}|$).  This is to provide the lightest possible sfermions
compatible with a neutralino LSP. In all of these cases, there is no mixing
  between sfermions.


The particle masses are available in an array \ft{mass($p$)}, where $p$ is the
particle code from table \ref{tab:2}. Similarly, particle decay width are
available as \ft{width($p$)}, but currently only the width of the Higgs bosons
are calculated, the other particles having fictitious widths of 1 or 5 GeV (for
the sole purpose of regularizing annihilation amplitudes close to poles).

\begin{table*}
\label{tab:2}
\caption{Particle codes (synonyms are separated by commas). }
\begin{center}
\begin{tabular}{| l l | l l | l l | l l |}
\hline
$\nu_e$ & \ft{knue,knu(1)} &
$\gamma$ & \ft{kgamma} &
$\tilde{\chi}^0_i$ & \ft{kn($i$)~$i=1\ldots4$} &
$\tilde{u}_1$ & \ft{ksu(1),ksqu(1)}
\\
$e$ & \ft{ke,kl(1)} &
$W^\pm$ & \ft{kw} &
$\tilde{\chi}^\pm_k$ & \ft{kcha($k$)~$k=1,2$} &
$\tilde{u}_2$ & \ft{ksu(2),ksqu(4)}
\\
$\nu_\mu$ & \ft{knumu,knu(2)} &
$Z^0$ & \ft{kz} &
$\tilde{g}$ & \ft{kgluin} &
$\tilde{d}_1$ & \ft{ksd(1),ksqd(1)}
\\
$\mu$ & \ft{kmu,kl(2)} &
$g$ & \ft{kgluon} &
$\tilde{\nu}_e$ & \ft{ksnue,ksnu(1)} &
$\tilde{d}_2$ & \ft{ksd(2),ksqd(4)}
\\
$\nu_\tau$ & \ft{knutau,knu(3)} &
&&
$\tilde{e}_1$ & \ft{kse(1),ksl(1)} & 
$\tilde{c}_1$ &\ft{ksc(1),ksqu(2)}
\\
$\tau$ & \ft{ktau,kl(3)} &
&&
$\tilde{e}_2$ & \ft{kse(2),ksl(4)} & 
$\tilde{c}_2$ &\ft{ksc(2),ksqu(5)}
\\
$u$ & \ft{ku,kqu(1)} &
$H^0$ & \ft{kh1} &
$\tilde{\nu}_\mu$ &\ft{ksnumu,ksnu(2)} &
$\tilde{s}_1$ &\ft{kss(1),ksqd(2)}
\\
$d$ & \ft{kd,kqd(1)} & 
$h^0$ & \ft{kh2} & 
$\tilde{\mu}_1$ &\ft{ksmu(1),ksl(2)} &
$\tilde{s}_2$ &\ft{kss(2),ksqd(5)}
\\
$c$ &\ft{kc,kqu(2)} &
$A^0$ & \ft{kh3} &
$\tilde{\mu}_2$ &\ft{ksmu(2),ksl(5)} &
$\tilde{b}_1$ &\ft{ksb(1),ksqd(3)}
\\
$s$ &\ft{ks,kqd(2)} &
$H^\pm$ & \ft{khc} &
$\tilde{\nu}_\tau$ &\ft{ksnuta,ksnu(3)} &
$\tilde{b}_2$ &\ft{ksb(2),ksqd(6)}
\\
$b$ &\ft{kb,kqd(3)} &
$G^0$ & \ft{kgold0} &
$\tilde{\tau}_1$ & \ft{kstau(1),ksl(3)} &
$\tilde{t}_1$ & \ft{kst(1),ksqu(3)}
\\
$t$ &\ft{kt,kqu(3)} &
$G^\pm$ & \ft{kgoldc} &
$\tilde{\tau}_2$ & \ft{kstau(2),ksl(6)} &
$\tilde{t}_2$ & \ft{kst(2),ksqu(6)}
\\
\hline
\end{tabular}
\end{center}
\end{table*}

\subsubsection{Three-particle vertices}

We define three-particle vertices \ft{gl($i$,$j$,$k$)}$=g^L_{ijk}$ and
\ft{gr($i$,$j$,$k$)}$=g^R_{ijk}$ as follows. We adopt the convention
that the order of the particles in the indices is the order in which
they appear in the corresponding lagrangian term, so the last particle
is always entering. If there are charged particles in the vertex, they
are both assumed positively charged, and the particle that exits the
vertex is indexed before the particle that enters.
\begin{itemize}
\item Three scalar bosons:
\begin{equation}
{\cal L}_{\rm int} = g_{\phi_i\phi_j\phi_k} m_W \phi_i \phi_j \phi_k
\end{equation}
where $\phi_i$ is a Higgs or a Goldstone boson. In this case, \ft{gl}={\tt
  gr}=$g$. Available vertices are $\phi_i\phi_j\phi_k$ = $H^0_iH^0_jH^0_k$,
$H^0_iH^-H^+$, $H^0_iA^0A^0$, $H^0_iG^0G^0$, $H^0_iG^-G^+$, $H^0_iG^-H^+$,
$H^0_iG^-G^+$, $A^0G^-H^+$, $A^0G^0H^0_i$, and permutations.

\item Two scalar and one vector bosons:
\begin{equation}
{\cal L}_{\rm int} = g_{V\phi_1\phi_2} V^\mu \phi_1 i \lrpartial_\mu \phi_2 .
\end{equation}
Available vertices are $V\phi_1\phi_2$ = $Z^0H^0_iA^0$, $Z^0H^-H^+$, $\gamma
H^-H^+$, $W^-H^+A^0$, $W^-H^+H^0_i$, and permutations.

\item One scalar and two vector bosons:
\begin{equation}
{\cal L}_{\rm int} = g_{\phi V_1 V_2} m_W g_{\mu\nu} \phi V^\mu_1 V^\nu_2
\end{equation}
Available vertices are $\phi V_1V_2$ = $H^0_iW^-W^+$, $H^0_iZ^0Z^0$.

\item Three vector bosons:
%\begin{equation}
%{\cal L}_{\rm int} = - g_{V_1V_2V_3} \epsilon_{abc} V_1^{\mu a} V_2^{\nu b} 
%\partial_\mu V_{3\nu}^c \hbox{\bf to be checked!}
%\end{equation}
%which corresponds to the vertex
\begin{equation}
i g_{V_1V_2V_3} \left[ (k_1-k_3)_\nu g_{\mu\lambda} + (k_3-k_2)_\mu
  g_{\lambda\nu} + (k_2-k_1) g_{\mu\nu} \right]
\end{equation}
with all momenta incoming and assigned as $V_1^\mu(k_1)$, $V_2^\nu(k_2)$ and
$V_3^\lambda(k_3)$. 
Available vertices are $Z^0W^-W^+$ and $\gamma W^-W^+$.

\item One scalar boson and two Dirac fermions:
\begin{equation}
{\cal L}_{\rm int} = \phi \overline{\psi}_1 ( g^L_{\phi\psi_1\psi_2} P_L +
g^R_{\phi\psi_1\psi_2} P_R ) \psi_2
\end{equation}
Available vertices are $\phi\psi_1\psi_2$ = 

\item One vector boson and two Dirac fermions:
\begin{equation}
{\cal L}_{\rm int} = V_\mu \overline{\psi}_1 \gamma^\mu
( g^L_{V\psi_1\psi_2} P_L + g^R_{V\psi_1\psi_2} P_R ) \psi_2
\end{equation}
Available vertices are $V\psi_1\psi_2$ = 

\item One scalar boson, one Dirac and one Majorana fermion:
\begin{equation}
  {\cal L}_{\rm int} = \phi \overline{\psi} 
  ( g^L_{\phi\psi\chi} P_L + g^R_{\phi\psi\chi} P_R ) \chi
\end{equation}
Available vertices are $ \phi\psi\chi$ = 

\item One vector boson, one Dirac and one Majorana fermion:
\begin{equation}
  {\cal L}_{\rm int} = V_\mu \overline{\psi} \gamma^\mu 
  ( g^L_{V\psi\chi} P_L + g^R_{V\psi\chi} P_R ) \chi
\end{equation}
Available vertices are $V\psi\chi$ = 

\item One scalar boson and two Majorana fermions:
\begin{equation}
{\cal L}_{\rm int} = 
\end{equation}
Available vertices are\ldots

\item One vector boson and two Majorana fermions:
\begin{equation}
{\cal L}_{\rm int} = 
\end{equation}
\end{itemize}

%\begin{eqnarray}
%{\cal L}_{\rm int} & = & 
%V_\mu \bar{\chi} \gamma^\mu (g^L_{V\chi\psi} P_L + g^R_{V\chi\psi} P_R) \psi +
%\phi^* \bar{\chi} (g^L_{\phi\chi\psi} P_L + g^R_{\phi\chi\psi} P_R) \psi 
%\nonumber \\
%&& \, + g^A_{Z\psi\psi'} Z_\mu \bar{\psi} \gamma^\mu \gamma_5 \psi'
%+ g^P_{\phi\psi\psi'} \phi^* \bar{\psi} \gamma_5 \psi' 
%+ {\rm h.c.} 
%\nonumber\\
%&&\, + {\textstyle{1\over2}}
% g^A_{Z\chi\chi'} Z_\mu \bar{\chi}  \gamma^\mu \gamma_5 \chi' +
%{\textstyle{1\over2}} 
%g^P_{\phi\chi\chi'} \phi^* \bar{\chi} \gamma^\mu \gamma_5 \chi 
%+ {\rm h.c.} 
%\end{eqnarray}
%Here $V_\mu$ is a vector boson, $\chi,\chi'$ are Majorana fermions,
%$\psi,\psi'$ are Dirac fermions, and $\phi$ can be a neutral ($\phi^*=\phi$) or
%a charged scalar.

Explicit expressions for the coupling constants $g_{ijk}$ can be obtained
in~\cite{Haber:1984rc,Gunion:1984yn,Haber:1988px}, with radiative corrections to trilinear scalar couplings in
\cite{haber97}. We have rederived from the superpotential all vertices we have
implemented.

Implemented vertices:  those listed above plus 
$Z^0W^\pm W^\mp $, $Z^0H^0_iH^0_i$, $W^\pm H^\mp A^0$,
$W^\pm H^\mp H^0_i$, $H^0_iW^\pm W^\mp $, $H^0_iZ^0Z^0$, $Z^0A^0H$,
$H^0_iA^0A^0$, $A^0ff$, $H^0_iff$, $Z^0ff$, $Z^0\tilde{\chi}^0\tilde{\chi}^0$,
$H^0_i\tilde{\chi}^0\tilde{\chi}^0$, $Z^0\tilde{\chi}^0\tilde{\chi}^0$,
$W^\mp\tilde{\chi}^0\tilde{\chi}^\pm$, $H^\mp\tilde{\chi}^0\tilde{\chi}^\pm$, $
\tilde{q}\tilde{g}q$, $\tilde{f}\tilde{\chi}^0 f$, $H^0_i\tilde{\chi}^\pm
\tilde{\chi}^\mp$, $A^0\tilde{\chi}^\pm \tilde{\chi}^\mp$ , $W^\pm f f'$,
$H^\pm f f'$, $\gamma W^\pm W^\mp$, $ \gamma H^\pm H^\mp$, $Z^0
\tilde{\chi}^\pm \tilde{\chi}^\mp$, $\gamma\tilde{\chi}^\pm \tilde{\chi}^\mp$,
$\gamma f f$, $GHH$, $GGH$, $G^\mp\tilde{\chi^0}\tilde{\chi}^\pm$.

%In appendix \ref{app:feyn}, most of the Feynman rules and the explicit
%expressions for the $g$'s are found. \joakim{Should we have Feynman rules?}


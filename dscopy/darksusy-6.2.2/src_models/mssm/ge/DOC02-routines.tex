%%%%%%%%%%%%%%%%%%%%%%%%%%%%%%%%%%%%%%%%%%%%%%%%%%%%%%%%%%%%
\subsection{General supersymmetry -- routines}

Input parameters, options, results, etc.\ are contained in common blocks in the
file \ft{dsmssm.h}, which the user has to include. The input parameters
are ($a=1,2,3$) \newline\begin{tabular}{llll} \ft{ma}$=m_A$, & \ft{tanbe}$=\tan\beta$, & \ft{mu}$=\mu$, &
  \ft{m1}$=M_1$, \\
  \ft{m2}$=M_2$, & \ft{m3}$=M_3$, & \ft{asofte($a$)}$=A_{Eaa}$, &
  {asoftu($a$)}$=A_{Uaa}$, \\
  \ft{asoftd($a$)}$=A_{Daa}$, & \ft{mass2q($a$)}$=M^2_{Qaa}$, & {\tt
    mass2l($a$)}$=M^2_{Laa}$, &
  \ft{mass2u($a$)}$=M^2_{Uaa}$, \\
  \ft{mass2d($a$)}$=M^2_{Daa}$, &
  \ft{mass2e($a$)}$=M^2_{Eaa}$. & & \\
\end{tabular}\newline
The options are (see previous subsections for a description)
\newline
\ft{higloop} choice of tree-level or radiatively corrected Higgs boson
  masses; 
\newline 
\ft{neuloop} choice of tree-level or radiatively corrected neutralino
  masses; 
\newline
\ft{msquarks,msleptons} choice of squark and slepton masses.

\bigskip

To initialize \ds\ for a new model, you should call
\begin{sub}{subroutine \textbf{dsmodelsetup}(unphys,warning)}
  \itit{Purpose:} To calculate the particle spectrum, widths and
    couplings.
  \itit{Output:} 
  \itv{unphys}{i} non-zero if the model is unphysical
  \itv{warning}{i} non-zero if (typically the Higgs) code has issued a warning.
\end{sub}
which calculates couplings, masses and some basic cross sections.

The following subroutines specify the values of the model parameters,
and read/write them to a file. The user should create his own versions
by editing a copy of them. Please call them with a different name.

\begin{sub}{subroutine
  \textbf{dsgive\_model}(mu,m2,ma,tanbe,msq,atm,abm)}
  \itit{Purpose:} Set the MSSM parameters as specified by the
  arguments.
  \itit{Inputs:}
  \itv{mu}{r8} The $\mu$ parameter in GeV.
  \itv{m2}{r8} The $M_2$ parameter in GeV.
  \itv{ma}{r8} The mass of the CP-odd Higgs boson, $m_A$ in
  GeV.
  \itv{tanbe}{r8} $\tan \beta$.
  \itv{msq}{r8} Sets ${\bf M}^Q$, etc.\ to a common mass
  scale $m_0$ in GeV.
  \itv{atm}{r8} Sets $A_t$ in units of $m_0$ (range: -3 --- 3).
  \itv{abm}{r8} Sets $A_b$ in units of $m_0$ (range: -3 --- 3).
\end{sub}

There are also routines to setup more general MSSM models, the currently most general being \code{dsgive\_model25}.

The following subroutines are useful in the analysis.

\begin{sub}{subroutine \ftb{wspctm}(unit)}
  \itit{Purpose:} Write the particle mass spectrum and mixing matrices
  to unit \ft{unit}.
  \itit{Inputs:}
  \itv{unit}{i} Unit number to write to.
\end{sub}

\begin{sub}{subroutine \ftb{wvertx}(unit)}
  \itit{Purpose:} Write all non-vanishing three-particle vertices to
  unit \ft{unit}.
  \itit{Inputs:}
  \itv{unit}{i} Unit number to write to.
\end{sub}

\begin{sub}{subroutine \ftb{wunph}(unit)}
  \itit{Purpose:} Write the reason for which the model is not
  physically acceptable (tachyons, etc.) to unit \ft{unit}.
  \itit{Inputs:}
  \itv{unit}{i} Unit number to write to.
\end{sub}

\begin{sub}{subroutine \ftb{wexcl}(unit)}
  \itit{Purpose:} Write the reason(s) for which the model is
  experimentally excluded  to unit \t{unit}.
  \itit{Inputs:}
  \itv{unit}{i} Unit number to write to.
\end{sub}  

\begin{sub}{subroutine \ftb{dswhwar}(unit)}
  \itit{Purpose:} Write the reason(s) for which the Higgs
  calculation issued warnings to unit \ft{unit}.
  \itit{Inputs:}
  \itv{unit}{i} Unit number to write to.
\end{sub}  



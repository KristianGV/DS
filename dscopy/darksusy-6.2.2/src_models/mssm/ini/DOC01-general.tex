%%%%%%%%%%%%%%%%%%%%%%%%%%%%%%%%%%%%%%%%%%%%%%%%%%%%%%%%%%%%%%%%%%%%

\subsection{Initialization routines}

Before \ds\ is used for some calculations, it needs to be
initialized. This is done with a call to \codeb{dsinit}, 
which in turn calls the particle-specific initialization routine 
\code{dsinit\_module} that resides in each module.
This means that the call to \codeb{dsinit} should be the first call in
any program using \ds. Any calls the user makes to other routines,
either to calculte things or select a different model (e.g.\ a
different halo model) should come after the call to \codeb{dsinit}.

In the \code{mssm} module, the routine \code{dsinit\_module}  that all standard parameters 
are defined, such as standard model parameters and particle codes.
 It also sets more particle-specific defaults for \ds\ routines where this is
 available, via calls to routines such as \code{dsddset\_mssm}  
 and \code{dsanset\_mssm}.


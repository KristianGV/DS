%%%%%%%%%%%%%%%%%%%%%%%%%%%%%%%%%%%%%%%%%%%%%%%%%%%%%%%%%%%%%%%%%%%%
\chapter{Generic WIMPs}
\label{ch:genWIMP}

Similar to the case of \code{generic\_decayingDM}, the module \code{generic\_wimp} provides 
a simple phenomenological template for a large class of DM candidates. Rather than being based 
on an actual particle physics model, it mostly serves to provide an illustration of how the functionalities 
of DarkSUSY can be used in phenomenological studies of ‘vanilla’ WIMP DM, when only providing 
the absolute minimum of input parameters.

A generic WIMP model in DarkSUSY is set up by a call to \code{dsgivemodel\_generic\_wimp}, 
and hence fully defined by the input parameters of that routine: the mass $m_\chi$ of the DM particle 
and a flag stating whether the DM particle is its own anti-particle or not; a constant annihilation 
rate $\sigma v$ in the CMS frame, along with the dominant annihilation channel into SM particles
(as usual stated in terms of PDG codes); finally, the spin-independent 
scattering cross section $\sigma_{\rm SI}$ of DM with nucleons.

This simple setup allows to use essentially the full functionality of the \code{core} library,
from relic density calculation to direct detection rates and cosmic ray propagation and
indirect detection signals in general.

\newpage
\section{Changes in Recent Versions}

        This section contains a record of changes in recently released
versions of ISAJET, taken from the memoranda distributed to users.
Note that the released version numbers are not necessarily consecutive.

\subsection{Version~7.88, January 2018}

We dedicate this version to Frank Paige (who passed away on October 16, 2017). 
Frank-- along with Serban Protopopescu--
created the original Isajet code in the late 1970s.

Version 7.88 includes the {\it natural} anomaly-mediated SUSY breaking model,
using keyword input NAMSB. This model includes independent bulk soft term 
contributions to Higgs scalars $m_{H_u}(bulk)$ and $m_{H_d}(bulk)$ along with
small bulk $A_0$ terms. These added terms can render AMSB as a natural
model with $m_h\sim 125$ GeV for appropriate parameter choices.
For natural parameter choices, then the LSP is a higgsino rather than a wino.
The parameter space is given by $m_0(1,2)$, $m_0(3)$, $m_{3/2}$, $A_0$,
$\tan\beta$, $\mu$ and $m_A$ where small $\mu\sim 100-300$ GeV,
$A_0\sim m_0(3)$ and 80 TeV$<m_{3/2}<150$ TeV lead to natural models.

\subsection{Version~7.87, July 2017}

Version 7.87 is functionally identical to 7.86. The only change is that
the format has been converted from a Patchy .car file to a Unix .tar
file with C Preprocessor directives. The conversion was mostly done
automatically by changing Patchy \verb|+CDE| to \verb|#include|,
Patchy \verb|+SELF,IF=| to \verb|#ifdef|, and Patchy |+SELF| to
\verb|#endif|. To facilitate the automatic conversion, the ISAJET and
ISASUSY patches have been combined. The ISARED relic density code has
not been modified.

\subsection{Version~7.86, January 2017}

Isajet 7.86 includes several changes.

1. Dominant two loop contributions from Dedes/Slavich to EWSB have been
incorporated, which are relevant to the fine-tining calculation in the pMSSM.

2. Some corrections added to bring Higgs widths into closer accord with
latest theory.

3. Some minor bugs in SUGEFF found by Peter Ruud are fixed.

4. New generalized mirage mediation model (GMM) based on arXiv:1610.06205
can be run using inputs $\alpha$, $m_{3/2}$, $c_m$, $c_{m3}$, $a_3$, $\tan\beta$,
$sign (\mu )$, $c_{H_u}$ and $c_{H_d}$. The latter two inputs may be traded
for $\mu$ and $m_A$ if the $NUHM$ keyword is used.

\subsection{Version~7.85, November 2015}

Isajet 7.85 includes several changes.

1. Quark and gluon matrix elements for spin-independent neutralino-nucleon
scattering rates in ISARES were updated according to 
Table 1 of Hisano et al., PRD87 (2013) 035020. 

2. Light SUSY Higgs boson $h$ coupling strengths are computed for Isasugra
and are located in common block KPHGGS: $\kappa_b$, 
$\kappa_t$, $\kappa_\tau$, $\kappa_W$, $\kappa_Z$, $\kappa_g$, $\kappa_\gamma$.

3. New Isasugra model which calculates non-universal 
soft terms via $D$-term splittings is programmed as model $\#11$: NUHMDT.
The input parameters are $m_{16},$, $m_{1/2}$, $A_0$, $\tan\beta$, $\mu$, 
$m_A$ where $\tan\beta$, $\mu$ and $m_A$ are weak scale inputs
and $m_{10}$ and $sign(M_D^2)\sqrt{|M_D^2|}$ are output parameters. 
The $D$-term splitting model is defined in Eq. 11.18 of 
{\it Weak Scale Supersymmetry}.

4. A scale factor multiplier SCLFAC can be input in scattering events
which multiplies the QCD coupling scale choice by a numerical factor
which can increase or decrease the tree level QCD cross section which is output.

\subsection{Version~7.84, June 2014}

Isajet 7.84 includes several changes.

1. Minor bug fixes implemented in SUGEFF and the measure $\Delta_{EW}$.

2. The LesHouchesAccord (LHA) file is now output to LesHouchesEvents (LHE) file.

3. The sparticle decay table is now output to LHA output file.

4. The ISAHEP routine was updated to allow for proper conversion
of STDHEP labels back to ISAJET labels.

5. Isajet LHE file now prints out all subprocess reactions along with differential cross
section $d\sigma/dp_t^2$ at $y=0$. In the case of SUSY with ALL jettypes, this can add up
to over 2500 subprocesses. Earlier versions of Pythia 6.xx allowed up to 500 subprocesses but
Pythia 8 can handle arbitrarily many subprocesses. 

6. The Isasusy (pMSSM) code will now output a LHA. Also there is the option to list 
in the LHA file from Isasusy all $e^+e^-\rightarrow SUSY/Higgs$ total cross sections 
depending on collider energy $\sqrt{s}$ and beam polarization $P_L(e^-)$ and $P_L(e^+)$ if
the flag PRTEESIG is set in the Makefile.

Thanks to Azar for help with some of the above.

\subsection{Version~7.83, June 2012}

Isajet 7.83 contains four changes. 

1. The branching fraction $BF(B_u\rightarrow\tau\nu_\tau )$ calculated
at tree level and output in IsaTools.

2. Previous versions could generate negative squared masses for third 
generation sfermions. The current version stops the calculation and
outputs an error message.

3. The percent electroweak fine-tuning is now output from Isasugra.
The effective potential calculation includes some small contributions from
$W$, $Z$ and $H^\pm$. Thanks to A. Mustafayev for help.

4. The Isasusy mass spectrum output has been improved so that it is similar to
the Isasugra output.

\subsection{Version~7.82, June 2011}

Isajet 7.82 contains some file name re-labelings needed to sensibly run \texttt{RGEFLAV}.
It also contains a bug fix needed for reliable event generation in NUHM SUSY models.
The gravitino mass is now output in LHA files.

\subsection{Version~7.81, April 2011}

This version adds an optional package ISAFLAVR for ISASUGRA that
calculates the flavor structure in more detail. For detailed description, 
see Section~\ref{sec:rgeflav} above and the new Makefile.

The latest version of Isajet provides \texttt{RGEFLAV} which is interfaced to ISASUGRA.
After running ISASUGRA, the user is prompted if they wish to invoke RGEFLAV by providing a
filename \texttt{Prefix}. The RGEFLAV run can take
of order a minute to run, so should only be used in special cases.

\texttt{RGEFLAV} contains the RGEs, including one-loop threshold corrections, 
for all dimensionless and dimensionful parameters of the MSSM, 
with full matrix structure and support for complex entries. 
It therefore includes the complex KM matrix rotation in the quark sector. 
In addition, there are options in the input file, \texttt{Prefix.rgein} 
(provided by the user; a sample file \texttt{sample.rgein} is contained in the release), 
to allow the user to maintain control over sources of new flavor-violation at the high scale, 
while still providing the freedom to input general matrices if desired.

After iterating to find a convergent solution, the code writes out a full list of the 
dimensionless and dimensionful couplings at the scales $Q=m_H$ and $Q=M_{\mathrm{HIGH}}$ 
and both the up-type and down-type squark mass matrices, in a specified current basis. 
The output files are labeled as \texttt{Prefix.gtout}, \texttt{Prefix.wkout}, 
\texttt{Prefix.sqm2u}, and \texttt{Prefix.sqm2d}.
Depending on this choice of basis, one of the two squark mass matrices is diagonalised 
and the eigenvectors and eigenvalues are also written to a file.
More detailed information is available in the Isajet users manual, or in the journal article:

Threshold and Flavour Effects in the Renormalization Group Equations of the MSSM II: Dimensionful couplings.
by Andrew D. Box and Xerxes Tata, . Oct 2008. (Published Oct 2008). 96pp.
Published in Phys.Rev.D79:035004,2009.
e-Print: arXiv:0810.5765 [hep-ph] 

In addition, Isajet 7.81 contains some slight modifications of Yukawa coupling thresholds
for SUSY RGE running, to gain accord with \texttt{RGEFLAV}, and some minor fixes
to NUHM models suggested by A. Mustafayev.

\subsection{Version~7.80, October 2009}

In Isajet 7.80, we have expanded the ISALHA code to output
Les Houches Accord (LHA) files for all varieties of supersymmetric 
models (earlier, just mSUGRA was enabled). We thank C. Balazs
for help with this piece of code. 

We have also fixed several bugs in the color flow assignments 
for subprocesses entering the Isajet Les Houches Event files (LHE). 


\subsection{Version~7.79, December 2008}

We have added the hypercharged anomaly mediation model of
Dermisek, Verlinde and Wang (PRL100, 131804 (2008)) to the SUSY
model list. This model can be activated via use of the keyword HCAMSB.

In ISASUGRA, we have also adjusted the Yukawa coupling beta-function thresholds
in subroutine SURG26. Previously, all squarks decoupled at a common scale
set at $m_{{\tilde u}_L}$ and all sleptons decoupled at a common scale
$m_{{\tilde e}_L}$. Now, the 1st/2nd and separately the 
3rd generation squarks and sleptons
decouple at their appropriate soft term values.

In addition, while 2-loop RGE terms were included for MSSM running,
no 2-loop RGE terms were included for RGE running between 
$M_Z$ and $M_{SUSY}$. Now, the MSSM 2-loop terms are included for
running between $M_Z$ and $M_{SUSY}$. The current version has improved agreement with the exact RGE decoupling solution given by Box and Tata:
PRD77, 055007 (2008) and arXiv:0810.5765.

\subsection{Version~7.78, March 2008}

In Isajet 7.78, several upgraded features have been added.

First, we have added $\tilde{t}_1\rightarrow bW\tz_1$ 3-body stop decays,
which were previously missing. We have also improved the loop calculation 
$\tilde{t}_1\rightarrow c\tz_1$, which often competes with the 3-body decay
(thanks to A. Box for help on this issue).

We have also improved the calculation of $b\rightarrow s\gamma$ decay in Isatools.
For full flavor structure in the decay, 137 MSSM RGEs must be solved simultaneously. 
These RGEs have been upgraded to two-loop ones, with double precision runnning
(thanks to A. Mustafayev).
While $BF(b\rightarrow s\gamma )$ is now more accurate, the calculation is 
slightly slower. Thus, when running Isatools, the user is now prompted 
as to which calculations are needed, so the slower $b\rightarrow s\gamma$ calculation 
can be avoided if the user is not interested in the result.

We have implemented in 7.76 the Les Houches Event output capability. 
By setting  the keyword WRTLHE to TRUE in the input file, parton showering,
hadronization and underlying event are turned off, so just the production
subprocess followed by (cascade) decays are allowed, and color flow information is
kept. Events are written in standard format to an output file named \verb|isajet.lhe|,
which can be read in by programs such as Pythia and Herwig, if alternative 
showering, hadronization and underlying event algorithms are desired, which may include 
color flow information.
For the Pythia read-in code, it is necessary to set the lightest neutralino
as stable by hand. This can be done via \verb|CALL PYGIVE('MDCY(C1000022,1)=0')|. 
Alternatively, if one has access to the \verb|PYDAT3| common block, one may set 
\verb|MDCY(PYCOMP(1000022),1)=0|.

We have also fixed some minor bugs in the initialization of the neutralino relic density 
calculation by Isared, and an initialization bug that affects sparticle spectra 
coming from successive runs of Isasugra.
We fixed a bug in sparticle width assignments going from IsaReD to CalcHEP.
Many thanks to Sasha Pukhov and Sasha Belyaev for scrutinization of this code!

We have also changed the scale at which Yukawa couplings are evaluated at in the relic 
density calculation, from $Q=M_{SUSY}$ to $Q=2m_{\tz_1}$. In Isatools, we now output
the thermally averaged neutralino annihilation cross section times velocity evaluated 
as $v\rightarrow 0$, which is useful for indirect detection of dark matter calculations.

\subsection{Version~7.75, January 2007}

In Isajet 7.75, we have added the mixed modulus-anomaly mediated
SUSY breaking model which is inspired by the KKLT construct of
type IIB string models compactified with fluxes to stabilize the moduli.
This model gives rise to the phenomenon of mirage unification, wherein
scalars and gauginos unify at an intermediate scale while gauge couplings still
unify at $M_{GUT}$. The model is implemented as model $\#9$ in Isasugra,
and via the \verb|MMAMSB| keyword for event generation. The inputs consist
of modulus-AMSB mixing parameter $\alpha$, gravitino mass $m_{3/2}$,
$\tan\beta$, $sign(\mu )$, the matter and Higgs field modular weights
$n_Q$, $n_D$, $n_U$, $n_L$, $n_E$, $n_{H_d}$ and $n_{H_u}$, and the
modulus powers $\ell_1$, $\ell_2$ and $\ell_3$ that enter the 
gauge kinetic function. The modular weights can take values of
$0,\ 1,\ {1/2}$ for fields on a D7 brane, D3 brane or brane intersection
respectively. The modulus power takes values of 1 or 0 for gauge fields
on a D7 brane or D3 brane, respectively. See hep-ph/0604253 for more
detailed discussion.

We have also fixed a bug in the evaluation of radiatively corrected 
sbottom mixing angle. The self-energies used in th mixing angle calculation
used the radiatively corrected sbottom mass scale, and thus were slightly
inconsistent with the mass m(b1), which used the tree-level value of
m(b1) in the self energies. The effect was amplified in some 
regions of parameter space where large cancellations occur in the 
sbottom mixing angle calculation. The error propagated into the mA calculation 
via Yukawa couplings, giving at large tan(beta) too large a value of mA.

\subsection{Version~7.74, February 2006}

In Isajet 7.73, we have modified the ISASUGRA SUSY spectrum calculator
to extract all running parameters for mixed sparticles (the -inos, third
generation squarks and sleptons) at the common scale ${\tt
HIGFRZ}=\sqrt{m_{\tilde t_L} m_{\tilde t_R}}$, to gain consistency in
implementing the one-loop radiative corrections. Non-mixing soft terms
are still extracted at their own scale, to minimize radiative
corrections.  Squark mixing contributions to the gluino mass radiative
correction have also been added. For details, see H. Baer, J. Ferrandis,
S. Kraml and W. Porod, hep-ph/0511123.
We have also adapted the MSSM Higgs mass calculation to use the
running $b$ and $t$ quark masses from the RG code. This typically
decreases the light Higgs mass by 2--3~GeV.

We have also generalized the decay routines and radiative corrections to
gain validity for either sign of the gaugino mass $M_3$. Previously, the
decay formulae were calculated under the assumption of positive gluino
mass, while all along either sign of $M_1$ and $M_2$ were allowed.

We have added an \verb|ISALHA| subroutine which outputs a sparticle
decay table in Les Houches Accord output format (thanks to C. Balazs for
help on this).

Several minor bugs have been corrected in the relic density routine
\verb|ISARED| (thanks to A. Belyaev and A. Pukhov).

\subsection{Version~7.72, August 2005}

Isajet Version 7.72 provides \verb|IsaTools|, including
\begin{enumerate}
\item \verb|IsaRED|, subroutines to evaluate the relic density of
(stable) neutralino dark matter in the universe;
\item \verb|IsaBSG|, subroutines to evalue the branching fraction
$BF(b\to s\gamma )$;
\item \verb|IsaAMU|, subroutines to evaluate supersymmetric
contributions to
$\Delta a_\mu\equiv (g-2)_\mu/2$;
\item \verb|IsaBMM|, subroutines to evaluate $BF(B_s\rightarrow
\mu^+\mu^-)$ and $BF(B_d\rightarrow \tau^+\tau^- )$ in the MSSM;
\item \verb|IsaRES|, subroutines to evaluate the spin-independent and
spin-dependent neutralino-proton and neutralino-neutron scattering cross
sections for direct detection of dark matter.
\end{enumerate}
\verb|IsaTools| is so far interfaced only to ISASUGRA, and it it
requires \verb|isared.tar| in addition to \verb|isajet.car| and the
standard \verb|Makefile| contained therein.

This version also optionally provides from ISASUGRA both an output file
compatible with the SUSY LesHouches Accord and an output file compatible
with the ISAWIG interface for HERWIG. The ISAWIG interface assumes that
$R$-parity is conserved.

Isajet 7.72 also allows for the entry of negative squared Higgs soft
masses in SUGRA models with non-universality using the NUSUG3 keyword.
In addition, in models with non-universal Higgs masses, one now has the
option to use either GUT scale Higgs soft masses using NUSUG3, or weak
scale parameters $\mu$ and $m_A$ can be used as inputs using the NUHM
keyword.

Various small corrections have been made.

\subsection{Version~7.70, October 2004}

The solution of the renormalization group equations for SUSY models has
been converted to double precision. This gives better numerical
stability, especially in difficult regions such as the large $m_0$ or
``focus point'' region of minimal SUGRA. The two-loop corrections for
the top mass have been included.

Non-minimal AMSB models have been added. The keyword \verb|AMSB2| can be
used to set the high-scale s-fermion masses:
$$
m_{\tilde f}^2=m_{\tilde f}^2({\rm AMSB})+c_f m_0^2\,.
$$
The processes $e^+e^-\to\gamma\gamma\to f\bar{f}$ ($f$ is a SM fermion)
have also been included using Peskin's photon structure function from
brem- and beamstrahlung. These gamma-gamma induced processes are
activated by setting the keyword \verb|GAMGAM| to be \verb|.TRUE.| when
running with \verb|EEBEAM|.

\subsection{Version~7.69, August 2003}

The complete set of 1-loop radiative corrections to sparticle masses has
been included in the ISAJET SUSY spectrum calculation, according to the
formulae given by Pierce {\it et al.}, Nucl.\ Phys.\ {\bf B491}, 3
(1997).  Previous versions of ISAJET had included the logarithmic 1-loop
corrections by freezing out soft mass parameters at a scale equal to
their mass. The current calculation includes all finite corrections as
well. Crucial contributions to the encoding of these expressions were
made by Tadas Krupovnickas. We have also adjusted slightly the input
$\overline{DR}$ gauge couplings at $M_Z$ to coincide with measured
central values.  Logarithmic threshold corrections to gauge and Yukawa
couplings are accounted for as in previous ISAJET versions by changing
the beta functions in the RGEs as various soft mass thresholds are
passed. We have also enlarged the set of SUSY RGEs to include running
of the Higgs field vevs. The net effect of these changes is to modify
various sparticle masses by typically 1-5\% from the predictions of
ISAJET 7.64.

The size of the small mass splitting between $\tilde\chi_1^\pm$ and
$\tilde\chi_1^0$ is important for AMSB phenomenology. Rather than
extracting this splitting from the general result, for the AMSB model we
calculate it from the tree and the 1-loop vector boson loop graphs
[taken from Cheng, Dobrescu and Matchev, Nucl.\ Phys.\ {\bf B543}, 47,
(1999)] and use it to set the $\tilde\chi_1^\pm$ mass.

We have modified the RGE solution to use the previous SUSY masses to
compute $\mu$ before computing the new SUSY masses and from them the
loop corrections to $\mu$. This seems to improve the convergence, e.g.,
in the focus point, or hyperbolic branch region, of the mSUGRA model. 
The RGE solution still does not converge properly in isolated regions.

We have also corrected a sign mistake in the $A$-term boundary
conditions for the anomaly-mediated SUSY breaking model.

\subsection{Version~7.64, September 2002}

     The talk by Sabine Kraml at SUSY02 stressed the sensitivity of the
allowed region of radiative electroweak symmetry breaking to fine
details of the calculation. We have reexamined the issue and corrected
several problems. A coding error in the Passarino-Veltman function $B_1$,
\verb|SSB1| has been fixed. The requirement $\mu^2>0$ is not imposed
until after the solution of the renormalization group equations has
stabilized. The value of $m_b(m_Z)$ has been updated to 2.83~GeV. A
running gluino mass has been used in the top and bottom self-energy,
along with $\alpha_s$ evaluated at a higher scale. Since the parameters
of the Higgs potential vary rapidly near the weak scale, the convergence
requirements on these have been loosened, while the requirements on the
other parameters have been somewhat tightened. The combined effect of
these changes is to shift the boundary for radiative electroweak
symmetry breaking to higher scalar masses.

     Stephan Lammel found that the matrix element used to generate
$\tilde g \to \tilde\chi_i^\pm q \bar q$ was missing some poles,
although the branching ratio was correct. This has been fixed.

     The radiative decays $\tilde\chi_i^0 \to \tilde\chi_j^0 \gamma$ 
have been included.

\subsection{Version~7.63, April 2002}

     The SUSY mass calculations have been improved, especially for $M_A$
in terms of other SUSY parameters, by using the MSSM Yukawa couplings
from the renormalization group equations. The numerical precision of the
solution to the SUSY renormalization group equations has also been
improved; this should give better stability near the boundaries of the
allowed regions. The complete 1-loop self-energies for the $t$, $b$, and
$\tau$ have been included from Pierce, Bagger, Matchev, and Zhang,
Nucl.\ Phys.\ B491, 3 (1997). Finally, a number of bugs have been fixed,
including one in the $\tau$ decay of $t$ quarks.

\subsection{Version~7.58, August 2001}

     The CTEQ5L parton distributions have been added and made the
default. 

     Keywords \verb|NOB| and \verb|NOTAU| have been added to turn off
$B$ and $\tau$ decays so that an external decay package such as QQ or
TAUOLA can be used. To preserve the polarization information for
$\tau$'s, separate (non-standard) IDENT codes for $\tau_L$ and $\tau_R$
are used for \verb|NOTAU|. The user must provide an appropriate
interface.

     The RANLUX random number generator has been added as a compile time
option. It has a very long period, and any 32-bit integer seed gives an
independent sequence. If RANLUX is used, the keyword SEED takes the
integer seed plus two additional integers that are normally zero but can
be used to restart the generator. See the CERN Program Library writeups
for more information.

     Right-handed neutrinos are now included if the keyword
\verb|SUGRHN| is used. The user must input the 3rd generation neutrino
mass (at scale $M_Z$), the intermediate scale right handed neutrino
Majorana mass $M_N$, and the soft SUSY-breaking masses $A_n$ and
$m_{\tilde\nu_R}$ at the GUT scale. Then the neutrino Yukawa coupling is
computed in the simple see-saw model, and renormalization group
evolution includes these effects between $M_{GUT}$ and $M_N$.

     The decays $\tilde g \to \tilde W_i u \bar d$ have been updated to
include non-degenerate squark masses. The arbitrary width for $\tilde t
\to \tilde Z_1 c$ used previously has been replaced by the calculated
value. $\overline{DR}$ masses are used consistently. Yukawa couplings in
the the SUGRA routine are now calculated in the $\overline{DR}$
regularization scheme to be consistent with two loop renormalization
group evolution.

     In solving the SUSY renormalization group equations, the
requirement of good electroweak symmetry breaking is imposed only at the
end. Previously a point could be rejected if there was no symmetry
breaking even in the initial iteration with a truncated set of
equations.

     The sign of the $A$ term in the AMSB model has been corrected.

     The Standard Model process $e^+e^- \to ZH$ was missing and has been
added.

     Function ITRANS has been updated to reflect the current PDG
particle codes for SUSY particles.

\subsection{Version~7.51, May 2000}

        Several improvements in the SUSY RGE's have been made. All
two-loop terms including both gauge and Yukawa couplings and the
contributions from right-handed neutrinos are now included. There is a
new keyword \verb|SSBCSC| to specify a scale other than the GUT scale
for the RGE boundary conditions.

        The process $Z+\gamma$ is now included in \verb|WPAIR|. (This
was omitted because it has no contribution from triple gauge boson
couplings.)

        An incorrect type declaration produced unphysical results for
beamsstrahlung on some computers. This has been fixed. While the bug is
serious for $e^+e^-$ with the \verb|EEBEAM| option, it has no effect on
other processes. Some other minor bugs have also been fixed.

\subsection{Version~7.47, December 1999}

        There are several improvements in the treatment of
supersymmetry. The Anomaly Mediated SUSY Breaking model of of Randall
and Sundrum and of Gherghetta, Giudice, and Wells (hep-ph/9904378) has
been added. The parameters of the model are a universal scalar mass
$m_0$ at the GUT scale, a gravitino mass $m_{3/2}$, and the usual
$\tan\beta$ and $\sgn\mu$. These are set by the \verb|AMSB| keyword. The
renormalization group equations have been extended to include two-loop
Yukawa terms and right-handed sneutrinos (with default masses above the
Planck scale). The $\tilde\nu_R$ play a role in the evolution for the
inverted hierarchy models of Bagger, Feng, and Polonsky, hep-ph/9905292.
SUSY loop corrections to Yukawa couplings have been incorporated in the
SUSY mass calculations.

        The Helas library of Murayama, Watanabe, and Hagiwara has been
incorporated together with a simple multi-body phase space generator.
This makes it possible to use code generated by MadGraph to produce
multi-body hard scattering processes. As a first example, a \verb|ZJJ|
process that generates $Z + \hbox{2 jets}$ has been added, with the $Z$
treated as a narrow resonance. Additional processes may be added in
future releases.

        A new \verb|EXTRADIM| process has been added to generate
Kaluza-Klein graviton production in association with a jet or photon in
models with extra dimensions at the TeV scale. The cross sections are
from G.F.Giudice et al., hep-ph/9811291. We thank I. Hinchliffe and L.
Vacavant for providing this.

        A number of bugs have been fixed, including in particular one in
the decay $\widetilde W_i \to \widetilde Z_j \tau \nu$.

\subsection{Version~7.44, April 1999}

        A serious bug introduced in Version~7.42 that could lead to
matrix elements being stored for the wrong mode has been corrected.
Some sign errors in the matrix elements for gaugino decays have also
been corrected.

\subsection{Version~7.42, January 1999}

        Beginning with this version, matrix elements are taken into
account in the event generator as well as in the calculation of decay
widths for MSSM three-body decays of the form $\tilde A \to \tilde B f
\bar f$, where $\tilde A$ and $\tilde B$ are gluinos, charginos, or
neutralinos. This is implemented by having ISASUSY save the poles and
their couplings when calculating the decay width and then using these
to reconstruct the matrix element. Other three-body decays may be
included in the future. Decays selected with \verb|FORCE| use the
appropriate matrix elements.

        As part of the changes to implement these matrix elements, the
format of the decay table has changed. It now starts with a header
line; if this does not match the internal version, then a warning is
printed. The decay table now includes an index MELEM that specifies the
matrix element to be used for all processes. This is also used for
\verb|FORCE| decays and is printed on the run listing for them. SUSY
3-body decays have internally generated negative values of MELEM.

        This version also includes both initial state radiation and
beamstrahlung for $e^+e^-$ interactions. For initial state radiation
(bremsstrahlung), if the \verb|EEBREM| keyword is selected, an electron
structure function will be used. For a convolution of both
bremsstrahlung and beamstrahlung, the keyword \verb|EEBEAM| must be
used, with appropriate inputs (see documentation).

\subsection{Version~7.40, October 1998}

        A new process WHIGGS generates $W^\pm+H$ and $Z+H$ events for
both the Standard Model and SUSY models and also Higgs pair production
for SUSY models. The types and $W$ decay modes are selected with
JETTYPE and WMODE as for WPAIR events. This process is of particular
interest for producing fairly light Higgs bosons at the Tevatron. See
the documentation for more details.

        Some non-minimal GMSB models can be generated using a new
keyword GMSB2. The optional parameters are an extra factor between the
gaugino and scalar masses, shifts in the Higgs masses, a $D$-term
proportional to hypercharge, and independent numbers of messenger
fields for the three gauge groups. The documentation gives more
details and references.

        The default for SUGRA models has been changed to use
$\alpha_s(M_Z)=0.118$, the experimental value. This means that the
couplings do not exactly unify at the GUT scale, presumably because of
the effects of heavy particles. The keyword AL3UNI can be used to
select exact unification, which produces too large a value for
$\alpha_s(M_Z)$.

        A number of three-body slepton decays that occur through
left-right mixing are now included. These are obviously small but
might compete with gravitino decays. In particular, a decay like
$\tilde\mu_R \to \tilde\tau_1 \nu\bar\nu$ might lead to a wrong
momentum measurement in the muon system. So far we have found no case
in which this is probable.

        The new release also includes a separate Unix tar file
\verb|mcpp.tar| containing C++ code to read a standard ISAJET output
file and copy all the information into C++ classes. The tar file
contains makefiles for Software Release Tools, documentation, and
examples as well as the code.

\subsection{Version~7.37, April 1998}

        Version~7.37 incorporates Gauge Mediated SUSY Breaking models
for the first time. In these models, SUSY is broken in a hidden sector
at a relatively low scale, and the masses of the MSSM fields are then
produced through ordinary gauge interactions with messenger fields.
The parameters of the GMSB model in ISAJET are $M_m$, the messenger
mass scale; $\Lambda_m = F_m/M_m$, where $F_m$ is the SUSY breaking
scale in the messenger sector; $N_5$, the number of messenger fields;
the usual $\tan\beta$ and $\sgn\mu$; and $C_{\rm grav} \ge 1$, a
factor which scales the gravitino mass and hence the lifetime for the
lightest MSSM particle to decay into it.

        GMSB models have a light gravitino $\tilde G$ as the lightest
SUSY particle. The phenomenology of the model depends mainly on the
nature of the next lightest SUSY particle, a $\tilde\chi_1^0$ or a
$\tilde\tau_1$, which changes with the number $N_5$ of messengers. The
phenomenology also depends on the lifetime for the $\tilde\chi_1^0 \to
\tilde G \gamma$ or $\tilde\tau_1 \to \tilde G \tau$ decay; this
lifetime can be short or very long. All the relevant decays are
included except for $\tilde\mu \to \nu \nu \tilde\tau_1$, which is very
suppressed.

        The keyword MGVTNO allows the user to independently input a
gravitino gravitino mass for the MSSM option. This allows studies of
SUGRA (or other types) of models where the gravitino is the LSP.

        Version~7.37 also contains an extension of the SUGRA model
with a variety of non-universal gaugino and sfermion masses and $A$
terms at the GUT scale. This makes it possible to study, for example,
how well the SUGRA assumptions can be tested.

        Two significant bugs have also been corrected. The decay modes
for $B^*$ mesons were missing from the decay table since Version~7.29
and have been restored. A sign error in the interference term for
chargino production has been corrected, leading to a larger chargino
pair cross section at the Tevatron.

\subsection{Version 7.32, November 1997}

        This version makes several corrections in various chargino and
neutralino widths, thus changing the branching ratios for large
$\tan\beta$. For $\tilde\chi_2^0$, for example, the $\tilde\chi_1^0
b\bar b$ branching ratio is decreased significantly, while the
$\tilde\chi_1^0 \tau^+ \tau^-$ one is increased. Thus the SUGRA
phenomenology for $\tan\beta \sim 30$ is modified substantially.

        The new version also fixes a few bugs, including a possible
numerical precision problem in the Drell-Yan process at high mass and
$q_T$. It also includes a missing routine for the Zebra interface.

\subsection{Version 7.31, August 1997}

        Version fixes a couple of bugs in Version~7.29. In
particular, the JETTYPE selection did not work correctly for
supersymmetric Higgs bosons, and there was an error in the interactive
interface for MSSM input. Since these could lead to incorrect results,
users should replace the old version. We thank Art Kreymer for finding
these problems. 

        Since top quarks decay before they have time to hadronize,
they are now put directly onto the particle list. Top hadrons ($t\bar
u$, $t\bar d$, etc.) no longer appear, and FORCE should be used
directly for the top quark, i.e.
\begin{verbatim}
FORCE
6,11,-12,5/
\end{verbatim}

        The documentation has been converted to LaTeX. Run either
LaTeX~2.09 or LaTeX~2e three times to resolve all the forward
references. Either US (8.5x11 inch) or A4 size paper can be used.

\subsection{Version 7.30, July 1997}

        This version fixes a couple of bugs in the previous version.
In particular, the JETTYPE selection did not work correctly for
supersymmetric Higgs bosons, and there was an error in the interactive
interface for MSSM input. Since these could lead to incorrect results,
users should replace the old version. We thank Art Kreymer for finding
these problems. 

        Since top quarks decay before they have time to hadronize,
they are now put directly onto the particle list. Top hadrons ($t\bar
u$, $tud$, etc.) no longer appear, and FORCE should be used directly
for the top quark, i.e.
\begin{verbatim}
FORCE
6,11,-12,5/
\end{verbatim}

        The documentation has been converted to \LaTeX. Run either
\LaTeX~2.09 or \LaTeX~2e three times to resolve all the forward
references. Either US ($8.5\times11$~inch) or A4 size paper can be
used.

\subsection{Version 7.29, May 1997}

      While the previous version was applicable for large as well as
small $\tan\beta$, it did contain approximations for the 3-body decays
$\tilde g \to t \bar b \tilde W_i$, $\tilde Z_i \to b \bar b \tilde
Z_j, \tau \tau \tilde Z_j$, and $\tilde W_i \to \tau \nu \tilde Z_j$.
The complete tree-level calculations for three body decays of the
gluino, chargino and neutralino, with all Yukawa couplings and
mixings, have now been included (thanks mainly to M. Drees).  We have
compared our branching ratios with those calculated by A.~Bartl and
collaborators; the agreement is generally good.

      The decay patterns of gluinos, charginos and neutralinos may
differ from previous expectations if $\tan\beta$ is large.  In
particular, decays into $\tau$'s and $b$'s are often enhanced, while
decays into $e$'s and $\mu$'s are reduced.  It could be important for
experiments to study new types of signatures, since the cross sections
for conventional signatures may be considerably reduced.

      We have also corrected several bugs, including a fairly
serious one in the selection of jet types for SUSY Higgs. We thank
A.~Kreymer for pointing this out to us.

\subsection{Version 7.27, January 1997}

      The new version contains substantial improvements in the
treatment of the Minimal Supersymmetric Standard Model (MSSM) and the
SUGRA model.  The squarks of the first two generations are no longer
assumed to be degenerate.  The mass splittings and all the two-body
decay modes are now correctly calculated for large $\tan\beta$.  While
there are still some approximations for three-body modes, ISAJET is
now usable for the whole range $1 \simle \tan\beta \simle M_t/M_b$.  The
most interesting new feature for large $\tan\beta$ is that third
generation modes can be strongly enhanced or even completely dominant.

      To accomodate these changes it was necessary to change the
MSSM input parameters.  To avoid confusion, the MSSM keywords have
been renamed MSSM[A-C] instead of MSSM[1-3], and the order of the
parameters has been changed.  See the input section of the manual for
details.

      Treatment of the MSSM Higgs sector has also been improved.  In
the renormalization group equations the Higgs couplings are frozen at
a higher scale, $Q = \sqrt{M(\tilde t_L)M(\tilde t_R)}$.  Running
$t$, $b$ and $\tau$ masses evaluated at that scale are used to
reproduce the dominant 2-loop effects.  There is some sensitivity to
the choice of $Q$; our choice seems to give fairly stable results over
a wide range of parameters and reasonable agreement with other
calculations.  In particular, the resulting light Higgs masses are
significantly lower than those from Version~7.22.  

      The default parton distributions have been updated to CTEQ3L.
A bug in the PDFLIB interface and other minor bugs have been fixed.

\subsection{Version 7.22, July 1996}

      The new version fixes errors in $\tilde b \to \tilde W t$ and in
some $\tilde t$ decays and Higgs decays. It also contains a new decay
table with updated $\tau$, $c$, and $b$ decays, based loosely on the
QQ decay package from CLEO.  The updated decays are less detailed than
the full CLEO QQ program but an improvement over what existed before.
The new decays involve a number of additional resonances, including
$f_0(980)$, $a_1(1260)$, $f_2(1270)$, $K_1(1270)$, $K_1^*(1400)$,
$K_2^*(1430)$, $\chi_{c1,2,3}$, and $\psi(2S)$, so users may have to
change their interface routines.

      A number of other small bugs have been corrected.

\subsection{Version 7.20, June 1996}

      The new version corrects both errors introduced in Version~7.19
and longstanding errors in the final state QCD shower algorithm. It
also includes the top mass in the cross sections for $g b \to W t$ and
$g t \to Z t$. When the $t$ mass is taken into account, the process $g
t \to W b$ can have a pole in the physical region, so it has been
removed; see the documentation for more discussion. 

        Steve Tether recently pointed out to us that the anomalous
dimension for the $q \to q g$ branching used in the final state QCD
branching algorithm was incorrect. In investigating this we found an
additional error, a missing factor of $1/3$ in the $g \to q \bar q$
branching. The first error produces a small but non-negligible
underestimate of gluon radiation from quarks. The second overestimates
quark pair production from gluons by about a factor of 3. In
particular, this means that backgrounds from heavy quarks $Q$ coming
from $g \to Q \bar Q$ have been overestimated.

      The new version also allows the user to set arbitrary masses
for the $U(1)$ and $SU(2)$ gaugino mases in the MSSM rather than
deriving these from the gluino mass using grand unification. This
could be useful in studying one of the SUSY interpretations of a CDF
$ee\gamma\gamma\etmiss$ event recently suggested by Ambrosanio, Kane,
Kribs, Martin and Mrenna.  Note, however, that radiative decay are
{\it not} included, although the user can force them and multiply by
the appropriate branching ratios calculated by Haber and Wyler,
Nucl.{} Phys.{} B323, 267 (1989). No explicit provision for the decay
$\tilde Z_1 \to \tilde G \gamma$ of the lightest zino into a gravitino
or goldstino and a photon has been made, but forcing the decay $\tilde
Z_1 \to \nu\gamma$ has the same effect for any collider detector.

      A number of other minor bugs have also been corrected. 

\subsection{Version 7.16, October 1995}

       The new version includes $e^+e^-$ cross sections for both SUSY
and Standard Model particles with polarized beams. The $e^-$ and $e^+$
polarizations are specified with a new keyword EPOL. Polarization
appears to be quite useful in studying SUSY particles at an $e^+e^-$
collider.

      The new release also includes some bug fixes for $pp$ reactions,
so you should upgrade even if you do not plan to use the polarized
$e^+e^-$ cross sections.

\subsection{Version 7.13, September 1994}

      Version 7.13 of ISAJET fixes a bug that we introduced in the
recently released 7.11 and another bug in $\tilde g \to \tilde q \bar
q$. We felt it was essential to fix these bugs despite the
proliferation of versions.

      The new version includes the cross sections for the $e^+e^-$
production of squarks, sleptons, gauginos, and Higgs bosons in Minimal
Supersymmetric Standard Model (MSSM) or the minimal supergravity
(SUGRA) model, including the effects of cascade decays. To generate
such events, select the \verb|E+E-| reaction type and either SUGRA or
MSSM, e.g.,
\begin{verbatim}
SAMPLE E+E- JOB
300.,50000,10,100/
E+E-
SUGRA
100,100,0,2,-1/
TMASS
170,-1,1/
END
STOP
\end{verbatim}
The effects of spin correlations in the production and decay, e.g., in
$e^+e^- \to \widetilde W_1^+ \widetilde W_1^-$, are not included. 

      It should be noted that the Standard Model $e^+e^-$ generator in
ISAJET does not include Bhabba scattering or $W^+W^-$ and $Z^0Z^0$
production. Also, its hadronization model is cruder than that
available in some other generators.

\subsection{Version 7.11, September 1994}

      The new version includes the cross sections for the $e^+e^-$
production of squarks, sleptons, gauginos, and Higgs bosons in Minimal
Supersymmetric Standard Model (MSSM) or the minimal supergravity
(SUGRA) model including the effects of cascade decays. To generate
such events, select the \verb|E+E-| reaction type and either SUGRA or
MSSM, e.g.,
\begin{verbatim}
SAMPLE E+E- JOB
300.,50000,10,100/
E+E-
SUGRA
100,100,0,2,-1/
TMASS
170,-1,1/
END
STOP
\end{verbatim}
The effects of spin correlations in the production and decay, e.g., in
$e^+e^- \to \widetilde W_1^+ \widetilde W_1^-$, are not included. 

      It should be noted that the Standard Model $e^+e^-$ generator in
ISAJET does not include Bhabba scattering or $W^+W^-$ and $Z^0Z^0$
production. Also, its hadronization model is cruder than that
available in some other generators.

\subsection{Version 7.10, July 1994}

       This version adds a new option that solves the renormalization group
equations to calculate the Minimal Supersymmetric Standard Model (MSSM)
parameters in the minimal supergravity (SUGRA) model, assuming only that the
low energy theory has the minimal particle content, that electroweak
symmetry is radiatively broken, and that R-parity is conserved.  The minimal
SUGRA model has just four parameters, which are taken to be the common
scalar mass $m_0$, the common gaugino mass $m_{1/2}$, the common trilinear
SUSY breaking term $A_0$, all defined at the GUT scale, and $\tan\beta$; the
sign of $\mu$ must also be given.  The renormalization group equations are
solved iteratively using Runge-Kutta integration including the correct
thresholds.  This program can be used either alone or as part of the event
generator.  In the latter case, the parameters are specified using
\begin{verse}
SUGRA\\
$m_0$, $m_{1/2}$, $A_0$, $\tan\beta$, $\sgn\mu$
\end{verse}
While the SUGRA option is less general than the MSSM, it is theoretically
attractive and provides a much more managable parameter space.

      In addition there have been a number of improvements and bug fixes.  An
occasional infinite loop in the minimum bias generator has been fixed.  A few
SUSY cross sections and decay modes and the JETTYPE flags for SUSY
particles have been corrected.  The treatment of $B$ baryons has been
improved somewhat.


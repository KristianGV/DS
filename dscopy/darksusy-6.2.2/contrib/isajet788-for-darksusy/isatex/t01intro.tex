\section{Introduction\label{INTRO}}

      ISAJET is a Monte Carlo program which simulates $pp$, 
$\bar pp$ and $e^+e^-$ interactions at high energies. 
ISAJET is based on
perturbative QCD plus phenomenological models for parton and beam jet
fragmentation. Events are generated in four distinct steps:
\begin{itemize}
\item A primary hard scattering is generated according to the
appropriate QCD cross section.
\item QCD radiative corrections are added for both the initial and the
final state.
\item Partons are fragmented into hadrons independently, and particles
with lifetimes less than about $10^{-12}$ seconds are decayed.
\item Beam jets are added assuming that these are identical to a
minimum bias event at the remaining energy.
\end{itemize}

      ISAJET incorporates ISASUSY, which evaluates branching ratios for
the minimal supersymmetric extension of the standard model. H.~Baer and
X.~Tata are coauthors of this package, and they have done the original
calculations with various collaborators. See the ISASUSY documentation
in the patch Section~\ref{SUSY}.

      ISAJET is now mainly developed and tested on Linux and Mac OSX
with the \verb|gfortran| compiler. Earlier versions have run on various
Unix systems, IBM VM/CMS, DEC VMS, and CDC 7600. ISAJET is written
mainly in ANSI standard FORTRAN 77, but it does contain some extensions
except in the ANSI version. The code is maintained with a combination of
RCS, the Revision Control System, \verb|cpp|, the C Preprocessor; see
Section~\ref{INSTALL} for details. A tar file is available from
\begin{verbatim}
http://www.nhn.ou.edu/~isajet
\end{verbatim}

      ISAPLT contains the skeleton of an HBOOK histogramming job, a
trivial calorimeter simulation, and a jet-finding algorithm. These are
provided for convenience only and are not supported.

\newpage
\section{Patchy and PAM Organization\label{PATCHY}}

      Patchy is a code management system developed at CERN and used to
maintain the CERN Library. It is used to provide versions of ISAJET for
a wide variety of computers. Instructions for using PATCHY are
available from \verb|http://wwwinfo.cern.ch/asdoc/Welcome.html|.

      A master source file in Patchy is called a ``PAM.'' The ISAJET
PAM contains all the source code and documentation plus Patchy
commands to include common blocks and to select the desired version. It
is divided into the following patches: 

      \verb|ISACDE|: contains all common blocks, etc. These are divided
into decks based on their usage.

      \verb|ISADATA|: contains block data ALDATA. This must always be
loaded when using ISAJET.

      \verb|ISAJET|: contains the code for generating events. Each
subroutine is in a separate deck with the same name.

      \verb|ISASSRUN|: contains the main program for ISASUSY, which
prompts for input parameters and prints out all the decay modes. It is
selected by \verb|*ISASUSY|.

      \verb|ISASUSY|: contains code to calculate all the decay widths
and branching fractions in the minimal supersymmetric model.

      \verb|ISATAPE|: contains the code for reading and writing tapes,
again with each subroutine on a separate deck.

      \verb|ISARUN|: contains a main program and a simple interactive
interface.  It is selected by \verb|IF=INTERACT|.

      \verb|ISAZEB|: contains Zebra format output routines, an
alternative to the ISATAPE routines.

      \verb|ISZRUN|: contains the analog of ISAPLT for the Zebra
format. 

      \verb|ISAPLT|: contains a simple calorimeter simulation and the
skeleton of a histogramming job using HBOOK.

      \verb|ISATEXT|: contains the instructions for using ISAJET, i.e.
the text of this document. It also includes the documentation for
ISASUSY.

      \verb|ISZTEXT|: contains the instructions for the Zebra output
routines and a description of the Zebra banks.

      \verb|ISADECAY|: contains the input decay table.

      The code is actually maintained using RCS on a Silicon Graphics
computer at BNL. Patchy is used primarily to handle common blocks and
machine dependent code.

      The input to YPATCHY must contain both \verb|+USE| cards, which
define the wanted program version, and \verb|+EXE| cards, which
determine which patches or decks are written to the ASM file. To
facilitate this selection, the ISAJET PAM contains the following pilot
patches:

      \verb|*ISADECAY|: USE selects ISADECAY and all corrections to it.

      \verb|*ISAJET|: USE selects ISACDE, ISADATA, ISAJET, ISATAPE,
ISARUN and all corrections to them. Note that ISARUN is not actually
selected without \verb|+USE,INTERACT|.

      \verb|*ISAPLT|: USE selects ISACDE, ISAPLT, and all corrections
to them.

      \verb|*ISASUSY|: USE selects CDESUSY, ISASUSY, and ISASSRUN to
create a program to calculate all the MSSM decay modes.

      \verb|*ISATEXT|: USE selects ISACDE, ISATEXT, and all corrections
to them. 

      \verb|*ISAZEB|: USE selects ISAJET with a Zebra output format.

      \verb|*ISZRUN|: USE selects the Zebra analysis package.

      Patches are provided to select the machine dependent features for
specific computers or operating systems:

      \verb|ANSI|: ANSI standard Fortran (no time or date functions)

      \verb|APOLLO|: APOLLO -- only tested by CERN

      \verb|CDC|: CDC 7600 and 60-bit CYBER (obsolete)

      \verb|CRAY|: CRAY with UNICOS

      \verb|DECS|: DEC Station with Ultrix 

      \verb|ETA|: ETA 10 running Unix System V (obsolete)

      \verb|HPUX|: HP/9000 7xx running Unix System V

      \verb|IBM|: IBM 370 and 30xx running VM/CMS 

      \verb|IBMRT|: IBM RS/6000 running AIX 3.x or 4.x

      \verb|LINUX|: PC running Linux with f2c/gcc or g77 compiler

      \verb|SGI|: Silicon Graphics running IRIX

      \verb|SUN|: Sun Sparcstation running SUNOS or Solaris

      \verb|VAX|: DEC VAX or Alpha running VMS

\noindent These patches in turn select a variety of patches and IF
flags, allowing one to select more specific features, as indicated
below. (Replace \verb|&| by \verb|+| everywhere.)
\begin{verbatim}
&PATCH,ANSI.                      GENERIC ANSI FORTRAN.
&USE,DOUBLE.                      DOUBLE PRECISION.
&USE,STDIO.                       STANDARD FORTRAN 77 TAPE INPUT/OUTPUT.
&USE,MOVEFTN.                     FORTRAN REPLACEMENT FOR MOVLEV.
&USE,RANFFTN,IF=-CERN.            FORTRAN RANF.
&USE,RANFCALL.                    STANDARD RANSET AND RANGET CALLS.
&USE,NORANLUX,IF=-RANLUX.         NO RANLUX RANDOM NUMBERS.
&USE,NOCERN,IF=-CERN.             NO CERN LIBRARY.
&EOD

&PATCH,APOLLO.
&DECK,BLANKDEK.
&USE,DOUBLE.                      DOUBLE PRECISION.
&USE,STDIO.                       STANDARD FORTRAN 77 TAPE INPUT/OUTPUT.
&USE,MOVEFTN.                     FORTRAN REPLACEMENT FOR MOVLEV.
&USE,RANFFTN,IF=-CERN.            FORTRAN RANF.
&USE,RANFCALL.                    STANDARD RANSET AND RANGET CALLS.
&USE,NORANLUX,IF=-RANLUX.         NO RANLUX RANDOM NUMBERS.
&USE,NOCERN,IF=-CERN.             NO CERN LIBRARY.
&USE,IMPNONE.                     IMPLICIT NONE
&EOD.

&PATCH,CDC.                       CDC 7600 OR CYBER 175.
&USE,SINGLE.                      SINGLE PRECISION.
&USE,LEVEL2.                      LEVEL 2 STORAGE.
&USE,CDCPACK.                     PACK 2 WORDS PER WORD FOR INPUT/OUTPUT.
&USE,RANFCALL.                    STANDARD RANSET AND RANGET CALLS.
&USE,NOCERN,IF=-CERN.             NO CERN LIBRARY.
&EOD

&PATCH,CRAY.                      CRAY XMP OR 2.
&USE,SINGLE.                      SINGLE PRECISION.
&USE,STDIO.                       STANDARD FORTRAN 77 TAPE INPUT/OUTPUT.
&USE,MOVEFTN.                     FORTRAN REPLACEMENT FOR MOVLEV.
&USE,NOCERN,IF=-CERN.             NO CERN LIBRARY.
&EOD

&PATCH,DECS.                      DEC STATION (ULTRIX)
&USE,SUN.
&EOD

&PATCH,ETA.                       ETA-10.
&USE,SINGLE.                      SINGLE PRECISION.
&USE,STDIO.                       STANDARD FORTRAN 77 TAPE INPUT/OUTPUT.
&USE,MOVEFTN.                     FORTRAN REPLACEMENT FOR MOVLEV.
&USE,RANFCALL.                    STANDARD RANSET AND RANGET CALLS.
&USE,NOCERN,IF=-CERN.             NO CERN LIBRARY.
&EOD

&PATCH,HPUX.                      HP/9000 7XX RUNNING UNIX.
&USE,DOUBLE.                      DOUBLE PRECISION.
&USE,STDIO.                       STANDARD FORTRAN 77 TAPE INPUT/OUTPUT.
&USE,MOVEFTN.                     FORTRAN REPLACEMENT FOR MOVLEV.
&USE,RANFFTN,IF=-CERN.            FORTRAN RANF.
&USE,RANFCALL.                    STANDARD RANSET AND RANGET CALLS.
&USE,NORANLUX,IF=-RANLUX.         NO RANLUX RANDOM NUMBERS.
&USE,NOCERN,IF=-CERN.             NO CERN LIBRARY.
&USE,IMPNONE.                     IMPLICIT NONE
&EOD

&PATCH,IBM.                       IBM 370 OR 30XX.
&USE,DOUBLE.                      DOUBLE PRECISION.
&USE,STDIO.                       STANDARD FORTRAN 77 TAPE INPUT/OUTPUT.
&USE,MOVEFTN.                     FORTRAN REPLACEMENT FOR MOVLEV.
&USE,RANFFTN,IF=-CERN.            FORTRAN RANF.
&USE,RANFCALL.                    STANDARD RANSET AND RANGET CALLS.
&USE,NORANLUX,IF=-RANLUX.         NO RANLUX RANDOM NUMBERS.
&USE,NOCERN,IF=-CERN.             NO CERN LIBRARY.
&EOD

&PATCH,IBMRT.                     IBM RS/6000 WITH AIX 3.1
&USE,DOUBLE.                      DOUBLE PRECISION.
&USE,STDIO.                       STANDARD FORTRAN 77 TAPE INPUT/OUTPUT.
&USE,MOVEFTN.                     FORTRAN REPLACEMENT FOR MOVLEV.
&USE,RANFFTN,IF=-CERN.            FORTRAN RANF.
&USE,RANFCALL.                    STANDARD RANSET AND RANGET CALLS.
&USE,NORANLUX,IF=-RANLUX.         NO RANLUX RANDOM NUMBERS.
&USE,NOCERN,IF=-CERN.             NO CERN LIBRARY.
&USE,IMPNONE.                     IMPLICIT NONE
&EOD

&PATCH,LINUX.                     IBM PC WITH LINUX 1.X
&USE,DOUBLE.                      DOUBLE PRECISION.
&USE,STDIO.  STANDARD FORTRAN 77 TAPE INPUT/OUTPUT.
&USE,MOVEFTN.                     FORTRAN REPLACEMENT FOR MOVLEV.
&USE,RANFFTN,IF=-CERN.            FORTRAN RANF.
&USE,RANFCALL.                    STANDARD RANSET AND RANGET CALLS.
&USE,NORANLUX,IF=-RANLUX.         NO RANLUX RANDOM NUMBERS.
&USE,NOCERN,IF=-CERN.             NO CERN LIBRARY.
&USE,IMPNONE.                     IMPLICIT NONE
&EOD

&PATCH,SGI.
SILICON GRAPHICS 4D/XX.
&USE,DOUBLE.                      DOUBLE PRECISION.
&USE,STDIO.                       STANDARD FORTRAN 77 TAPE INPUT/OUTPUT.
&USE,MOVEFTN.                     FORTRAN REPLACEMENT FOR MOVLEV.
&USE,RANFFTN,IF=-CERN.            FORTRAN RANF.
&USE,RANFCALL.                    STANDARD RANSET AND RANGET CALLS.
&USE,NORANLUX,IF=-RANLUX.         NO RANLUX RANDOM NUMBERS.
&USE,NOCERN,IF=-CERN.             NO CERN LIBRARY.
&EOD

&PATCH,SUN.                       SUN (SPARC)
&USE,DOUBLE.                      DOUBLE PRECISION.
&USE,STDIO.                       STANDARD FORTRAN 77 TAPE INPUT/OUTPUT.
&USE,MOVEFTN.                     FORTRAN REPLACEMENT FOR MOVLEV.
&USE,RANFFTN,IF=-CERN.            FORTRAN RANF.
&USE,RANFCALL.                    STANDARD RANSET AND RANGET CALLS.
&USE,NORANLUX,IF=-RANLUX.         NO RANLUX RANDOM NUMBERS.
&USE,NOCERN,IF=-CERN.             NO CERN LIBRARY.
&EOD

&PATCH,VAX.                       DEC VAX 11/780 OR 8600.
&USE,DOUBLE.                      DOUBLE PRECISION.
&USE,STDIO.                       STANDARD FORTRAN 77 TAPE INPUT/OUTPUT.
&USE,MOVEFTN.                     FORTRAN REPLACEMENT FOR MOVLEV.
&USE,RANFFTN,IF=-CERN.            FORTRAN RANF.
&USE,RANFCALL.                    STANDARD RANSET AND RANGET CALLS.
&USE,NORANLUX,IF=-RANLUX.         NO RANLUX RANDOM NUMBERS.
&USE,NOCERN,IF=-CERN.             NO CERN LIBRARY.
&USE,IMPNONE.                     IMPLICIT NONE
&EOD
\end{verbatim}

      An empty patch INTERACT selects a main program and an interactive
interface which will prompt the user for parameters and do some error
checking. A patch CERN allows ISAJET to take the random number generator
RANF and several other routines from the CERN Library; to use this
include the Patchy command
\begin{verbatim}
&USE,CERN.
\end{verbatim}
Similarly, a patch PDFLIB enables the interface to the PDFLIB parton
distribution compilation by H. Plothow-Besch:
\begin{verbatim}
&USE,PDFLIB
\end{verbatim}
The only internal links with PDFLIB are calls to the routines PDFSET,
PFTOPDG, and DXPDF, and the common blocks W50510 and W50517,
\begin{verbatim}
\end{verbatim}
\begininc{w50510.inc}
\end{verbatim}
\begininc{w50517.inc}
\end{verbatim}
which are used to specify the level of output messages and the logical
unit number for them. A patch RANLUX selects the RANLUX random number
generator instead of the 48-bit congruential generator RANF:
\begin{verbatim}
&USE,RANLUX
\end{verbatim}
RANLUX has a very long period, and each 32-bit integer seed gives an
independent sequence.

      In general it should be sufficient to run YPATCHY with the
following cradle (replace \verb|&| with \verb|+| everywhere):
\begin{verbatim}
&USE,(*ISAJET,*ISATEXT,*ISADECAY,*ISAPLT).     CHOOSE ONE.
&USE,ANSI,DECS,HPUX,IBM,IBMRT,SGI,SUN,....     CHOOSE ONE.
&[USE,INTERACT].                               FOR INTERACTIVE MODE.
&[USE,CERN.]                                   FOR CERN LIBRARY.
&[USE,HBOOK3.]                                 HBOOK 3 FOR ISAPLT.
&EXE.
&PAM.
&QUIT.
\end{verbatim}

      The input to YPATCHY can also contain changes by the user. It is
suggested that these not be made permanent parts of the PAM to avoid
possible conflicts with later corrections.

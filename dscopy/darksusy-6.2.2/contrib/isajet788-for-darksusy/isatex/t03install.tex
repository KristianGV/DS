\newpage
\section{Installation\label{INSTALL}}

      Beginning with Version 7.87, ISAJET is distributed as a single
Unix tar file. Unpacking this in an empty directory gives
\begin{verbatim}
Makefile  isadecay.dat  isainc/  isajet/   isaplt/  isared/  isatex/
runjet.f  sample.inrge  ssrun.f  sugrun.f
\end{verbatim}

Directory \verb|isainc/| contains all the include files, primarily
COMMON blocks. Directory \verb|isajet/| contains most of the Fortran
source code including ISASUSY, which was previously separate. Directory
\verb|isaplt/| contains the (obsolete) HBOOK-based simple analysis
framework. Directory \verb|isared/| contains the machine-generated code
that calculates the relic density of dark matter. Directory
\verb|isatex/| contains the \LaTeX{} source for the documentation. There
are also Fortran files for the main programs, two data files, and a Unix
Makefile.

The C Preprocessor (\verb|cpp|) directive \verb|#include| is used to
include COMMON blocks from \verb|isainc/| into the Fortran source code,
so \verb|isainc/| must be in the include path. A \LaTeX{} macro handles
includes for the documentation. Conditional code is mostly handled using
the \verb|#ifdef|\dots\verb|#endif| and \verb|#define| directives. In a
few places slightly more complicated conditional expressions are needed.
By convention, all \verb|#define| flags are upper case and end in
\verb|_X|.

All Fortran source files include \verb|PILOT.inc| at the beginning. This
contains \verb|cpp| directives that define the default:
\begin{verbatim}
#define DOUBLE_X
#define STDIO_X
#define MOVEFTN_X
#define RANLUX_X
#define NOCERN_X
#define ISATOOLS_X
#define ISAFLAVR_X
#undef  PRTEESIG_X
\end{verbatim}
where, for example, \verb|DOUBLE_X| means to use DOUBLE PRECISION where
needed. It also defines a number of switches to select options for
various operating systems and compilers. The \verb|LINUX_X| switch
selects additional features for Linux with the \verb|gfortran| compiler:
\begin{verbatim}
#ifdef  LINUX_X
#define IMPNONE_X
#define IDATE_X
#define ETIME_X
#endif
\end{verbatim}
Thus adding \verb|-DLINUX_X| as a compiler option selects the defaults
and enables the date/time features. \verb|MACOS_X| is also provided;
currently it is identical. Several obsolete versions are also defined,
even the original one for the CDC 7600. There are also additional, more
specialized \verb|cpp| flags. The \verb|cpp| directives can be expanded 
running stand-alone \verb|cpp|, e.g.,
\begin{verbatim}
cpp -I../isainc -DCDC_X aldata.f | grep -v # > aldata.for
\end{verbatim}
But do not expect old versions to be usable.

To compile ISAJET, first edit the top-level Makefile and select (or
modify) the compiler \verb|FC|, the compiler options \verb|FFLAGS|, and
the loader options \verb|LFLAGS| and \verb|LIBS|. Then run \verb|make|
in the top level directory, which will also run it in the
subdirectories. That will produce two libraries: \verb|libisajet.a| from
the \verb|isajet/| directory files and \verb|libisared.a| from the
relic density code in \verb|isared/|. (Both libraries are produced even
if \verb|NOISATOOLS_X| is defined but will not be used.) It will produce
the event generator executable \verb|isajet.x|; sample jobs for it are
given in the following sections. It will also produce executables
\verb|isasusy.x| and \verb|isasugra.x| for the SUSY model calculations.

The main programs supplied should work with any modern Unix-based compiler.
Other main programs for (very) old systems are discussed in
Section~\ref{MAIN}.

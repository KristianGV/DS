\newpage
\section{Input\label{INPUT}}

\subsection{Input Format}

      ISAJET is controlled by commands read from the specified input
file by subroutine READIN. (In the interactive version, this file is
first created by subroutine DIALOG.) Syntax errors will generate a
message and stop execution. Based on these commands, subroutine LOGIC
will setup limits for all variables and check for inconsistencies.
Several runs with different parameters can be combined into one job.
The required input format is:
\begin{verbatim}
Title
Ecm,Nevent,Nprint,Njump/
Reaction
(Optional parameters)
END
(Optional additional runs)
STOP
\end{verbatim}
with all lines starting in column 1 and typed in {\it upper} case. These
lines are explained below.

      Title line: Up to 80 characters long. If the first four letters
are STOP, control is returned to main program. If the first four letters
are SAME, the parameters from previous run are used excepting those
which are explicitly changed.

      Ecm line: This line must always be given even if the title is
SAME. It must give the center of mass energy (Ecm) and the number of
events (Nevent) to be generated. One may also specify the number of
events to be printed (Nprint) and the increment (Njump) for printing.
The first event is always printed if Nprint $>$ 0. For example:
\begin{verbatim}
800.,1000,10,100/
\end{verbatim}
generates 1000 events at $E_{\rm cm} = 800\,\GeV$ and prints 10
events. The events printed are: 1,100,200,\dots. Note that an event
typically takes several pages of output. This line is read with a list
directed format (READ*).

     After Nprint events have been printed, a single line containing the
run number, the event number, and the random number seed is printed
every Njump events (if Njump is nonzero). This seed can be used to start
a new job with the given event if in the new run NSIGMA is set equal to
zero:
\begin{verbatim}
SEED
value/
NSIGMA
0/
\end{verbatim}
In general the same events will only be generated on the same type of
computer.

      Reaction line: This line must be given unless title is SAME, when
it must be omitted. It selects the type of events to be generated. The
present version can generate TWOJET, E+E-, DRELLYAN, MINBIAS, WPAIR,
SUPERSYM, HIGGS, PHOTON, TCOLOR, or WHIGGS events. This line is read
with an A8 format.

\subsection{Optional Parameters}

      Each optional parameter requires two lines.
The first line is a keyword specifying the parameter and the second
line gives the values for the parameter. The parameters can be given in
any order. Numerical values are read with a list directed format
(READ*), jet and particle types are read with a character format and
must be enclosed in quotes, and logical flags with an L1 format. All
momenta are in GeV and all angles are in radians.

      The parameters can be classified in several groups:
\begin{center}
\begin{tabular}{lllll}
\hline\hline
Jet Limits: & W/H Limits: & Decays:     & Constants:  & Other: \\
\hline
JETTYPE1    & HTYPE       & FORCE       & AMSB        & BEAMS \\
JETTYPE2    & PHIW        & FORCE1      & AMSB2       & EPOL \\
JETTYPE3    & QMH         & NOB         & CUTJET      & EEBEAM \\
MIJLIM      & QMW         & NODECAY     & CUTOFF      & EEBREM \\
MTOT        & QTW         & NOETA       & EXTRAD      & GAMGAM \\
P           & THW         & NOEVOLVE    & GRAGMENT    & NPOMERON \\
PHI         & WTYPE       & NOFRGMNT    & GAUGINO     & NSIGMA \\
PT          & XW          & NOGRAV      & GNMIRAGE    & NTRIES \\
TH          & YW          & NOPI0       & GMSB        & PDFLIB \\
X           &             & NOTAU       & GMSB2       & SEED \\
Y           &             &             & HCAMSB      & STRUC \\
WMODE1      &             &             & HMASS       & WFUDGE \\
WMODE2      &             &             & HMASSES     & WMMODE \\
            &             &             & LAMBDA      & WPMODE \\
            &             &             & MGVTNO      & Z0MODE \\
            &             &             & MMAMSB      & WRTLHE \\
            &             &             & MSSMA       &  \\
            &             &             & MSSMB       & \\
            &             &             & MSSMC       & \\
            &             &             & MSSMD       & \\
            &             &             & MSSME       & \\
            &             &             & NUSUG1      & \\
            &             &             & NUHM        & \\
            &             &             & NUHMDT      & \\
            &             &             & NUSUG2      & \\
            &             &             & NUSUG3      & \\
            &             &             & NUSUG4      & \\
            &             &             & NUSUG5      & \\
            &             &             & SCLFAC      & \\
            &             &             & SIGQT       & \\
            &             &             & SIN2W       & \\
            &             &             & SLEPTON     & \\
            &             &             & SQUARK      & \\
            &             &             & SSBCSC      & \\
            &             &             & SUGRA       & \\
            &             &             & SUGRHN      & \\
            &             &             & TCMASS      & \\
            &             &             & TMASS       & \\
            &             &             & WMASS       & \\
\hline\hline
\end{tabular}
\end{center}

      It may be helpful to know that the TWOJET, WPAIR, PHOTON,
SUPERSYM, and WHIGGS processes use the same controlling routines and
so share many of the same variables.  In particular, PT limits should
normally be set for these processes, and JETTYPE1 and JETTYPE2 are
used to select the reactions. Similarly, the DRELLYAN, HIGGS, and
TCOLOR processes use the same controlling routines since they all
generate s-channel resonances. The mass limits for these processes are
set by QMW.  Normally the QMW limits will surround the $W^\pm$, $Z^0$,
or Higgs mass, but this is not required.  (QMH acts like QMW for the
Higgs process.) For historical reasons, JETTYPE1 and JETTYPE2 are used
to select the W decay modes in DRELLYAN, while WMODE1 and WMODE2 select
the W decay modes for WPAIR, HIGGS, and WHIGGS. Also, QTW can be used
to generate DRELLYAN events with non-zero transverse momentum, whereas
HIGGS automatically fixes QTW to be zero. (Of course, non-zero
transverse momentum will be generated by gluon radiation.)

      For example the lines
\begin{verbatim}
P
40.,50.,10.,100./
\end{verbatim}
would set limits for the momentum of jet 1 between 40 and 50 GeV, and
for jet 2 between 10 and 100 GeV. As another example the lines
\begin{verbatim}
WTYPE
'W+'/
\end{verbatim}
would specify that for DRELLYAN events only W+ events will be generated.
If for a kinematic variable only the lower limit is specified then that
parameter is fixed to the given value. Thus the lines
\begin{verbatim}
P
40.,,10./
\end{verbatim}
will fix the momentum for jet 1 to be 40 GeV and for jet 2 to be 10
GeV. If only the upper limit is specified then the default value is used
for the lower limit. Jet 1 or jet 2 parameters for DRELLYAN events refer
to the W decay products and cannot be fixed. If QTW is fixed to 0, then
standard Drell-Yan events are generated.

      A complete list of keywords and their default values follows.

\newpage
\begin{center}
\begin{tabular}{lll}
\hline\hline
Keyword                &                   & Explanation                    \\
Values                 & Default values    &                                \\
\hline
AMSB                   &                   & Anomaly-mediated SUSY breaking \\
$m_0$,$m_{3/2}$,$\tan\beta$,$\sgn\mu$ & none & scalar mass, gravitino mass, \\
                       &                   & VEV ratio, sign                \\
                       &                   &                                \\
AMSB2                  &                   & Non-minimal AMSB               \\
$c_Q,c_D,c_U,c_L,c_E,c_{H_d},c_{H_u}$ & 1  & multiplies $m_0^2$ cont'n. to  \\
                       &                   & soft SUSY masses               \\
                       &                   &                                \\
BEAMS                  &                   & Initial beams. Allowed are     \\
type$_1$,type$_2$      & 'P','P'           & 'P','AP','N','AN'.             \\
                       &                   &                                \\
CUTJET                 &                   & Cutoff mass for QCD jet        \\
$\mu_c$                & 6.                & evolution.                     \\
                       &                   &                                \\
CUTOFF                 &                   & Cutoff $qt^2=\mu^2Q^\nu$ for   \\
$\mu^2$, $\nu$         & .200,1.0          & DRELLYAN events.               \\
                       &                   &                                \\
EEBEAM                 &                   & impose brem/beamstrahlung      \\
$\sqrt{\hat{s}}_{min}$, $\sqrt{\hat{s}}_{max}$, $\Upsilon$, $\sigma_z$ &
none & min and max subprocess energy, \\
                       & & beamstrahlung parameter $\Upsilon$ \\
                       & & longitudinal beam size $\sigma_z$ in mm \\
                       &                   &                                \\
EEBREM                 &                 & impose bremsstrahlung for $e^+e^-$ \\
$\sqrt{\hat{s}}_{min}$, $\sqrt{\hat{s}}_{max}$ & none & min and max subprocess 
energy \\
                       &                   &                                \\
EPOL                   &              & Polarization of $e^-$ ($e^+$) beam, \\
$P_L(e^-),P_L(e^+)$    & 0,0               & $P_L(e)=(n_L-n_R)/(n_L-n_R)$,  \\
                       &                   & so that $-1 \le P_L \le 1$     \\
                       &                   &                                \\
EXTRAD                 &                   & Parameters for EXTRADIM process\\
$\delta$,$M_D$,UVCUT   & None              & UVCUT is logical flag          \\
                       &                   &                                \\
FORCE                  &                   & Force decay of particles,      \\
$i,i_1,...,i_5$/       & None              & $\pm i \to \pm(i1+...+i5)$.    \\
                       &                   & Can call 20 times.             \\
                       &                   & See note for $i$ = quark.      \\
                       &                   &                                \\
FORCE1                 &                   & Force decay $i \to i1+...+i5$. \\
$i,i_1,...,i_5$/       & None              & Can call 40 times.             \\
                       &                   & See note for $i$ = quark.      \\
\hline\hline
\end{tabular}
\end{center}

\newpage
\begin{center}
\begin{tabular}{lll}
\hline\hline
FRAGMENT               &                   & Fragmentation parameters.      \\
$P_{ud}$,\dots         & .4,\dots          & See also SIGQT, etc.           \\
                       &                   &                                \\
GAMGAM                 &             & Activate $\gamma\gamma\to f\bar{f}$  \\
TRUE or FALSE          & FALSE             & in $e^+e^-$ collisions         \\
                       &                   &                                \\
GAUGINO                &                   & Masses for $\tilde g$, 
$\tilde\gamma$,                                                             \\
$m_1$,$m_2$,$m_3,m_4$  & 50,0,100,100      & $\tilde W^+$, and $\tilde Z^0$ \\
                       &                   &                                \\
GNMIRAGE               &                   & Gen. mirage mediation           \\
$\alpha$,$m_{3/2}$,$c_m$,$c_{m3}$,$a_3$,$\tan\beta$,$\sgn\mu$,$c_{H_u}$,$c_{H_d}$ & none & mirage-mixing par., gravitino mass, etc.\\
                       &                   &                                \\
GMSB                   &                   & GMSB messenger SUSY breaking,  \\
$\Lambda_m$,$M_m$,$N_5$ & none             & mass, number of $5+\bar5$, VEV \\
$\tan\beta$,$\sgn\mu$,$C_{\rm gr}$ &       & ratio, sign, gravitino scale   \\
                       &                   &                                \\
GMSB2                  &                   & non-minimal GMSB parameters    \\
$\slashchar{R}$,$\delta M_{H_d}^2$,$\delta M_{H_u}^2$,$D_Y(M)$ & 1,0,0,0 & 
gaugino mass multiplier \\
$N_{5_1}$,$N_{5_2}$,$N_{5_3}$ & $N_5$     & Higgs mass shifts, D-term mass$^2$\\
                       &                   & indep. gauge group messengers  \\
                       &                   &                                \\
HCAMSB                 &                   & Hypercharged AMSB              \\
$\alpha$,$m_{3/2}$,$\tan\beta$,$\sgn\mu$ & none & HC-mixing par., gravitino mass, \\
                       &                   & VEV ratio, sign                \\
                       &                   &                                \\
HMASS                  & 0                 & Mass for standard Higgs.       \\
$m$                    &                   &                                \\
                       &                   &                                \\
HMASSES                &                   & Higgs meson masses for         \\
$m_1$,\dots,$m_9$      & 0,...,0           & charges 0,0,0,0,0,1,1,2,2.     \\
                       &                   &                                \\
HTYPE                  &                   & One MSSM Higgs type ('HL0',    \\
'HL0'/ or...           & none              & 'HH0', or 'HA0')               \\
                       &                   &                                \\
JETTYPE1               &                   & )Select types for jets:        \\
'GL','UP',...          & 'ALL'             & )'ALL'; 'GL'; 'QUARKS'='UP',   \\
                       &                   & )'UB','DN','DB','ST','SB',     \\
JETTYPE2               &                   & )'CH','CB','BT','BB','TP',     \\
'GL','UP',...          & 'ALL'             & )'TB','X','XB','Y','YB';       \\
                       &                   & )'LEPTONS'='E-','E+','MU-',    \\
JETTYPE3               &                   & )'MU+','TAU-','TAU+'; 'NUS';   \\
'GL','UP',...          & 'ALL'             & )'GM','W+','W-','Z0'           \\
                       &                   & ) See note for SUSY types.     \\
\hline\hline
\end{tabular}
\end{center}

\newpage
\begin{center}
\begin{tabular}{lll}
\hline\hline
LAMBDA                 &                   & QCD scale                      \\
$\Lambda$              & .2                &                                \\
                       &                   &                                \\
MGVTNO                 &                   & Gravitino mass -- ignored for  \\
$M_{\rm gravitino}$    & $10^{20}$~GeV     & GMSB model                     \\
                       &                   &                                \\
MIJLIM                 &                   & Multimet mass limits           \\
$i$,$j$,$M_{\rm min}$,$M_{\rm max}$ & 0,0,$1\,\GeV$,$1\,\GeV$ &             \\
                       &                   &                                \\
MMAMSB                 &                   & Mixed modulus-AMSB model       \\
$\alpha$,$m_{3/2}$,$\tan\beta$,$\sgn\mu ,$ & none & mixing par., 
grav. mass, $\tan\beta$, $sgn(\mu )$ \\
$n_Q,n_D,n_U,n_L,n_E,n_{H_d},n_{H_u}$  & none  & modular weights            \\
$\ell_a,\ \ell_s,\ \ell_3$ & none  & moduli power in GKF                    \\
                       &                   &                                \\
MSSMA                  &                   & MSSM parameters --             \\
$m(\tilde g)$,$\mu$,   & Required          & Gluino mass, $\mu$, $A$ mass,  \\
$m(A)$,$\tan\beta$     &                   & $\tan\beta$                    \\
                       &                   &                                \\
MSSMB                  &                   & MSSM 1st generation --         \\
$m(q_1)$,$m(d_r)$,$m(u_r)$, & Required     & Left and right soft squark and \\
$m(l_1)$,$m(e_r)$      &                   & slepton masses                 \\
                       &                   &                                \\
MSSMC                  &                   & MSSM 3rd generation --         \\
$m(q_3)$,$m(b_r)$,$m(t_r)$,  & Required    & Soft squark masses, slepton    \\
$m(l_3)$,$m(\tau_r)$,  &                   & masses, and squark and slepton \\
$A_t$,$A_b$,$A_\tau$   &                   & mixings                        \\
                       &                   &                                \\
MSSMD                  &                   & MSSM 2nd generation --         \\
$m(q_2)$,$m(s_r)$,$m(c_r)$,  & from MSSMB  & Left and right soft squark and \\
$m(l_2)$,$m(mu_r)$     &                   & slepton masses                 \\
                       &                   &                                \\
MSSME                  &                   & MSSM gaugino masses --         \\
$M_1$,$M_2$            & MSSMA + GUT       & Default is to scale from gluino\\
                       &                   &                                \\
MTOT                   &                   & Mass range for multiparton     \\
$M_{\rm min}$,$M_{\rm max}$ & None         & processes                      \\
                       &                   &                                \\
NOB                    &                   & Suppress B decays to use       \\
TRUE or FALSE          & FALSE             & external package.              \\
                       &                   &                                \\
NODECAY                &                   & Suppress all decays.           \\
TRUE or FALSE          & FALSE             &                                \\
\hline\hline
\end{tabular}
\end{center}

\newpage
\begin{center}
\begin{tabular}{lll}
\hline\hline
NOETA                  &                   & Suppress eta decays.           \\
TRUE or FALSE          & FALSE             &                                \\
                       &                   &                                \\
NOEVOLVE               &                   & Suppress QCD evolution and     \\
TRUE or FALSE          & FALSE             & hadronization.                 \\
                       &                   &                                \\
NOGRAV                 &                   & Suppress gravitino decays in   \\
TRUE or FALSE          & FALSE             & GMSB model                     \\
                       &                   &                                \\
NOHADRON               &                   & Suppress hadronization of      \\
TRUE or FALSE          & FALSE             & jets and beam jets.            \\
                       &                   &                                \\
NONUNU                 &                   & Suppress $Z^0$ neutrino decays.\\
TRUE or FALSE          & FALSE             &                                \\
                       &                   &                                \\
NOPI0                  &                   &Suppress $\pi^0$ decays.        \\
TRUE or FALSE          & FALSE             &                                \\
                       &                   &                                \\
NOTAU                  &                   & Suppress tau decays to use     \\
TRUE or FALSE          & FALSE             & external package.              \\
                       &                   &                                \\
NPOMERON               &                   & Allow $n_1<n<n_2$ cut pomerons.\\
$n_1$,$n_2$            & 1,20              & Controls beam jet mult.        \\
                       &                   &                                \\
NSIGMA                 &                   & Generate n unevolved events    \\
$n$                    & 20                & for SIGF calculation.          \\
                       &                   &                                \\
NTRIES                 &                   & Stop if after n tries          \\
$n$                    & 1000              & cannot find a good event.      \\
                       &                   &                                \\
NUHM                   &                   & Optional non-universal SUGRA   \\
$\mu (M_{weak})$, $m_A(M_{weak})$ & none   & Higgs masses in terms of       \\
                       &                   & $\mu$, $m_A$                   \\
                       &                   &                                \\
NUHMDT                 &                   & Optional non-universal soft terms \\
$\mu (M_{weak})$, $m_A(M_{weak})$ & none      & with D-term splitting; input  \\
                       &                   & $\mu$, $m_A$                   \\
\hline\hline
\end{tabular}
\end{center}

\newpage
\begin{center}
\begin{tabular}{lll}
\hline\hline
NUSUG1                 &                   & Optional non-universal SUGRA   \\
$M_1$,$M_2$,$M_3$      & none              & gaugino masses                 \\
                       &                   &                                \\
NUSUG2                 &                   & Optional non-universal SUGRA   \\
$A_t$,$A_b$,$A_\tau$   & none              & $A$ terms                      \\
                       &                   &                                \\
NUSUG3                 &                   & Optional non-universal SUGRA   \\
$M_{H_d}$,$M_{H_u}$    & none              & Higgs masses                   \\
                       &                   &                                \\
NUSUG4                 &                   & Optional non-universal SUGRA   \\
$M_{u_L}$,$M_{d_R}$,$M_{u_R}$, & none      & 1st/2nd generation masses      \\
$M_{e_L}$,$M_{e_R}$    &                   &                                \\
                       &                   &                                \\
NUSUG5                 &                   & Optional non-universal SUGRA   \\
$M_{t_L}$,$M_{b_R}$,$M_{t_R}$, & none      & 3rd generation masses          \\
$M_{\tau_L}$,$M_{\tau_R}$ &                &                                \\
                       &                   &                                \\
P                      &                   & Momentum limits for jets.      \\
$p_{\rm min}(1)$,\dots,$p_{\rm max}(3)$ & 
1.,$0.5E_{\rm cm}$                         &                                \\

PDFLIB                 &                   & CERN PDFLIB parton distribution\\
'name$_1$',val$_1$,\dots & None            & parameters. See PDFLIB manual. \\
                       &                   &                                \\
PHI                    &                   & Phi limits for jets.           \\
$\phi_{\rm min}(1)$,\dots,$\phi_{\rm max}(3)$ & 0,$2\pi$ &                  \\
                       &                   &                                \\
PHIW                   &                   & Phi limits for W.              \\
$\phi_{\rm min}$,$\phi_{\rm max}$ & 
0,$2\pi$                                   &                                \\
                       &                   &                                \\
PT or PPERP            &                   & $p_t$ limits for jets.         \\
$p_{t,{\rm min}}(1)$,\dots,$p_{t,{\rm max}}(3)$  & 
$.05E_{\rm cm}$,$.2E_{\rm cm}$             & Default for TWOJET only.       \\
                       &                   &                                \\
QMH                    &                   & Mass limits for Higgs.         \\
$q_{\rm min}$,$q_{\rm max}$ & 
$.05E_{\rm cm}$,$.2E_{\rm cm}$             & Equivalent to QMW.             \\
                       &                   &                                \\
QMW                    &                   & Mass limits for $W$.           \\
$q_{\rm min}$,$q_{\rm max}$ & 
$.05E_{\rm cm}$,$.2E_{\rm cm}$             &                                \\
                       &                   &                                \\
QTW                    &                   & $q_t$ limits for $W$. Fix 
$q_t=0$                                                                     \\
$q_{t,{\rm min}}$,$q_{t,{\rm max}}$ & 
.1,$.025E_{\rm cm}$                        & for standard Drell-Yan.        \\
                       &                   &                                \\
SCLFAC                 &                   & Scale factor multiplier in $\alpha_s$   \\
$Q\rightarrow nQ$      & $n=1.0$           & to vary QCD total cross section.   \\
\hline\hline
\end{tabular}
\end{center}

\newpage
\begin{center}
\begin{tabular}{lll}
\hline\hline
SEED                   &                   & Seed $<$281474976710656.D0 for \\
real/integer           & 0                 & RANF or $<$$2^{31}$ for RANLUX.\\
                       &                   &                                \\
SIGQT                  &                   & Internal $k_t$ parameter for   \\
$\sigma$               & .35               & jet fragmentation.             \\
                       &                   &                                \\
SIN2W                  &                   & Weinberg angle. See WMASS.     \\
$\sin^2(\theta_W)$     & .232              &                                \\
                       &                   &                                \\
SLEPTON                &                   & Masses for $\tilde \nu_e$,
$\tilde e$, $\tilde\nu_\mu$, $\tilde\mu$, $\tilde\nu_\tau$, $\tilde\tau$    \\
$m_1$,\dots,$m_6$      & 100,\dots,101.8   &                                \\
                       &                   &                                \\
SQUARK                 &                   & Masses for $\tilde u$,
$\tilde d$, $\tilde s$, $\tilde c$, $\tilde b$, $\tilde t$                  \\
$m_1$,\dots,$m_6$      & 100.3,...,240.    &                                \\
                       &                   &                                \\
SSBCSC                 &                   & Alternate mass scale for RGE   \\
$M$                    & $M_{GUT}$         & boundary conditions.           \\
                       &                   &                                \\
STRUC                  &                   & Structure functions. CTEQ5L,   \\
name                   & 'CTEQ5L'          & CTEQ3L, CTEQ2L, EHLQ, OR DO    \\
                       &                   &                                \\
SUGRA                  &                   & Minimal supergravity parameters\\
$m_0$,$m_{1/2}$,$A_0$, & none              & scalar M, gaugino M, trilinear \\
$\tan\beta$,$\sgn\mu$  &                   & breaking term, vev ratio, +-1  \\
                       &                   &                                \\
SUGRHN                 &                   & SUGRA see-saw $\nu$-effect     \\
$m_{\nu_\tau}$,$M_N$,$A_n$,$m_{\tilde\nu_R}$ & $0,1E20,0,0$ & nu-mass, 
int. scale, \\
                       &                   & GUT scale nu SSB terms         \\
                       &                   &                                \\
TH or THETA            &                   & Theta limits for jets. Do not  \\
$\theta_{\rm min}(1)$,\dots,$\theta_{\rm max}(3)$ & 0,$\pi$ & also set Y.   \\
                       &                   &                                \\
THW                    &                   & Theta limits for W. Do not     \\
$\theta_{\rm min}$,$\theta_{\rm max}$ & 0,$\pi$ & also set YW.              \\
                       &                   &                                \\
TCMASS                 &                   & Technicolor mass and width.    \\
$m$,$\Gamma$           & 1000,100          &                                \\
                       &                   &                                \\
TMASS                  &                   & t, y, and x quark masses.      \\
$m_t$,$m_y$,$m_x$      & 180.,-1.,-1.      &                                \\
                       &                   &                                \\
WFUDGE                 &                   & Fudge factor for DRELLYAN      \\
factor                 & 1.85              & evolution scale.               \\
\hline\hline
\end{tabular}
\end{center}

\newpage
\begin{center}
\begin{tabular}{lll}
\hline\hline
WMASS                  &                   & W and Z masses. See SIN2W.     \\
$M_W$,$M_Z$            & 80.2, 91.19       &                                \\
                       &                   &                                \\
WMMODE                 &                   & Decay modes for $W^-$ in parton\\
'UP',\dots,'TAU+'      & 'ALL'             & cascade. See JETTYPE.          \\
                       &                   &                                \\
WMODE1                 &                   & )                              \\
'UP','UB',\dots        & 'ALL'             & )Decay modes for WPAIR.        \\
                       &                   & )Same code for quarks and      \\
WMODE2                 &                   & )leptons as JETTYPE.           \\
'UP','UB',\dots        & 'ALL'             & )                              \\
                       &                   &                                \\
WPMODE                 &                   & Decay modes for $W^+$ in parton\\
'UP',\dots,'TAU+'      & 'ALL'             & cascade. See JETTYPE.          \\
                       &                   &                                \\
WRTLHE                 &                   &                                \\
TRUE or FALSE          & FALSE             & Write events according to      \\
                       &                   & Les Houches accord             \\
                       &                   &                                \\
WTYPE                  &                   & Select W type: W+,W-,GM,Z0.    \\
type$_1$,type$_2$      & 'GM','Z0'         & Do not mix W+,W- and GM,Z0.    \\
                       &                   &                                \\
X                      &                   & Feynman x limits for jets.     \\
$x_{\rm min}(1)$,\dots,$x_{\rm max}(3)$ & 
$-1$,1                                     &                                \\
                       &                   &                                \\
XGEN                   &                   & Jet fragmentation, Peterson    \\
a(1),\dots,a(8)        & .96,3,0,.8,.5,... & with $\epsilon=a(n)/m^2$, 
$n=4$-8.                                                                    \\
                       &                   &                                \\
XGENSS                 &                   & Fragmentation of GLSS, UPSS,   \\
a(1),\dots,a(7)        & .5,.5,...         & etc. with $\epsilon=a(n)/m**2$ \\
                       &                   &                                \\
                       &                   &                                \\
XW                     &                   & Feynman x limits for W.        \\
$x_{\rm min}$,$x_{\rm max}$ & $-1$,1       &                                \\
                       &                   &                                \\
Y                      &                   & Y limits for each jet.         \\
$y_{\rm min}(1)$,\dots,$y_{\rm max}(3)$ & from PT & Do not also set TH.     \\
                       &                   &                                \\
YW                     &                   & Y limits for W.                \\
$y_{\rm min}$,$y_{\rm max}$ & from QTW,QMW & Do not set both YW and THW.    \\
                       &                   &                                \\
Z0MODE                 &                   & Decay modes for $Z^0$ in parton\\
'UP',\dots,'TAU+'      & 'ALL'             & cascade. See JETTYPE.          \\
\hline\hline
\end{tabular}
\end{center}

\newpage
\subsection{Kinematic and Parton-type Parameters}

      While the TWOJET PT limits and the DRELLYAN QMW limits are
formally optional parameters, they are set by default to be fractions of
$\sqrt{s}$. Thus, for example, the parameter file
\begin{verbatim}
DEFAULT TWOJET JOB
14000,100,1,100/
TWOJET
END
STOP
\end{verbatim}
will execute, but it will generate jets between 5\% and 20\% of
$\sqrt{s}$, which is probably not what is wanted. Similarly, the
parameter file
\begin{verbatim}
DEFAULT DRELLYAN JOB
14000,100,1,100/
DRELLYAN
END
STOP
\end{verbatim}
will generate $\gamma + Z$ events with masses between 5\% and 20\% of
$\sqrt{s}$, not masses around the $Z$ mass, and transverse momenta
between $1\,{\rm GeV}$ and 2.5\% of $\sqrt{s}$.

      Normally the user should set PT limits for TWOJET, PHOTON, WPAIR,
SUPERSYM, and WHIGGS events and QMW and QTW limits for DRELLYAN,
HIGGS, and TCOLOR events. If these limits are not set, they will be
selected as fractions of $E_{\rm cm}$. This can give nonsense. For
TWOJET the $p_t$ range should usually be less than about a factor of
two except for $b$ and $t$ jets at low $p_t$ to produce uniform
statistics. For $W^+$, $W^-$, or $Z^0$ events or for Higgs events the
QMW (QMH) range should usually include the mass. But one can select
different limits to study, e.g., virtual $W$ production or the effect
of a lighter or heavier Higgs on WW scattering. If only $t$ decays are
selected, then the lower QMW limit must be above the $t$ threshold.
For standard Drell-Yan events QTW should be fixed to zero,
\begin{verbatim}
QTW
0/
\end{verbatim}
Transverse momenta will then be generated by initial state gluon
radiation. A range of QTW can also be given. For SUPERSYM either the
masses and decay modes should be specified, or the MSSM, SUGRA, GMSB, or
AMSB parameters should be given. For fourth generation quarks it is
necessary to specify the quark masses.

      Note that if the limits given cover too large a kinematic range,
the program can become very inefficient, since it makes a fit to the
cross section over the specified range. NTRIES has to be increased if
narrow limits are set for X, XW or for jet 1 and jet 2 parameters in
DRELLYAN events. For larger ranges several runs can be combined together
using the integrated cross section per event SIGF/NEVENT as the weight.
This cross section is calculated for each run by Monte Carlo integration
over the specified kinematic limits and is printed at the end of the
run. It is corrected for JETTYPEi, WTYPE, and WMODEi selections; it
cannot be corrected for branching ratios of forced decays or for WPMODE,
WMMODE, or Z0MODE selections, since these can affect an arbitrary number
of particles.

      To generate events over a large range, it is much more efficient
to combine several runs. This is facilitated by using the special job
title SAME as described above. Note that SAME cannot be used to combine
standard DRELLYAN events (QTW fixed equal to 0) and DRELLYAN events with
nonzero QTW.

      The cross sections for multiparton final states in general have
infrared and collinear singularities. To obtain sensible results, it
is in general essential to set limits both on the $p_T$ of each final
parton using PT and on the mass of each pair of partons using MIJLIM.
The default lower limits are all $1\,{\rm GeV}$. Using these default
limits without thought will likely give absurd results.

      For TWOJET, DRELLYAN, and most other processes, the JETTYPEi and
WTYPEi keywords should be used to select the subprocesses to be
included. For $e^+ e^- \to W^+ W^-$, $Z^0 Z^0$, use FORCE and FORCE1
instead of WMODEi to select the $W$ decay modes. Note that these {\it
do not} change the calculated cross section. (In the E+E- process, the
$W$ and $Z$ decays are currently treated as particle decays, whereas in
the WPAIR and HIGGS processes they are treated as $2 \to 4$ parton
processes.)

      For HIGGS with $W^+W^-$ or $Z^0Z^0$ decays allowed it is
generally necessary to set PT limits for the W's, e.g.
\begin{verbatim}
PT
50,20000,50,20000/
\end{verbatim} 
If this is not done, then the default lower limit of 1 GeV is used,
and the $t$-channel exchanges will dominate, as they should in the
effective $W$ approximation. Depending on the other parameters, the
program may fail to generate an event in NTRIES tries.

\subsection{SUSY Parameters}

      SUPERSYM (SUSY) by default generates just gluinos and squarks in
pairs. There are no default masses or decay modes. Masses can be set
using GAUGINO, SQUARK, SLEPTON, and HMASSES. Decay modes can be
specified with FORCE or by modifying the decay table. Left and right
squarks are distinguished but assumed to be degenerate, except for
stops. Since version 7.11, types must be selected with JETTYPEi using
the supersymmetric names, e.g.
\begin{verbatim}
JETTYPE1
'GLSS','UPSSL','UPSSR'/
\end{verbatim}
Use of the corresponding standard model names, e.g.
\begin{verbatim}
JETTYPE1
'GL','UP'/
\end{verbatim}
and generation of pure photinos, winos, and zinos are no longer
supported.

      If MSSMA, MSSMB and MSSMC are given, then the specified parameters
are used to calculate all the masses and decay modes with the ISASUSY
package assuming the minimal supersymmetric extension of the standard
model (MSSM). There are no default values, so you must specify values
for each MSSMi, i=A-C. MSSMD can optionally be used to set the second
generation squark and slepton parameters; if it is omitted, then the
first generation ones are used. MSSME can optionally be used to set the
U(1) and SU(2) gaugino masses; if it is omitted, then the grand
unification values are used. The parameters and the use of the MSSM is
preserved if the title is SAME. FORCE can be used to override the
calculated branching ratios. 

      The MSSM option also generates charginos and neutralinos with
cross sections based on the MSSM mixing angles in addition to squarks
and sleptons. These can be selected with JETTYPEi; the complete list of
supersymmetric options is:
\begin{verbatim}
'GLSS',
'UPSSL','UBSSL','DNSSL','DBSSL','STSSL','SBSSL','CHSSL','CBSSL',
'BTSS1','BBSS1','TPSS1','TBSS1',
'UPSSR','UBSSR','DNSSR','DBSSR','STSSR','SBSSR','CHSSR','CBSSR',
'BTSS2','BBSS2','TPSS2','TBSS2',
'W1SS+','W1SS-','W2SS+','W2SS-','Z1SS','Z2SS','Z3SS','Z4SS',
'NUEL','ANUEL','EL-','EL+','NUML','ANUML',MUL-','MUL+','NUTL',
'ANUTL','TAU1-','TAU1+','ER-','ER+','MUR-','MUR+','TAU2-','TAU2+',
'Z0','HL0','HH0','HA0','H+','H-',
'SQUARKS','GAUGINOS','SLEPTONS','ALL'.
\end{verbatim}
Note that mixing between $L$ and $R$ stop states results in 1 (light)
and 2 (heavy) stop, sbottom and stau eigenstates, which depend on the
input parameters of left- and right- scalar masses, plus $A$ terms,
$\mu$ and $\tan\beta$. The last four JETTYPE's generate respectively
all allowed combinations of squarks and antisquarks, all combinations
of charginos and neutralinos, all combinations of sleptons and
sneutrinos, and all SUSY particles. 

      For SUSY Higgs pair production or associated production in E+E-,
select the appropriate JETTYPE's, e.g.
\begin{verbatim}
JETTYPE1
'Z0'/
JETTYPE2
'HL0'/
\end{verbatim}

As usual, this gives only half the cross section. For single production
of neutral SUSY Higgs in $pp$ and $\bar pp$ reactions, use the HIGGS
process together with the MSSMi, SUGRA, GMSB, or AMSB keywords. You must
specify one and only one Higgs type using
\begin{verbatim}
HTYPE
'HL0' or 'HH0' or 'HA0'/     <<<<< One only!
\end{verbatim}
If no QMH range is given, one is calculated using $M \pm 5 \Gamma$ for
the selected Higgs. Decays into quarks, leptons, gauge bosons, lighter
Higgs bosons, and SUSY particles are generated using the on-shell
branching ratios from ISASUSY. You can use JETTYPEi to select the
allowed Higgs modes and WMODEi to select the allowed decays of W and Z
bosons. Since heavy SUSY Higgs bosons couple weakly to W pairs, WW
fusion and WW scattering are not included. 

      SUGRA can be used instead of MSSMi to generate MSSM decays with
parameters determined from $m_0$, $m_{1/2}$, $A_0$, $\tan\beta$, and
$\sgn\mu=\pm1$ in the minimal supergravity framework. The NUSUGi
keywords can optionally be used to specify additional parameters for
non-universal SUGRA models, while SUGRHN is used to specify the
parameterf of an optional right-handed neutrino. Similarly, the GMSB
keyword is used to specify the $\Lambda$, $M_m$, $N_5$, $\tan\beta$,
$\sgn\mu=\pm1$, and $C_{\rm grav}$ parameters of the minimal Gauge
Mediated SUSY Breaking model. GMSB2 can optionally be used to specify
additional parameters of non-minimal GMSB models. The AMSB keyword is
used to specify $m_0$, $m_{3/2}$, $\tan\beta$, and $\sgn\mu$ for the
minimal Anomaly Mediated SUSY Breaking model. Note that $m_{3/2}$ is
much larger than the weak scale, typically 50~TeV.

      WHIGGS is used to generate $W$ plus neutral Higgs events. For the
Standard Model the JETTYPE is \verb|HIGGS|. If any of the SUSY models
is specified, then the appropriate SUSY Higgs type should be used,
most likely \verb|HL0|. In either case WMODEi is used to specify the
$W$ decay modes. The Higgs is treated as a particle; its decay modes
can be set using FORCE.

\subsection{Forced Decay Modes}

      The FORCE keyword requires special care. Its list must contain the
numerical particle IDENT codes, e.g.
\begin{verbatim}
FORCE
140,130,-120/
\end{verbatim}
The charge-conjugate mode is also forced for its antiparticle. Thus the
above example forces both $\bar D^0 \to K^+ \pi^-$ and $D^0 \to K^-
\pi^+$. If only a specific decay is wanted one should use the FORCE1
command; e.g.
\begin{verbatim}
FORCE1
140,130,-120/
\end{verbatim}
only forces $\bar D^0 \to K^+ \pi^-$.

      To force a heavy quark decay one must generally separately force
each hadron containing it. If the decay is into three leptons or quarks,
then the real or virtual W propagator is inserted automatically. Since
Version 7.30, top and fourth generation quarks are treated as
particles and decayed directly rather than first being made into
hadrons. Thus for example
\begin{verbatim}
FORCE1
6,-12,11,5/
\end{verbatim}
forces all top quarks to decay into an positron, neutrino and a
b-quark (which will be hadronized). For the physical top mass, the
positron and neutrino will come from a real W. Note that forcing $t
\to W^+ b$ and $W^+ \to e^+ \nu_e$ does {\it not} give the same
result; the first uses the correct $V-A$ matrix element, while the
second decays the $W$ according to phase space.

      Forced modes included in the decay table or generated by ISASUSY
will automatically be put into the correct order and will use the
correct matrix element. Modes not listed in the decay table are
allowed, but caution is advised because a wrong decay mode can cause
an infinite loop or other unexpected effects.

      FORCE (FORCE1) can be called at most 20 (40) times in any run plus
all subsequent 'SAME' runs. If it is called more than once for a given
parent, all calls are listed, and the last call is used. Note that FORCE
applies to particles only, but that for gamma, W+, W-, Z0 and
supersymmetric particles the same IDENT codes are used both as jet types
and as particles.

\subsection{Parton Distributions}

      The default parton distributions are fit CTEQ5L from the CTEQ
Collaboration using lowest order QCD. The CTEQ3L, CTEQ2L, and the old
EHLQ and Duke-Owens distributions can be selected using the STRUC
keyword.

      If PDFLIB support is enabled (see Section 4), then any of the
distributions in the PDFLIB compilation by H. Plothow-Besch can be
selected using the PDFLIB keyword and giving the proper parameters,
which are identical to those described in the PDFLIB manual and are
simply passed to the routine PDFSET. For example, to select fit 29
(CTEQ3L) by the CTEQ group, leaving all other parameters with their
default values, use
\begin{verbatim}
PDFLIB
'CTEQ',29D0/
\end{verbatim}
Note that the fit-number and the other parameters are of type DOUBLE
PRECISION (REAL on 64-bit machines). There is no internal passing of
parameters except for those which control the printing of messages.

\subsection{Multiparton Processes}

      For multiparton final states one should in general set limits
on the total mass \verb|MTOT| of the final state, on the minimum
\verb|PT| of each light parton, and on the minimum mass \verb|MIMLIM|
of each pair of light partons. Limits for \verb|PT| are set in the
ususal way. Limits for the mass $M_{ij}$ of partons $i,j$ are set using
\begin{verbatim}
MIJLIM
i,j,Mmin,Mmax
\end{verbatim}
If $i=j=0$, the limit is applied to all jet pairs. For example the
following parameter file generates \verb|ZJJ| events at the LHC with a
mimimum $p_T$ of $20\,\GeV$ and a minimum mass of $20\,\GeV$ for all
jet pairs:
\begin{verbatim}
GENERATE ZJJ with PTMIN = 20 GEV AND MMIN = 20 GEV
14000,100,1,100/
ZJJ
PT
20,7000,20,7000,20,7000/
MIJLIM
0,0,20,7000/
MTOT
100,500/
NSIGMA
200/
NTRIES
10000/
END
STOP
\end{verbatim}
The default lower limits for \verb|PT| and \verb|MIJLIM| are
$1\,\GeV$. While these limits are sufficient to make the cross
sections finite, they will in general not give physically sensible
results. Thus, {\it the user must think carefully about what limits
should be set.}

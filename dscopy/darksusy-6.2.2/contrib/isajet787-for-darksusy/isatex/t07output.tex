\newpage
\section{Output\label{OUTPUT}}

      The output tape or file contains three types of records. A
beginning record is written by a call to ISAWBG before generating a set
of events; an event record is written by a call to ISAWEV for each
event; and an end record is written for each run by a call to ISAWND.
These subroutines load the common blocks described below into a single
\begin{verbatim}
COMMON/ZEVEL/ZEVEL(1024) 
\end{verbatim}
and write it out when it is full. A subroutine RDTAPE, described in
the next section, inverts this process so that the user can analyze
the event.

      ZEVEL is written out to TAPEj by a call to BUFOUT. For the CDC
version IF = PAIRPAK is selected; BUFOUT first packs two words from
ZEVEL into one word in 
\begin{verbatim}
COMMON/ZVOUT/ZVOUT(512) 
\end{verbatim}
using subroutine PAIRPAK and then does a buffer out of ZVOUT to TAPEj.
Typically at least two records are written per event. For all other
computers IF=STDIO is selected, and ZEVEL is written out with a
standard FORTRAN unformatted write.

\subsection{Beginning Record}

      At the start of each run ISAWBG is called. It writes out the
following common blocks:
\begin{verbatim}
\end{verbatim}
\begininc{dylim.inc}
\end{verbatim}
\begin{tabular}{lcl}
QMIN,QMAX          &=& $W$ mass limits\\
QTMIN,QTMAX        &=& $W$ $q_t$ limits\\
YWMIN,YWMAX        &=& $W$ $\eta$ rapidity limits\\
XWMIN,XWMAX        &=& $W$ $x_F$ limits\\
THWMIN,THWMAX      &=& $W$ $\theta$ limits\\
PHWMIN,PHWMAX      &=& $W$ $\phi$ limits\\
\end{tabular}

\begin{verbatim}
\end{verbatim}
\begininc{idrun.inc}
\end{verbatim}
\begin{tabular}{lcl}
IDVER              &=& program version\\
IDG(1)             &=& run date (10000$\times$month+100$\times$day+year)\\
IDG(2)             &=& run time (10000$\times$hour+100$\times$minute+second)\\
IEVT               &=& event number\\
\end{tabular}

\begin{verbatim}
\end{verbatim}
\begininc{jetlim.inc}
\end{verbatim}
\begin{tabular}{lcl}
PMIN,PMAX          &=& jet momentum limits\\
PTMIN,PTMAX        &=& jet $p_t$ limits\\
YJMIN,YJMAX        &=& jet $\eta$ rapidity limits\\
PHIMIN,PHIMAX      &=& jet $\phi$ limits\\
THMIN,THMAX        &=& jet $\theta$ limits\\
\end{tabular}

\begin{verbatim}
\end{verbatim}
\begininc{keys.inc}
\end{verbatim}
\begin{tabular}{lcl}
KEYON              &=& normally TRUE, FALSE if no good reaction\\
KEYS               &=& TRUE if reaction I is chosen\\
                   && 1 for TWOJET\\
                   && 2 for E+E-\\
                   && 3 for DRELLYAN\\
                   && 4 for MINBIAS\\
                   && 5 for SUPERSYM\\
                   && 6 for WPAIR\\
REAC               &=& character reaction code\\
\end{tabular}

\begin{verbatim}
\end{verbatim}
\begininc{primar.inc}
\end{verbatim}
\begin{tabular}{lcl}
NJET               &=& number of jets per event\\
SCM                &=& square of com energy\\
HALFE              &=& beam energy\\
ECM                &=& com energy\\
IDIN               &=& ident code for initial beams\\
NEVENT             &=& number of events to be generated\\
NTRIES             &=& maximum number of tries for good jet parameters\\
NSIGMA             &=& number of extra events to determine SIGF\\
\end{tabular}

\begin{verbatim}
\end{verbatim}
\begininc{q1q2.inc}
\end{verbatim}
\begin{tabular}{lcl}
GOQ(I,K)           &=& TRUE if quark type I allowed for jet k\\
                   && I = 1  2  3  4  5  6  7  8  9 10 11 12 13\\
                   && \ \ $\Rightarrow$ $g$ $u$ $\bar u$ $d$ $\bar d$ $s$ 
                      $\bar s$ $c$ $\bar c$ $b$ $\bar b$ $t$ $\bar t$\\
                   && I = 14   15 16 17  18   19  20  21  22   23   24   25\\
                   && \ \ $\Rightarrow$ $\nu_e$ $\bar\nu_e$ $e^-$ $e^+$ 
                      $\nu_\mu$ $\bar\nu_\mu$ $\mu^-$ $\mu^+$ $\nu_\tau$ 
                      $\bar\nu_\tau$ $\tau^-$ $\tau^+$\\
GOALL(K)           &=& TRUE if all jet types allowed\\
GODY(I)            &=& TRUE if $W$ type I is allowed.\\
                    I= 1  2  3  4\\
                      GM W+ W- Z0\\
STDDY              &=& TRUE if standard DRELLYAN\\
GOWW(I,K)          &=& TRUE if I is allowed in the decay of K for WPAIR.\\
ALLWW(K)           &=& TRUE if all allowed in the decay of K for WPAIR.\\
\end{tabular}

\begin{verbatim}
\end{verbatim}
\begininc{qcdpar.inc}
\end{verbatim}
\begin{tabular}{lcl}
ALAM               &=& QCD scale $\Lambda$\\
ALAM2              &=& QCD scale $\Lambda^2$\\
CUTJET             &=& cutoff for generating secondary partons\\
ISTRUC             &=& 3 for Eichten (EHLQ), \\
                   &=& 4 for Duke (DO) \\
                   &=& 5 for CTEQ 2L\\
                   &=& 6 for CTEQ 3L\\
                   &=& $-999$ for PDFLIB\\
\end{tabular}

\begin{verbatim}
\end{verbatim}
\begininc{qlmass.inc}
\end{verbatim}
\begin{tabular}{lcl}
AMLEP(6:8)         &=& $t$,$y$,$x$ masses, only elements written\\
\end{tabular}

\subsection{Event Record}

      For each event ISAWEV is called. It writes out the following
common blocks:
\begin{verbatim}
\end{verbatim}
\begininc{final.inc}
\end{verbatim}
\begin{tabular}{lcl}
SIGF              &=& integrated cross section, only element written\\
\end{tabular}

\begin{verbatim}
\end{verbatim}
\begininc{idrun.inc}
\end{verbatim}
\begin{tabular}{lcl}
IDVER              &=& program version\\
IDG                &=& run identification\\
IEVT               &=& event number\\
\end{tabular}

\begin{verbatim}
\end{verbatim}
\begininc{jetpar.inc}
\end{verbatim}
\begin{tabular}{lcl}
P                  &=& jet momentum $\vert\vec p\vert$\\
PT                 &=& jet $p_t$\\
YJ                 &=& jet $\eta$ rapidity\\
PHI                &=& jet $\phi$\\
XJ                 &=& jet $x_F$\\
TH                 &=& jet $\theta$\\
CTH                &=& jet $\cos(\theta)$\\
STH                &=& jet $\sin(\theta)$\\
JETTYP             &=& jet type. The code is listed under /Q1Q2/ above\\
                   &&  {\it continued\dots}\\
\end{tabular}

\begin{tabular}{lcl}
SHAT,THAT,UHAT     &=& hard scattering $\hat s$, $\hat t$, $\hat u$\\
QSQ                &=& effective $Q^2$\\
X1,X2              &=& initial parton $x_F$\\
PBEAM              &=& remaining beam momentum\\
QMW                &=& $W$ mass\\
QW                 &=& $W$ momentum\\
QTW                &=& $W$ transverse momentum\\
YW                 &=& $W$ rapidity\\
XW                 &=& $W$ $x_F$\\
THW                &=& $W$ $\theta$\\
QTMW               &=& $\sqrt{q_{t,W}^2+Q^2}$\\
PHIW               &=& $W$ $\phi$\\
SHAT1,THAT1,UHAT1  &=& invariants for $W$ decay\\
JWTYP              &=& $W$ type. The code is listed under /Q1Q2/ above.\\
ALFQSQ             &=& QCD coupling $\alpha_s(Q^2)$\\
CTHW               &=& $W$ $\cos(\theta)$\\
STHW               &=& $W$ $\sin(\theta)$\\
Q0W                &=& $W$ energy\\
\end{tabular}

\begin{verbatim}
\end{verbatim}
\begininc{jetset.inc}
\end{verbatim}
\begin{tabular}{lcl}
NJSET              &=& number of partons\\
PJSET(1,I)         &=& $p_x$ of parton I\\
PJSET(2,I)         &=& $p_y$ of parton I\\
PJSET(3,I)         &=& $p_z$ of parton I\\
PJSET(4,I)         &=& $p_0$ of parton I\\
PJSET(5,I)         &=& mass of parton I\\
JORIG(I)           &=& JPACK*JET+K if I is a decay product of K.\\
                   && IF K=0 then I is a primary parton.\\
                   && (JET = 1,2,3 for final jets.)\\
                   && (JET = 11,12 for initial jets.)\\
JTYPE(I)           &=& IDENT code for parton I\\
JDCAY(I)           &=& JPACK*K1+K2 if K1 and K2 are decay products of I.\\
                   &&  If JDCAY(I)=0 then I is a final parton\\
MXJSET             &=& dimension for /JETSET/ arrays.\\
JPACK              &=& packing integer for /JETSET/ arrays.\\
\end{tabular}

\begin{verbatim}
\end{verbatim}
\begininc{jetsig.inc}
\end{verbatim}
\begin{tabular}{lcl}
SIGMA              &=& cross section summed over types\\
SIGS(I)            &=& cross section for reaction I (not written)\\
NSIGS              &=& number of nonzero cross sections (not written)\\
INOUT(I)           &=& packed partons for process I (not written)\\
MXSIGS             &=& dimension for JETSIG arrays (not written)\\
SIGEVT             &=& partial cross section for selected channel\\
\end{tabular}

\begin{verbatim}
\end{verbatim}
\begininc{partcl.inc}
\end{verbatim}
\begin{tabular}{lcl}
NPTCL              &=& number of particles\\
PPTCL(1,I)         &=& $p_x$ for particle I\\
PPTCL(2,I)         &=& $p_y$ for particle I\\
PPTCL(3,I)         &=& $p_z$ for particle I\\
PPTCL(4,I)         &=& $p_0$ for particle I\\
PPTCL(5,I)         &=& mass for particle I\\
IORIG(I)           &=& IPACK*JET+K if I is a decay product of K.\\
                   &=& -(IPACK*JET+K) if I is a primary particle from\\
                   &&  parton K in /JETSET/.\\
                   &=& 0 if I is a primary beam particle.\\
                   && (JET = 1,2,3 for final jets.)\\
                   && (JET = 11,12 for initial jets.)\\
IDENT(I)           &=& IDENT code for particle I\\
IDCAY(I)           &=& IPACK*K1+K2 if decay products are K1-K2 inclusive.\\
                   && If IDCAY(I)=0 then particle I is stable.\\
MXPTCL             &=& dimension for /PARTCL/ arrays.\\
IPACK              &=& packing integer for /PARTCL/ arrays.\\
\end{tabular}

\begin{verbatim}
\end{verbatim}
\begininc{pinits.inc}
\end{verbatim}
\begin{tabular}{lcl}
PINITS(1,I)        &=& $p_x$ for initial parton I\\
PINITS(2,I)        &=& $p_y$ for initial parton I\\
PINITS(3,I)        &=& $p_z$ for initial parton I\\
PINITS(4,I)        &=& $p_0$ for initial parton I\\
PINITS(5,I)        &=& mass for initial parton I\\
IDINIT(I)          &=& IDENT for initial parton I\\
\end{tabular}

\begin{verbatim}
\end{verbatim}
\begininc{pjets.inc}
\end{verbatim}
\begin{tabular}{lcl}
PJETS(1,I)         &=& $p_x$ for jet I\\
PJETS(2,I)         &=& $p_y$ for jet I\\
PJETS(3,I)         &=& $p_z$ for jet I\\
PJETS(4,I)         &=& $p_0$ for jet I\\
PJETS(5,I)         &=& mass for jet I\\
IDJETS(I)          &=& IDENT code for jet I\\
QWJET(1)           &=& $p_x$ for $W$\\
QWJET(2)           &=& $p_y$ for $W$\\
QWJET(3)           &=& $p_z$ for $W$\\
QWJET(4)           &=& $p_0$ for $W$\\
QWJET(5)           &=& mass for $W$\\
IDENTW             &=& IDENT CODE for $W$\\
PPAIR(1,I)         &=& $p_x$ for WPAIR decay product I\\
PPAIR(2,I)         &=& $p_y$ for WPAIR decay product I\\
PPAIR(3,I)         &=& $p_z$ for WPAIR decay product I\\
PPAIR(4,I)         &=& $p_0$ for WPAIR decay product I\\
PPAIR(5,I)         &=& mass for WPAIR decay product I\\
IDPAIR(I)          &=& IDENT code for WPAIR product I\\
JPAIR(I)           &=& JETTYPE code for WPAIR product I\\
NPAIR              &=& 2 for $W^\pm\gamma$ events, 4 for $WW$ events\\
\end{tabular}

\begin{verbatim}
\end{verbatim}
\begininc{totals.inc}
\end{verbatim}
\begin{tabular}{lcl}
NKINPT             &=& number of kinematic points generated.\\
NWGEN              &=& number of W+jet events accepted.\\
NKEEP              &=& number of events kept.\\
SUMWT              &=& sum of weighted cross sections.\\
WT                 &=& current weight. (SIGMA$\times$WT = event weight.)\\
\end{tabular}

\begin{verbatim}
\end{verbatim}
\begininc{wsig.inc}
\end{verbatim}
\begin{tabular}{lcl}
SIGLLQ             &=& cross section for $W$ decay.\\
\end{tabular}

      Of course irrelevant common blocks such as /WSIG/ for TWOJET
events are not written out.

\subsection{End Record} 

      At the end of a set ISAWND is called. It writes out the
following common block:
\begin{verbatim}
\end{verbatim}
\begininc{final.inc}
\end{verbatim}
\begin{tabular}{lcl}
NKINF             &=& number of points generated to calculate SIGF\\
SIGF              &=& integrated cross section for this run\\
ALUM              &=& equivalent luminosity for this run\\
ACCEPT            &=& ratio of events kept over events generated\\
NRECS             &=& number of physical records for this run\\
\end{tabular}

      Events within a given run have uniform weight. Separate runs can
be combined together using SIGF/NEVENT as the weight per event. This
gives a true cross section in mb units.

      The user can replace subroutines ISAWBG, ISAWEV, and ISAWND to
write out the events in a different format or to update histograms
using HBOOK or any similar package.

\subsection{LesHouches Event output}

A Les Houches Event accord (LHE) was developed in which any parton level 
event generator could dump out events in a particular format, which could 
then be read into a general purpose event generator to add 
parton showering, hadronization and underlying event structure:
see E.Boos {\it et al.}, hep-ph/0109068. 
Later, a standard format for Les Houches Events was developed,
whereby LHE events would be written out in a standard format, 
which could then be directly read in to general purpose event generators:
see J. Alwall {\it et al.}, hep-ph/0609017.
While Isajet does not allow to read in LHE output, version 7.76 and above 
do allow one to {\it write out} LHE events in a format which can be read in by
event generators Pythia or Herwig. Since color flow is accounted for in the
Isajet LHE events, these can then be showered and hadronized by programs
which include color flow in their hadronization schemes.

The Isajet LHE output is enabled by use of the keyword WRTLHE in the 
input file. By default, WRTLHE is FALSE, but by stipulating it to be TRUE,
then showering, hadronization and underlying event is turned off, 
and the subprocess events followed by (cascade) decays 
including color flow information are written in standard format to a 
file called \verb|isajet.lhe|, which in turn should be readable by Pythia 
or Herwig.
If running for ALL SUSY JETYPEs, then Isajet generates well over 1000
subprocesses, which is more than Pythia can input. Thus, the LHE lines 
containing subprocess listing and cross sections stops at a maximum of 500.
However, the events in the LHE file are stll in accord with the complete
total cross section and the complete list of all allowed subprocesses.
Furthermore, the LHE output XSECUP contains $d\sigma /dp_T^2dy_1/dy_2$
evaluated at $y_i=0$ for hadronic interactions and $d\sigma /dz$ for
$e^+e^-$. The LHE listed subprocess differential cross sections are given in $mb/GeV^2$ or $mb$, 
respectively. Also, the subprocess label
LPRUP is given using the Isajet internal INOUT code.
Presumably these lines are unnecessary for use of LHEs.

If SUSY processes are generated with Isajet and sent to Pythia 
for showering, hadronization etc., then Pythia will need to know 
that the lightest neutralino is stable. One can set the \verb|Z1SS|
to be stable in the Pythia read-in code by setting
\verb|CALL PYGIVE('MDCY(C1000022,1)=0')|. 
Alternatively, if one includes the
\verb|PYDAT3| common block, one can instead set 
\begin{verbatim}
      MDCY(PYCOMP(1000022),1)=0 
\end{verbatim}


%%%%%%%%%%%%%%%%%%%%%%%%%%%%%%%%%%%%%%%%%%%%%%%%%%%%%%%%%%%%%%%%%%%%
\chapter{Basic principles and common routines}

The general concept of  a particle physics module, and how it communicates with the \code{core}
library via interface functions, was already introduced in Chapter \ref{ch:philosophy} -- see in particular
Section \ref{sec:particle_modules}, and Table \ref{tab:if} for an automatically updated list of all interface 
functions that the \code{core} library is aware of. Note that there is no {\it principal} restriction on 
which interface functions a particle module must provide: the main program will determine
at compilation time whether it needs a functionality of the \code{core} library that requires 
certain interface functions to exist.

Every particle physics module must provide an independent representation of the
full particle content of the respective particle theory. How this is done is fully flexible, and 
completely up to the module. In practice, however, there are common frameworks,
like the Standard Model, that appear repeatedly. For convenience, we therefore store auxiliary 
routines and setups that {\it may} be used by more than just one particle module in 
\code{src\_models/common}. Common blocks and header files included by more than one
particle module are found in \code{src\_model/include}. In order to keep up the modularity
of the code, routines in \code{src\_models/common} thus {\it only} include files in 
\code{src\_model/include}.

\comment{The subsections below should probably be automatically harvested -- but this does not work (yet)...}


\section{\code{common/aux}: auxiliary routines}

Here we collect various routines that belong to particle physics, and hence do not reside in \code{ds\_core},
but are not only useful for one specific particle module. Currently, the most important are
\begin{itemize}
\item A set of functions to handle {\it unique model IDs} for each particle model, which is essential
        in order not to repeat identical calculations when changing the model and thus to optimize 
        numerical performance. A new such model 
        ID should be assigned with \code{dsnewidnumber} whenever a new particle model is initalized 
        (for the modules provided with the \ds\ release, this is automatically done in \code{dsmodelsetup}).
        Any routine in \code{src\_models} can test with \code{dsidnumberset} whether this ID number
        is indeed set, and retrieve its value with \code{dsidnumber} (which then can be compared 
        to the locally stored value of the ID number that was valid at the time when the respective routine 
        was called the previous time).
\item The function \code{dsanthreshold} returns the correction factor to a 2-body rate close to a 
         kinematical threshold, resulting from the fact that one or both of the particles may be slightly off-shell.
         This implements the simplified treatment presented in Ref.~\cite{Bringmann:2017sko}, c.f.~their Fig.~5,
         and hence assumes that the decay products of the virtual particle are (effectively) massless.
\end{itemize}


\section{\code{common/sm}: standard model}

The routines collected in this folder provide a convenient shortcut to include the most basic 
properties of the standard model in a BSM particle module. Currently, the most important functionalites
collected here are given by

\begin{itemize}
\item An initialization routine \code{dsinit\_sm}, which can be called directly from the corresponding 
        \code{dsinit\_module}. After a call to this routine, the functions \code{dsmass(kPDG)} and 
        \code{dswidth(kPDG)} correctly return masses and widths of the standard model particles 
        -- \code{kPDG} being the PDG code  \cite{Groom:2000in} -- as stored centrally in 
        \code{src\_models/include/dssmparam.h}.
\item Running quark masses are provided up to the 4-loop level, with SM contributions only, 
         both for pole and $\overline{MS}$ masses \cite{Chetyrkin:2000yt}.
\item The SM contributions to the strong coupling constant are also provided up to the 4-loop level.
 \item Another convenience function is \code{dsgf2s2thw}, which calculates $\sin^2(\theta_W)$ at the 
         $m_Z$ scale: in praxis, its most important application is to ensure that a particle module can adopt a 
         consistent relation between $m_Z$ and $m_W$, which for example can be crucial in order to 
         keep large numerical cancellations under control.
\end{itemize}

Let us stress that standard model physics in \ds\  is by far not restricted to the 
routines collected in \code{common/sm}. Much is presently still contained in the \code{mssm} module.
It will be (further) disentangled from the SUSY-specific parts as new modules are added to the code that
require this functionality. 



%%%%%%%%%%%%%%%%%%%%%%%%%%%%%%%%%%%%%%%%%%%%%%%%%%%%%%%%%%%%%%%%%%%%
\label{ch:vdSIDM_rd}

In \code{src\_models/vdSIDM/rd} we collect a set of routines to handle the temperature of the dark sector
in this module. First of all, the interface function \code{dsrdxi} returns the temperature ratio $\xi=T_{\tilde\gamma}/T$,
where $T$ is the photon temperature. This is calculated by assuming that the two sectors were  in thermal
equilubrium ($\xi=1$) at very early times when the DM particles were still relativistic, e.g.~due to some portal
that is effective only at very high energies and hence not relevant for the low-energy Lagrangian. 
After decoupling of the two sectors at a temperature $T_\mathrm{dc}\gg m_\chi$, 
entropy would be separately conserved, which implies that the temperature evolves as 
\be
 \xi(T) = 
 \frac{\left[{g_*^\mathrm{SM}}(T)/{g_*^\mathrm{DS}(T)}\right]^\frac13}
 {\left[{g_*^\mathrm{SM}}(T_\mathrm{dc})/{g_*^\mathrm{DS}(T_\mathrm{dc})}\right]^\frac13}\,,
 \ee
where $g_*^{\mathrm{SM}, \mathrm{DS}}$ are the entropy degrees of freedom in the visible
and dark sector, respectively. Of course,  \code{dsrdxi} could be replaced in the usual way by any user-supplied function 
in order to implement a different scenario for how the temperature of the dark sector evolves with time.
Care must be taken, however, to ensure that such an implementation is consistent with the model assumptions.
A constant value of $\xi$ for example, as is often assumed for illustration, is {\it not} consistent with the assumption
that there is no interaction between dark and visible sector.

Any additional relativistic energy density, as provided by the DR particles, is conventionally stated in terms of
\be
 \Delta N_\mathrm{eff}\equiv \frac{\rho_\mathrm{DS}}{\rho_\mathrm{1\nu}}=  
 \frac47g_*^\mathrm{DS}\left(\frac{T_{\tilde\gamma}}{T_\nu}\right)^4\,,
\ee
where $\rho_\mathrm{1\nu}$ is the energy density contributed by one massless neutrino species, and $T_\nu$ is the neutrino
temperature (which differs from the photon temperature after $e^\pm$ annihilation). For convenience, \ds\ provides a 
function \code{dsrddeltaneff} to compute $\Delta N_\mathrm{eff}$ as a function of (photon) temperature.

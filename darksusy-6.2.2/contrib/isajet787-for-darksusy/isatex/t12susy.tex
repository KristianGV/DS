\newpage
\section{ISASUSY: Decay Modes in the Minimal Supersymmetric
Model\label{SUSY}}

      The code in patch ISASUSY of ISAJET calculates decay modes of
supersymmetric particles based on the work of H. Baer, M. Bisset, M.
Drees, D. Dzialo (Karatas), X. Tata, J. Woodside, and their
collaborators. The calculations assume the minimal supersymmetric
extension of the standard model. The user specifies the gluino mass,
the pseudoscalar Higgs mass, the Higgsino mass parameter $\mu$,
$\tan\beta$, the soft breaking masses for the first and third
generation left-handed squark and slepton doublets and right-handed
singlets, and the third generation mixing parameters $A_t$, $A_b$, and
$A_\tau$.  Supersymmetric grand unification is assumed by default in
the chargino and neutralino mass matrices, although the user can
optionally specify arbitrary $U(1)$ and $SU(2)$ gaugino masses at the
weak scale. The first and second generations are assumed by default to
be degenerate, but the user can optionally specify different values.
These inputs are then used to calculate the mass eigenstates, mixings,
and decay modes.

      Most calculations are done at the tree level, but one-loop
results for gluino loop decays, $H \to \gamma\gamma$ and $H \to gg$, loop
corrections to the Higgs mass spectrum and couplings, and leading-log
QCD corrections to $H \to q \bar q$ are included. The Higgs masses have
been calculated using the effective potential approximation including
both top and bottom Yukawa and mixing effects. Mike Bisset and Xerxes
Tata have contributed the Higgs mass, couplings, and decay routines.
Manuel Drees has calculated several of the three-body decays including
the full Yukawa contribution, which is important for large tan(beta).
Note that e+e- annihilation to SUSY particles and SUSY Higgs bosons
have been included in ISAJET versions $>7.11$. ISAJET versions $>7.22$
include the large $\tan\beta$ solution as well as non-degenerate
sfermion masses.

Other processes may be added in future versions as the physics 
interest warrants. Note that
the details of the masses and the decay modes can be quite sensitive
to choices of standard model parameters such as the QCD coupling ALFA3
and the quark masses.  To change these, you must modify subroutine
SSMSSM. By default, ALFA3=.12.

      All the mass spectrum and branching ratio calculations in ISASUSY 
are performed by a call to subroutine SSMSSM. Effective with version 7.23,
the calling sequence is
\begin{verbatim}
      SUBROUTINE SSMSSM(XMG,XMU,XMHA,XTANB,XMQ1,XMDR,XMUR,
     $XML1,XMER,XMQ2,XMSR,XMCR,XML2,XMMR,XMQ3,XMBR,XMTR,
     $XML3,XMLR,XAT,XAB,XAL,XM1,XM2,XMT,IALLOW,IMODEL)
\end{verbatim}
where the following are taken to be independent parameters:

\smallskip\noindent
\begin{tabular}{lcl}
      XMG    &=& gluino mass\\
      XMU    &=& $\mu$ = SUSY Higgs mass\\
             &=& $-2*m_1$ of Baer et al.\\
      XMHA   &=& pseudo-scalar Higgs mass\\
      XTANB  &=& $\tan\beta$, ratio of vev's\\
             &=& $1/R$ (of old Baer-Tata notation).\\
\end{tabular}

\noindent
\begin{tabular}{lcl}
      XMQ1   &=& $\tilde q_l$ soft mass, 1st generation\\
      XMDR   &=& $\tilde d_r$ mass, 1st generation\\
      XMUR   &=& $\tilde u_r$ mass, 1st generation\\
      XML1   &=& $\tilde \ell_l$ mass, 1st generation\\
      XMER   &=& $\tilde e_r$ mass, 1st generation\\
\\
      XMQ2   &=& $\tilde q_l$ soft mass, 2nd generation\\
      XMSR   &=& $\tilde s_r$ mass, 2nd generation\\
      XMCR   &=& $\tilde c_r$ mass, 2nd generation\\
      XML2   &=& $\tilde \ell_l$ mass, 2nd generation\\
      XMMR   &=& $\tilde\mu_r$ mass, 2nd generation\\
\\
      XMQ3   &=& $\tilde q_l$ soft mass, 3rd generation\\
      XMBR   &=& $\tilde b_r$ mass, 3rd generation\\
      XMTR   &=& $\tilde t_r$ mass, 3rd generation\\
      XML3   &=& $\tilde \ell_l$ mass, 3rd generation\\
      XMTR   &=& $\tilde \tau_r$ mass, 3rd generation\\
      XAT    &=& stop trilinear term $A_t$\\
      XAB    &=& sbottom trilinear term $A_b$\\
      XAL    &=& stau trilinear term $A_\tau$\\
\\
      XM1    &=& U(1) gaugino mass\\
             &=& computed from XMG if > 1E19\\
      XM2    &=& SU(2) gaugino mass\\
             &=& computed from XMG if > 1E19\\
\\
      XMT    &=& top quark mass\\
      IALLOW &=& return flag\\
      IMODEL &=& 1 for SUGRA or MSSM\\
             &=& 2 for GMSB
\end{tabular}
\smallskip

\noindent The variable IALLOW is returned:

\smallskip\noindent
\begin{tabular}{lcl}
      IALLOW &=& 1 if Z1SS is not LSP, 0 otherwise\\
\end{tabular}
\smallskip

\noindent All variables are of type REAL except IALLOW and IMODEL, which
are INTEGER, and all masses are in GeV. The notation is taken to
correspond to that of Haber and Kane, although the Tata Lagrangian is
used internally. All other standard model parameters are hard wired in
this subroutine; they are not obtained from the rest of ISAJET. The
theoretically favored range of these parameters is
\begin{eqnarray*}
& 100 < M(\tilde g) < 2000\,\GeV &\\
& 100 < M(\tilde q) < 2000\,\GeV &\\
& 100 < M(\tilde\ell) < 2000\,\GeV &\\
& -1000 < \mu < 1000\,\GeV &\\
& 1 < \tan\beta < m_t/m_b &\\
& M(t) \approx 175\,\GeV &\\
& 100 < M(A) < 2000\,\GeV &\\
& M(\tilde t_l), M(t_r) < M(\tilde q) &\\
& M(\tilde b_r) \sim M(\tilde q) &\\
& -1000 < A_t < 1000\,\GeV &\\
& -1000 < A_b < 1000\,\GeV &
\end{eqnarray*}
It is assumed that the lightest supersymmetric particle is the lightest
neutralino $\tilde Z_1$, the lighter stau $\tilde\tau_1$, or the
gravitino $\tilde G$ in GMSB models. Some choices of the above
parameters may violate this assumption, yielding a light chargino or
light stop squark lighter than $\tilde Z_1$. In such cases SSMSSM does
not compute any branching ratios and returns IALLOW = 1.

      SSMSSM does not check the parameters or resulting masses against
existing experimental data. SSTEST provides a minimal test. This routine
is called after SSMSSM by ISAJET and ISASUSY and prints suitable warning
messages.

      SSMSSM first calculates the other SUSY masses and mixings and puts
them in the common block /SSPAR/:
\begin{verbatim}
C          SUSY parameters
C          AMGLSS               = gluino mass
C          AMULSS               = up-left squark mass
C          AMELSS               = left-selectron mass
C          AMERSS               = right-slepton mass
C          AMNiSS               = sneutrino mass for generation i
C          TWOM1                = Higgsino mass = - mu
C          RV2V1                = ratio v2/v1 of vev's
C          AMTLSS,AMTRSS        = left,right stop masses
C          AMT1SS,AMT2SS        = light,heavy stop masses
C          AMBLSS,AMBRSS        = left,right sbottom masses
C          AMB1SS,AMB2SS        = light,heavy sbottom masses
C          AMLLSS,AMLRSS        = left,right stau masses
C          AML1SS,AML2SS        = light,heavy stau masses
C          AMZiSS               = signed mass of Zi
C          ZMIXSS               = Zi mixing matrix
C          AMWiSS               = signed Wi mass
C          GAMMAL,GAMMAR        = Wi left, right mixing angles
C          AMHL,AMHH,AMHA       = neutral Higgs h0, H0, A0 masses
C          AMHC                 = charged Higgs H+ mass
C          ALFAH                = Higgs mixing angle
C          AAT                  = stop trilinear term
C          THETAT               = stop mixing angle
C          AAB                  = sbottom trilinear term
C          THETAB               = sbottom mixing angle
C          AAL                  = stau trilinear term
C          THETAL               = stau mixing angle
C          AMGVSS               = gravitino mass
C          MTQ                  = top mass at MSUSY
C          MBQ                  = bottom mass at MSUSY
C          MLQ                  = tau mass at MSUSY
C          FBMA                 = b-Yukawa at mA scale
C          VUQ                  = Hu vev at MSUSY
C          VDQ                  = Hd vev at MSUSY
      COMMON/SSPAR/AMGLSS,AMULSS,AMURSS,AMDLSS,AMDRSS,AMSLSS
     $,AMSRSS,AMCLSS,AMCRSS,AMBLSS,AMBRSS,AMB1SS,AMB2SS
     $,AMTLSS,AMTRSS,AMT1SS,AMT2SS,AMELSS,AMERSS,AMMLSS,AMMRSS
     $,AMLLSS,AMLRSS,AML1SS,AML2SS,AMN1SS,AMN2SS,AMN3SS
     $,TWOM1,RV2V1,AMZ1SS,AMZ2SS,AMZ3SS,AMZ4SS,ZMIXSS(4,4)
     $,AMW1SS,AMW2SS
     $,GAMMAL,GAMMAR,AMHL,AMHH,AMHA,AMHC,ALFAH,AAT,THETAT
     $,AAB,THETAB,AAL,THETAL,AMGVSS,MTQ,MBQ,MLQ,FBMA,
     $VUQ,VDQ
      REAL AMGLSS,AMULSS,AMURSS,AMDLSS,AMDRSS,AMSLSS
     $,AMSRSS,AMCLSS,AMCRSS,AMBLSS,AMBRSS,AMB1SS,AMB2SS
     $,AMTLSS,AMTRSS,AMT1SS,AMT2SS,AMELSS,AMERSS,AMMLSS,AMMRSS
     $,AMLLSS,AMLRSS,AML1SS,AML2SS,AMN1SS,AMN2SS,AMN3SS
     $,TWOM1,RV2V1,AMZ1SS,AMZ2SS,AMZ3SS,AMZ4SS,ZMIXSS
     $,AMW1SS,AMW2SS
     $,GAMMAL,GAMMAR,AMHL,AMHH,AMHA,AMHC,ALFAH,AAT,THETAT
     $,AAB,THETAB,AAL,THETAL,AMGVSS,MTQ,MBQ,MLQ,FBMA,VUQ,VDQ
      REAL AMZISS(4)
      EQUIVALENCE (AMZISS(1),AMZ1SS)
      SAVE /SSPAR/
\end{verbatim}
It then calculates the widths and branching ratios and puts them in the
common block /SSMODE/:
\begin{verbatim}
C          MXSS         =  maximum number of modes
C          NSSMOD       = number of modes
C          ISSMOD       = initial particle
C          JSSMOD       = final particles
C          GSSMOD       = width
C          BSSMOD       = branching ratio
C          MSSMOD       = decay matrix element pointer
C          LSSMOD       = logical flag used internally by SSME3
      INTEGER MXSS
      PARAMETER (MXSS=1000)
      COMMON/SSMODE/NSSMOD,ISSMOD(MXSS),JSSMOD(5,MXSS),GSSMOD(MXSS)
     $,BSSMOD(MXSS),MSSMOD(MXSS),LSSMOD
      INTEGER NSSMOD,ISSMOD,JSSMOD,MSSMOD
      REAL GSSMOD,BSSMOD
      LOGICAL LSSMOD
      SAVE /SSMODE/
\end{verbatim}
Decay modes for a given particle are not necessarily adjacent in this
common block.  Note that the branching ratio calculations use the full
matrix elements, which in general will give nonuniform distributions in
phase space, but this information is not saved in /SSMODE/.  In
particular, the decays $H \to Z + Z^* \to Z + f + \bar f$ give no
indication that the $f \bar f$ mass is strongly peaked near the upper
limit.

      All IDENT codes are defined by parameter statements in the PATCHY
keep sequence SSTYPE:
\begin{verbatim}
C          SM ident code definitions. These are standard ISAJET but
C          can be changed.
      INTEGER IDUP,IDDN,IDST,IDCH,IDBT,IDTP
      INTEGER IDNE,IDE,IDNM,IDMU,IDNT,IDTAU
      INTEGER IDGL,IDGM,IDW,IDZ,IDH
      PARAMETER (IDUP=1,IDDN=2,IDST=3,IDCH=4,IDBT=5,IDTP=6)
      PARAMETER (IDNE=11,IDE=12,IDNM=13,IDMU=14,IDNT=15,IDTAU=16)
      PARAMETER (IDGL=9,IDGM=10,IDW=80,IDZ=90,IDH=81)
C          SUSY ident code definitions. They are chosen to be similar
C          to those in versions < 6.50 but may be changed.
      INTEGER ISUPL,ISDNL,ISSTL,ISCHL,ISBT1,ISTP1
      INTEGER ISNEL,ISEL,ISNML,ISMUL,ISNTL,ISTAU1
      INTEGER ISUPR,ISDNR,ISSTR,ISCHR,ISBT2,ISTP2
      INTEGER ISNER,ISER,ISNMR,ISMUR,ISNTR,ISTAU2
      INTEGER ISZ1,ISZ2,ISZ3,ISZ4,ISW1,ISW2,ISGL
      INTEGER ISHL,ISHH,ISHA,ISHC
      INTEGER ISGRAV
      INTEGER IDTAUL,IDTAUR
      PARAMETER (ISUPL=21,ISDNL=22,ISSTL=23,ISCHL=24,ISBT1=25,ISTP1=26)
      PARAMETER (ISNEL=31,ISEL=32,ISNML=33,ISMUL=34,ISNTL=35,ISTAU1=36)
      PARAMETER (ISUPR=41,ISDNR=42,ISSTR=43,ISCHR=44,ISBT2=45,ISTP2=46)
      PARAMETER (ISNER=51,ISER=52,ISNMR=53,ISMUR=54,ISNTR=55,ISTAU2=56)
      PARAMETER (ISGL=29)
      PARAMETER (ISZ1=30,ISZ2=40,ISZ3=50,ISZ4=60,ISW1=39,ISW2=49)
      PARAMETER (ISHL=82,ISHH=83,ISHA=84,ISHC=86)
      PARAMETER (ISGRAV=91)
      PARAMETER (IDTAUL=10016,IDTAUR=20016)
\end{verbatim}
These are based on standard ISAJET but can be changed to interface with
other generators.  Since masses except the t mass are hard wired, one
should check the kinematics for any decay before using it with possibly
different masses.

      Instead of specifying all the SUSY parameters at the electroweak
scale using the MSSMi commands, one can instead use the SUGRA parameter
to specify in the minimal supergravity framework the common scalar mass
$m_0$, the common gaugino mass $m_{1/2}$, and the soft trilinear SUSY
breaking parameter $A_0$ at the GUT scale, the ratio $\tan\beta$ of
Higgs vacuum expectation values at the electroweak scale, and $\sgn\mu$,
the sign of the Higgsino mass term. The \verb|NUSUGi| keywords allow one
to break the assumption of universality in various ways. \verb|NUSUG1|
sets the gaugino masses; \verb|NUSUG2| sets the $A$ terms; \verb|NUSUG3|
sets the Higgs masses; \verb|NUSUG4| sets the first generation squark
and slepton masses; and \verb|NUSUG5| sets the third generation masses.
The keyword \verb|SSBCSC| can be used to specify an alternative scale
(i.e., not the coupling constant unification scale) for the RGE boundary
conditions.

      The renormalization group equations now include all the two-loop
terms for both gauge and Yukawa couplings and the possible contributions
from right-handed neutrinos. These equations are solved iteratively using
Runge-Kutta numerical integration to determine the weak scale parameters
from the GUT scale ones:
\begin{enumerate}
%
\item The 2-loop RGE's for the gauge, Yukawa, and soft breaking terms
are run from the weak scale $M_Z$ up to the GUT scale (where $\alpha_1 =
\alpha_2$) or other high scale defined by the model, taking all
thresholds into account.
%
\item The GUT scale boundary conditions are imposed, and the RGE's are
run back to $M_Z$, again taking thresholds into account.
%
\item The masses of the SUSY particles and the values of the soft 
breaking parameters B and mu needed for radiative symmetry are
computed, e.g.
$$
\mu^2(M_Z) = {M_{H_1}^2 - M_{H_2}^2  \tan^2\beta \over
\tan^2\beta-1} - M_Z^2/2
$$
These couplings are frozen out at the scale $\sqrt{M(t_L)M(t_R)}$.
%
\item The 1-loop radiative corrections are computed.
%
\item The process is then iterated until stable results are obtained.
\end{enumerate}
This is essentially identical to the procedure used by several other
groups. Other possible constraints such as $b$-$\tau$ unification and 
limits on proton decay have not been included.

      An alternative to the SUGRA model is the Gauge Mediated SUSY
Breaking (GMSB) model of Dine and Nelson, Phys.\ Rev.\ {\bf D48}, 1277
(1973); Dine, Nelson, Nir, and Shirman, Phys.\ Rev.\ {\bf D53}, 2658
(1996). In this model SUSY is broken dynamically and communicated to the
MSSM through messenger fields at a messenger mass scale $M_m$ much less
than the Planck scale. If the messenger fields are in complete
representations of $SU(5$), then the unification of couplings suggested
by the LEP data is preserved. The simplest model has a single $5+\bar5$
messenger sector with a mass $M_m$ and and a SUSY-breaking VEV $F_m$ of
its auxiliary field $F$. Gauginos get masses from one-loop graphs
proportional to $\Lambda_m = F_m / M_m$ times the appropriate gauge
coupling $\alpha_i$; sfermions get squared-masses from two-loop graphs
proportional to $\Lambda_m$ times the square of the appropriate
$\alpha_i$. If there are $N_5$ messenger fields, the gaugino masses and
sfermion masses-squared each contain a factor of $N_5$.

      The parameters of the GMSB model implemented in ISAJET are
\begin{itemize}
\item $\Lambda_m = F_m/M_m$: the scale of SUSY breaking, typically
10--$100\,{\rm TeV}$;
\item $M_m > \Lambda_m$: the messenger mass scale, at which the boundary
conditions for the renormalization group equations are imposed;
\item $N_5$: the equivalent number of $5+\bar5$ messenger fields.
\item $\tan\beta$: the ratio of Higgs vacuum expectation values at the
electroweak scale;
\item $\sgn\mu=\pm1$: the sign of the Higgsino mass term;
\item $C_{\rm grav}\ge1$: the ratio of the gravitino mass to the value it
would have had if the only SUSY breaking scale were $F_m$.
\end{itemize}
The solution of the renormalization group equations is essentially the
same as for SUGRA; only the boundary conditions are changed. In
particular it is assumed that electroweak symmetry is broken radiatively
by the top Yukawa coupling.

      In GMSB models the lightest SUSY particle is always the nearly
massless gravitino $\tilde G$. The phenomenology depends on the nature
of the next lightest SUSY particle (NLSP) and on its lifetime to decay
to a gravitino. The NLSP can be either a neutralino $\tilde\chi_1^0$ or
a slepton $\tilde\tau_1$. Its lifetime depends on the gravitino mass,
which is determined by the scale of SUSY breaking not just in the
messenger sector but also in any other hidden sector. If this is set by
the messenger scale $F_m$, i.e., if $C_{\rm grav}\approx1$, then this
lifetime is generally short. However, if the messenger SUSY breaking
scale $F_m$ is related by a small coupling constant to a much larger
SUSY breaking scale $F_b$, then $C_{\rm grav}\gg1$ and the NLSP can be
long-lived. The correct scale is not known, so $C_{\rm grav}$ should be
treated as an arbitrary parameter. More complicated GMSB models may be
run by using the GMSB2 keyword.

      Patch ISASSRUN of ISAJET provides a main program SSRUN and some
utility programs to produce human readable output.  These utilities must
be rewritten if the IDENT codes in /SSTYPE/ are modified.  To create the
stand-alone version of ISASUSY with SSRUN, run YPATCHY on isajet.car
with the following cradle (with \verb|&| replaced by \verb|+|):
\begin{verbatim}
&USE,*ISASUSY.                         Select all code
&USE,NOCERN.                           No CERN Library
&USE,IMPNONE.                          Use IMPLICIT NONE
&EXE.                                  Write everything to ASM
&PAM,T=C.                              Read PAM file
&QUIT.                                 Quit
\end{verbatim}
Compile, link, and run the resulting program, and follow the prompts for
input.  Patch ISASSRUN also contains a main program SUGRUN that reads
the minimal SUGRA, non-universal SUGRA, or GMSB parameters, solves the
renormalization group equations, and calculates the masses and branching
ratios. To create the stand-alone version of ISASUGRA, run YPATCHY with
the following cradle:
\begin{verbatim}
&USE,*ISASUGRA.                        Select all code
&USE,NOCERN.                           No CERN Library
&USE,IMPNONE.                          Use IMPLICIT NONE
&EXE.                                  Write everything to ASM
&PAM.                                  Read PAM file
&QUIT.                                 Quit
\end{verbatim}
The documentation for ISASUSY and ISASUGRA is included with that for
ISAJET.

      ISASUSY is written in ANSI standard Fortran 77 except that
IMPLICIT NONE is used if +USE,IMPNONE is selected in the Patchy cradle. 
All variables are explicitly typed, and variables starting with
I,J,K,L,M,N are not necessarily integers.  All external names such as
the names of subroutines and common blocks start with the letters SS. 
Most calculations are done in double precision.  If +USE,NOCERN is
selected in the Patchy cradle, then the Cernlib routines EISRS1 and its
auxiliaries to calculate the eigenvalues of a real symmetric matrix and
DDILOG to calculate the dilogarithm function are included.  Hence it is
not necessary to link with Cernlib.

      The physics assumptions and details of incorporating the Minimal
Supersymmetric Model into ISAJET have appeared in a conference
proceedings entitled ``Simulating Supersymmetry with ISAJET 7.0/ISASUSY
1.0'' by H. Baer, F. Paige, S. Protopopescu and X. Tata; this has
appeared in the proceedings of the workshop on {\sl Physics at Current
Accelerators and Supercolliders}, ed.\ J. Hewett, A. White and D.
Zeppenfeld, (Argonne National Laboratory, 1993). Detailed references
may be found therein. Users wishing to cite an appropriate source should,
however, cite the most recent ISAJET manual, e.g. hep-ph/0001086 (1999).

\newpage
\section{IsaTools \label{Tools}}

\def\te{\tilde e}
\def\tl{\tilde l}
\def\tu{\tilde u}
\def\ts{\tilde s}
\def\tb{\tilde b}
\def\tc{\tilde c}
\def\tf{\tilde f}
\def\td{\tilde d}
\def\tQ{\tilde Q}
\def\tL{\tilde L}
\def\tH{\tilde H}
\def\tst{\tilde t}
\def\ttau{\tilde \tau}
\def\tmu{\tilde \mu}
\def\tg{\tilde g}
\def\tnu{\tilde\nu}
\def\tell{\tilde\ell}
\def\tg{\tilde g}
\def\tq{\tilde q}
\def\tw{\tilde W}
\def\tz{\tilde Z}
\def\alt{\stackrel{<}{\sim}}
\def\agt{\stackrel{>}{\sim}}
\hyphenation{super-symmet-ry}

\verb|IsaTools| is a optional set of subroutines included with Isajet
7.72 and later for the evaluation of various low-energy and cosmological
constraints on supersymmetric models using the Isajet supersymmetry
code. The package consists of:
\begin{enumerate}
\item \verb|IsaRED|, subroutines to evaluate the relic density of
(stable) neutralino dark matter in the universe;
\item \verb|IsaBSG|, subroutines to evalue the branching fraction
$BF(b\to s\gamma )$;
\item \verb|IsaAMU|, subroutines to evaluate supersymmetric contributions to
$\Delta a_\mu\equiv (g-2)_\mu/2$;
\item \verb|IsaBMM|, subroutines to evaluate $BF(B_s\rightarrow
\mu^+\mu^-)$ and $BF(B_d\rightarrow \tau^+\tau^- )$ in the MSSM;
\item \verb|IsaRES|, subroutines to evaluate the spin-independent and
spin-dependent neutralino-proton and neutralino-neutron scattering cross
sections for direct detection of dark matter.
\end{enumerate}
Below we provide a brief description of each subroutine along with
appropriate references. The main code author is indicated by
an underline and should be the first choice of contact for any bug 
related problems.

Details on the application of IsaTools and on the structure of IsaRED,
IsaBSG, IsaBMM and IsaRES could be found at the web page\hfil\break
\verb|http://hep.pa.msu.edu/belyaev/proj/dark_matter/isatools_public/|\,.

\subsection{IsaRED (H. Baer, C. Balazs and \underbar{A. Belyaev})} 

\verb|IsaRED| evaluates the neutralino relic density in the MSSM.  The
complete set of tree-level $\tz_1\tz_1$ annihilation and co-annihilation
processes is evaluated. Calculations are based on the matrix element
library created using the CompHEP program and interfaced to Isajet. The
following SUSY particles in the initial state are taken into account:
$\tz_1$, $\tz_2$, $\tw_1$, $\tilde{e_1}$, $\tilde{\mu_1}$,
$\tilde{\tau_1}$, $\tilde{\nu_e}$, $
\tilde{\nu_\mu}$, $\tilde{\nu_\tau}$, $\tilde{u_1}$, $\tilde{d_1}$, $
\tilde{c_1}$, $\tilde{s_1}$, $ \tilde{t_1}$, $\tilde{b_1}$, $\tilde{g}$.

The fully relativistic thermally averaged cross section times velocity
is computed using the Gondolo-Gelmini\cite{gg} and
Gondolo-Edsjo\cite{ge} formalism.  The freeze out temperature is solved
for iteratively, and then the relic density is computed.  To achieve our
final result with relativistic thermal averaging, we perform a
three-dimensional integral over {\it i.}) the final state subprocess
scattering angle, {\it ii.}) the subprocess energy parameter, and {\it
iii.}) the temperature from freeze-out to the present day temperature of
the universe.  We perform the three-dimensional integral using the BASES
algorithm\cite{bases}, which implements sequentially improved sampling
in multi-dimensional Monte Carlo integration, generally with good
convergence properties.  See Ref. \cite{bb1,bbb} for details.

\subsection{IsaBSG (H. Baer and \underbar{M. Brhlik})} 

This subroutine evaluates the $BF(b\to s\gamma)$ using the effective
field theory approach of Anlauf\cite{anlauf}, wherein the Wilson
co-efficients of various operators are calculated at the relevant scales
where various sparticles are integrated out of the theory.  This method
is used to handle the $tW$, $tH^-$ and $\tst_{1,2}\tw_{1,2}$ loops.  The
Wilson co-efficients are evolved to scale $Q=M_W$, wherein the effective
theory is taken to be the SM.  The SM Wilson co-efficients are evolved
to $Q=m_b$ using NLO anomalous dimension matrices. Once at scale
$Q=m_b$, the complete NLO corrections to the $O_2$, $O_7$ and $O_8$
operators are included\cite{greub}.  The scale dependence of the final
result is of order $10\%$.  In the high scale calculation, the weak
scale value of $m_b(M_{SUSY})$ is calculated using two loop RG evolution
with full one loop corrections to $m_b$. This effect is important at
large $\tan\beta$. Also, the $\tg\tq$ and $\tz_i\tq$ loops are included
directly at scale $Q=M_W$, according to the calculation of Masiero et
al.\cite{masiero}.  This necessitates an RG computation of well over 100
soft terms and couplings in order to generate the proper off diagonal
soft terms at scale $Q=M_W$.

\subsection{IsaAMU ({\underbar{H. Baer}} and C. Balazs)}

Supersymmetric contributions to $a_\mu\equiv (g-2)_\mu/2$ come from
$\tw_i\tnu_\mu$ and $\tz_i\tmu_{1,2}$ loops.  Complete formulae for
these contributions in the MSSM can be found in the article by T. Moroi,
Ref. \cite{moroi}. Numerical analyses based on his result are presented
in Ref. \cite{bbft}.

\subsection{IsaBMM (\underbar{J. K. Mizukoshi}, X. Tata and
Y. Wang)}

This subroutine evaluates the branching ratio of the decays
$B_s\rightarrow \mu^+\mu^-$ and $B_d\rightarrow \tau^+\tau^-$, according
to formulae given in Ref. \cite{mtw}:
\begin{eqnarray*}
B(B_{d'} \to \ell^+ \ell^-) &=& \frac{G_F^2\alpha^2 m^3_{B_{d'}} 
\tau_{B_{d'}} f^2_{B_{d'}}}{64\pi^3}
|V^\ast_{tb}V_{td'}|^2\sqrt{1-\frac{4m^2_\ell}{m^2_{B_{d'}}}}
\nonumber \\
&&\times\biggl[\biggl(1-\frac{4m^2_\ell}{m^2_{B_{d'}}}\biggr)
\bigg|\frac{m_{B_{d'}}}
{m_b+m_{d'}}c_{Q_1}\bigg|^2 + \bigg|\frac{2m_\ell}{m_{B_{d'}}}c_{10}
- \frac{m_{B_{d'}}}{m_b+m_{d'}} c_{Q_2}\bigg|^2\biggr],
\end{eqnarray*}
%
where $d'= s, d$ and $\ell = \mu, \tau$. The coefficients of the
effective Hamiltonian of the above processes,
\begin{eqnarray*}
c_{Q_1} &=& \frac{2\pi}{\alpha} \chi_{FC} \frac{m_b m_\ell}{\cos^2\beta
\sin^2\beta}\biggl(\frac{\cos(\beta+\alpha)\sin\alpha}{m^2_h} -
\frac{\sin(\beta+\alpha)\cos\alpha}{m^2_H}\biggr)\;, \nonumber
\\
c_{Q_2} &=& \frac{2\pi}{\alpha} \chi_{FC} \frac{m_b m_\ell}{\cos^2\beta}
\frac{1}{m^2_A}
\end{eqnarray*}
contain the  Higgs-mediated flavor changing  neutral currents that arise
as a consequence of coupling of Higgs superfield $\hat{h}_u$ with down
type fermions at one-loop level. Since this coupling is enhanced for 
large ${\langle h_u \rangle / \langle
h_d \rangle} \equiv \tan\beta$,  the calculations were  performed
keeping only  terms that are most  enhanced by powers of
$\tan\beta$. Therefore,  the $\chi_{FC}$ parameter given in
Ref. \cite{mtw} is  valid only for $\tan\beta \agt 25-30$.  Moreover,  
in the calculations of  one-loop $h, H b d'$,  and   
$A b d'$ vertex corrections, it has been assumed that the chargino masses 
are well approximated by  $|M_2|$ and $|\mu|$.

\subsection{IsaRES (C. Balazs, \underbar{A. Belyaev} and M. Brhlik)} 

\verb|IsaRES| evaluates the spin-independent and spin-dependent
neutralino-proton and neutralino-neutron scattering cross sections
according to formulae contained in Refs. \cite{bb2,bbbo}.


The interactions for elastic scattering of neutralinos on nuclei can be
described by the sum of spin-independent (${\cal L}^{eff}_{scalar}$) and 
spin-dependent (${\cal L}^{eff}_{spin}$) Lagrangian terms:
\begin{equation}
 {\cal L}^{eff}_{elastic}={\cal L}^{eff}_{scalar}+{\cal L}^{eff}_{spin} .
\end{equation}
$\sigma_{SI}$ for 
neutralino scattering off of nuclei  is the main experimental observable 
since $\sigma_{SI}$ contributions from individual nucleons in the nucleus add 
coherently and can be expressed via SI nuclear form-factors. The cross section 
$\sigma_{SI}$ receives contributions from neutralino-quark interactions via 
squark, $Z$ and Higgs boson exchanges, and from neutralino-gluon interactions 
involving quarks, squarks and Higgs bosons at the 1-loop level. The differential 
$\sigma_{SI}$ off a nucleus $X_Z^A$ with mass $m_A$ takes the 
form~\cite{kam_review}
\begin{equation}
 \frac{d\sigma^{SI}}{d|\vec{q}|^2}=\frac{1}{\pi v^2}[Z f_p +(A-Z) f_n]^2 
 F^2 (Q_r),
\end{equation}                                 
where $\vec{q}=\frac{m_A m_{\widetilde Z_1}}{m_A+m_{\widetilde Z_1}}\vec{v}$ is 
the three-momentum transfer, $Q_r=\frac{|\vec{q}|^2}{2m_A}$ and $F^2(Q_r)$ is 
the scalar nuclear form factor, $\vec{v}$ is the velocity of the incident 
neutralino and $f_p$ and $f_n$ are effective neutralino couplings to protons and 
neutrons respectively.
The original calculation has been done in~\cite{drees-nojiri} and can 
be expressed as
\begin{equation}
{f_N \over m_N} = \sum_{q=u,d,s} \frac{f_{Tq}^{(N)}}{m_q} \left[
f_q^{(\tq )}+f_q^{(H)}-{1\over 2}m_q m_{\tz_1}g_q\right]  + 
\frac{2}{27} f^{(N)}_{TG} \sum_{c,b,t} \frac{f_q^{(H)}}{m_q} +\cdots
\end{equation}
%
where $N=p,\ n$ for neutron, proton respectively, and
$f^{(N)}_{TG} = 1 - \sum_{q=u,d,s} f^{(N)}_{Tq} $.
The expressions for the $f_q^{(H)}$ couplings as well as other terms
denoted by $\cdots$ are omitted for the sake of brevity but can be found 
in~\cite{drees-nojiri,baer-brhlik}.

The parameters  $f_{Tq}^{(p)}$, defined by
\begin{equation}
<N|m_q \bar{q} q|N> = m_N f_{Tq}^{(N)} \qquad (q=u,d,s)
\end{equation}
contain uncertainties due to errors on the experimental measurements of quark masses.
We have adopted values of renormalization-invariant  constants $f_{Tq}^{(p)}$
and their uncertainties
determined in~\cite{dirdet-ellis}:
\begin{equation}
f_{Tu}^{(p)} = 0.020 \pm 0.004, \qquad f_{Td}^{(p)} = 0.026 \pm 0.005,
\qquad f_{Ts}^{(p)} = 0.118 \pm 0.062
\end{equation}
\begin{equation}
f_{Tu}^{(n)} = 0.014 \pm 0.003, \qquad f_{Td}^{(n)} = 0.036 \pm 0.008,
\qquad f_{Ts}^{(n)} = 0.118 \pm 0.062.
\end{equation}

The cross section $\sigma_p^{SI}$
for neutralino scattering off the proton is calculated in the limit of zero momentum transfer
\begin{equation}
\sigma^{SI}=\frac{4}{\pi}{m_r^N}^2 f_N^2
\end{equation}
where $m_r^N=m_N m_{\tilde{Z_1}}/(m_N+m_{\tilde{Z_1}})$.

In our calculations we are using the CTEQ5L set of parton density 
functions~\cite{CTEQ5}
evaluated at the QCD scale $Q=\sqrt{M_{SUSY}^2-m_{\tilde{Z}_1}^2}$.
The PDF  parton density 
function can be easily updated to any other PDF upon request.

\subsection{Compiling and Using IsaTools (\underbar{F. Paige})}

\verb|IsaTools| is so far interfaced only to ISASUGRA, not to ISAJET or
ISASUSY. Since \verb|IsaRED| contains about 1.4M lines of Fortran code
generated by CompHEP, both compilation and execution are somewhat slow. 
It was therefore decided to make \verb|IsaTools| optional and to keep
keep this code in a separate file, \verb|isared.tar|.

If you {\it do not} want to use \verb|IsaTools|, edit the
\verb|Makefile| and select
\begin{verbatim}
ISATOOLS =
LIBTOOLS =
\end{verbatim}
Then build ISAJET, ISASUSY, and ISASUGRA as described previously. The
variable \verb|ISATOOLS| serves as a Patchy flag, and the blank value
turns off the calls to \verb|IsaTools| in ISASUGRA.

If you {\it do} want to use \verb|IsaTools|, edit the \verb|Makefile|
and select
\begin{verbatim}
ISATOOLS = ISATOOLS
LIBTOOLS = -lisared
\end{verbatim}
This is the default, and it requires \verb|isared.tar| to be in the same
directory as \verb|isajet.car| and \verb|Makefile|. First build
\verb|libisared.a| for \verb|IsaTools| and then build the rest of
ISAJET, ISASUSY, and ISASUGRA with the commands
\begin{verbatim}
make isatools
make
\end{verbatim}
The outputs from \verb|IsaTools| will appear after the mass spectrum and
before the list of decay modes.

\def\refname{\large\bf References for IsaTools}
\begin{thebibliography}{99}
%
\bibitem{gg} G.\ Gelmini and P.\ Gondolo, Nucl.\ Phys.\ {\bf B351}, 623 (1991).
%
\bibitem{ge} J.\ Edsj\"o and P.\ Gondolo, Phys.\ Rev.\ {\bf D56}, 1879 (1997).\ 
%
\bibitem{bases} S.~Kawabata,
``Monte Carlo Integration Packages Bases And Dice,''
{\it Prepared for 2nd International Workshop on Software Engineering, 
Artificial Intelligence and Expert Systems for High-energy and Nuclear
Physics, La Londe Les Maures, France, 13-18 Jan 1992}.

\bibitem{bb1} H.\ Baer and M.\ Brhlik, Phys.\ Rev.\ {\bf D53}, 597 (1996).
%
\bibitem{bbb} H.\ Baer, C.\ Balazs and A.\ Belyaev,
JHEP{\bf 0203}, 042 (2002) and hep-ph/0211213 (2002).\ 
%
\bibitem{bb_bsg,bbct} H.\ Baer and M.\ Brhlik, Phys.\ Rev.\ {\bf D55}, 3201 (1997);
H.\ Baer, M.\ Brhlik, D.\ Castano and X.\ Tata, 
Phys.\ Rev.\ {\bf D58}, 015007 (1998).
%
\bibitem{anlauf} H.\ Anlauf, Nucl.\ Phys.\ {\bf B430}, 245 (1994).
%
\bibitem{greub} C.\ Greub, T.\ Hurth and Wyler, 
Phys.\ Rev.\ {\bf D54}, 3350 (1996).
%
\bibitem{masiero} S.\ Bertolini, F.\ Borzumati, A.\ Masiero and G.\ Ridolfi, 
Nucl.\ Phys.\ {\bf B353}, 591 (1991).
%
\bibitem{moroi} T.\ Moroi, Phys.\ Rev.\ {\bf D53}, 6565 (1996),
Erratum-ibid.{\bf D56}, 4424 (1997).
%
\bibitem{bbft} H.\ Baer, C.\ Balazs, J.\ Ferrandis and X.\ Tata,
Phys.\ Rev.\ {\bf D64}, 035004 (2001).
%
\bibitem{mtw} J.\ K.\ Mizukoshi, X.\ Tata and Y.\ Wang, 
Phys.\ Rev.\ {\bf D66}, 115003 (2002).
%
\bibitem{bb2} H.\ Baer and M.\ Brhlik, Phys.\ Rev.\ {\bf D57}, 567 (1998).
%
\bibitem{bbbo} H.\ Baer, C.\ Balazs, A.\ Belyaev and J.\ O'Farrill, 
JCAP{\bf 0309}, 007 (2003).

\bibitem{kam_review}
G.~Jungman, M.~Kamionkowski and K.~Griest,
%``Supersymmetric dark matter,''
Phys.\ Rept.\  {\bf 267}, 195 (1996)
[arXiv:hep-ph/9506380].

\bibitem{drees-nojiri}
M.~Drees and M.~Nojiri,
%``Neutralino - nucleon scattering revisited,''
Phys.\ Rev.\ D {\bf 47}, 4226 (1993) and
Phys.\ Rev.\ D {\bf 48}, 3483 (1993)
[arXiv:hep-ph/9307208];

\bibitem{baer-brhlik}
H.~Baer and M.~Brhlik,
%``Neutralino dark matter in minimal supergravity: Direct detection vs.\  collider searches,''
Phys.\ Rev.\ D {\bf 57}, 567 (1998)
[arXiv:hep-ph/9706509].
%%CITATION = HEP-PH 9706509;%%

\bibitem{dirdet-ellis}
J.~R.~Ellis, A.~Ferstl and K.~A.~Olive,
%``Re-evaluation of the elastic scattering of supersymmetric dark matter,''
Phys.\ Lett.\ B {\bf 481}, 304 (2000)
[arXiv:hep-ph/0001005];
%%CITATION = HEP-PH 0001005;%%

%\cite{Lai:1999wy}
\bibitem{CTEQ5}
H.~L.~Lai {\it et al.}  [CTEQ Collaboration],
%``Global {QCD} analysis of parton structure of the nucleon: CTEQ5 parton  distributions,''
Eur.\ Phys.\ J.\ C {\bf 12}, 375 (2000)
[arXiv:hep-ph/9903282].
%%CITATION = HEP-PH 9903282;%%

\end{thebibliography}

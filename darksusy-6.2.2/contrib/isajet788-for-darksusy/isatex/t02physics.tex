\newpage
\section{Physics\label{PHYSICS}}

      ISAJET is a Monte Carlo program which simulates $pp$, $\bar pp$
and $e^+e^-$ interactions at high energy. 
The program incorporates
perturbative QCD cross sections, initial state and final state QCD
radiative corrections in the leading log approximation, independent
fragmentation of quarks and gluons into hadrons, and a
phenomenological model tuned to minimum bias and hard scattering data
for the beam jets.

\subsection{Hard Scattering\label{hard}}

      The first step in simulating an event is to generate a primary
hard scattering according to some QCD cross section. This has the
general form
$$
\sigma = \sigma_0  F(x_1,Q^2) F(x_2,Q^2)
$$
where $\sigma_0$ is a cross section calculated in QCD perturbation
theory, $F(x,Q^2)$ is a structure function incorporating QCD scaling
violations, $x_1$ and $x_2$ are the usual parton model momentum
fractions, and $Q^2$ is an appropriate momentum transfer scale.

      For each of the processes included in ISAJET, the basic cross
section $\sigma_0$ is a two-body one, and the user can set limits on
the kinematic variables and type for each of the two primary jets. For
DRELLYAN and WPAIR events the full matrix element for the decay of the
W's into leptons or quarks is also included.

      The following processes are available:

\subsubsection{Minbias} No hard scattering at all, so that the event
consists only of beam jets. Note that at high energy the jet cross
sections become large. To represent the total cross section it is
better to use a sample of TWOJET events with the lower limit on pt
chosen to give a cross section equal to the inelastic cross section or
to use a mixture of MINBIAS and TWOJET events.

\subsubsection{Twojet} All order $\alpha_s^2$ QCD processes, which
give rise in lowest order to two high-$p_t$ jets. Included are, e.g.
\begin{eqnarray*}
g + g &\to& g + g\\
g + q &\to& g + q \\
g + g &\to& q + \bar q
\end{eqnarray*}
Masses are neglected for $c$ and lighter quarks but are taken into
account for $b$ and $t$ quarks. The $Q^2$ scale is taken to be
$$
Q^2 = 2stu/(s^2+t^2+u^2)
$$
The default parton distributions are those of the CTEQ Collaboration,
fit CTEQ5L, using lowest order QCD evolution. A few older sets of parton
distributions are included. There is also an interface to the CERN
PDFLIB compilation of parton distributions. Note that structure
functions for heavy quarks are included, so that processes like
$$
g + t \to g + t
$$
can be generated. The Duke-Owens parton distributions do not contain b
or t quarks.

      Since the $t$ is so heavy, it decays before it can hadronize, so
instead of $t$ hadrons a $t$ quark appears in the particle list. It is
decayed using the $V-A$ matrix element including the $W$ propagator
with a nonzero width, so the same decays should be used for $m_t < m_W$
and $m_t > m_W$; the $W$ should {\it not} be listed as part of the decay
mode.  The partons are then evolved and fragmented as usual; see
below. The real or virtual $W$ and the final partons from the decay,
including any radiated gluons, are listed in the particle table,
followed by their fragmentation products.  Note that for semileptonic
decays the leptons appear twice: the lepton parton decays into a
single particle of the same type but in general somewhat different
momentum. In all cases only particles with $\verb|IDCAY| = 0$ should be
included in the final state.

      A fourth generation $x,y$ is also allowed. Fourth generation
quarks are produced only by gluon fusion. Decay modes are not included
in the decay table; for a sequential fourth generation they would be
very similar to the t decays. In decays involving quarks, it is
essential that the quarks appear last.

\subsubsection{Drellyan} Production of a $W$ in the standard model,
including a virtual $\gamma$, a $W^+$, a $W^-$, or a $Z^0$, and its
decay into quarks or leptons. If the transverse momentum QTW of the
$W$ is fixed equal to zero then the process simulated is
\begin{eqnarray*}
q + \bar q \to W &\to& q + \bar q \\
                 &\to& \ell + \bar\ell
\end{eqnarray*}
Thus the $W$ has zero transverse momentum until initial state QCD
corrections are taken into account. If non-zero limits on the
transverse momentum $q_t$ for the $W$ are set, then instead the
processes
\begin{eqnarray*}
q + \bar q &\to& W + g \\
g + q      &\to& W + q
\end{eqnarray*}
are simulated, including the full matrix element for the $W$ decay.
These are the dominant processes at high $q_t$, but they are of course
singular at $q_t=0$. A cutoff of the $1/q_t^2$ singularity is made by
the replacement
$$
1/q_t^2 \to 1/\sqrt{q_t^4+q_{t0}^4} \quad q_{t0}^2 =  (.2\,\GeV) M
$$
This cutoff is chosen to reproduce approximately the $q_t$ dependence
calculated by the summation of soft gluons and to give about the right
integrated cross section. Thus this option can be used for low as well
as high transverse momenta.

      The scale for QCD evolution is taken to be proportional to the
mass for lowest order Drell-Yan and to the transverse momentum for
high-$p_t$ Drell-Yan. The constant is adjusted to get reasonable
agreement with the $W + n\,{\rm jet}$ cross sections calculated from
the full QCD matrix elements by F.A. Berends, et al., Phys.\ 
Lett.\ B224, 237 (1989).

      For the processes $g + b \to W + t$ and $g + t \to Z + t$, cross
sections with a non-zero top mass are used for the production and the
$W/Z$ decay. These were calculated using FORM 1.1 by J.~Vermaseren. The
process $g + t \to W + b$ is {\it not} included. Both $g + b \to W^- +
t$ and $g + \bar t \to W^- + \bar b$ of course give the same $W^- + t
+ \bar b$ final state after QCD evolution. While the latter process is
needed to describe the $m_t = 0$(!) mass singularity for $q_t \gg
m_t$, it has a pole in the physical region at low $q_t$ from on-shell
$t \to W + b$ decays. There is no obvious way to avoid this without
introducing an arbitrary cutoff.  Hence, selecting only $W + b$ will
produce a zero cross section. The $Q^2$ scale for the parton
distributions in these processes is replaced by $Q^2 + m_t^2$; this
seems physically sensible and prevents the cross sections from
vanishing at small $q_t$.

\subsubsection{Photon} Single and double photon production through the
lowest order QCD processes
\begin{eqnarray*}
g + q &\to& \gamma + q \\
q + \bar q &\to& \gamma + g \\
q + \bar q &\to& \gamma + \gamma
\end{eqnarray*}
Higher order corrections are not included. But $\gamma$'s, $W$'s, and
$Z$'s are radiated from final state quarks in all processes, allowing
study of the bremsstrahlung contributions.

\subsubsection{Wpair} Production of pairs of W bosons in the standard
model through quark-antiquark annihilation,
\begin{eqnarray*}
q + \bar q &\to& W^+ + W^- \\
           &\to& Z^0 + Z^0 \\
           &\to& W^+ + Z^0, W^- + Z^0 \\
           &\to& W^+ + \gamma, W^- + \gamma \\
           &\to& Z^0 + \gamma
\end{eqnarray*}
The full matrix element for the W decays, calculated in the narrow
resonance approximation, is included. However, the higher order
processes, e.g.
$$
q + q \to q + q + W^+ + W^-
$$
are ignored, although they in fact dominate at high enough mass.
Specific decay modes can be selected using the WMODEi keywords.

\subsubsection{Higgs} Production and decay of the standard model Higgs
boson. The production processes are
\begin{eqnarray*}
g + g      &\to& H \quad\hbox{(through a quark loop)} \\
q + \bar q &\to& H \quad\hbox{(with $t + \bar t$ dominant)} \\
W^+ + W^-  &\to& H \quad\hbox{  (with longitudinally polarized $W$)} \\
Z^0 + Z^0  &\to& H \quad\hbox{ (with longitudinally polarized $Z$)}
\end{eqnarray*}
If the (Standard Model) Higgs is lighter than $2 M_W$, then it will
decay into pairs of fermions with branching ratios proportional to
$m_f^2$. If it is heavier than $2 M_W$, then it will decay primarily
into $W^+ W^-$ and $Z^0 Z^0$ pairs with widths given approximately by
\begin{eqnarray*}
\Gamma(H \to W^+ W^-) &=& {G_F M_H^3 \over 8 \pi \sqrt{2} } \\
\Gamma(H \to Z^0 Z^0) &=& {G_F M_H^3 \over 16 \pi \sqrt{2} }
\end{eqnarray*}
Numerically these give approximately
$$
\Gamma_H = 0.5\,{\rm TeV} \left({M_H \over 1\,{\rm TeV}}\right)^3
$$
The width proportional to $M_H^3$ arises from decays into longitudinal
gauge bosons, which like Higgs bosons have couplings proportional to
mass.

      Since a heavy Higgs is wide, the narrow resonance approximation is
not valid. To obtain a cross section with good high energy behavior, it
is necessary to include a complete gauge-invariant set of graphs for the
processes
\begin{eqnarray*}
W^+ W^- &\to& W^+ W^- \\
W^+ W^- &\to& Z^0 Z^0 \\
Z^0 Z^0 &\to& W^+ W^- \\
Z^0 Z^0 &\to& Z^0 Z^0
\end{eqnarray*}
with longitudinally polarized $W^+$, $W^-$, and $Z^0$ bosons in the
initial state. This set of graphs and the corresponding angular
distributions for the $W^+$, $W^-$, and $Z^0$ decays have been
calculated in the effective $W$ approximation and included in HIGGS.
The $W$ structure functions are obtained by integrating the EHLQ
parameterization of the quark ones term by term. The Cabibbo-allowed
branchings
\begin{eqnarray*}
q &\to& W^+ + q' \\
q &\to& W^- + q' \\
q &\to& Z^0 + q
\end{eqnarray*}
are generated by backwards evolution, and the standard QCD evolution is
performed. This correctly describes the $W$ collinear singularity and
so contains the same physics as the effective $W$ approximation.

      If the Higgs is lighter than $2M_W$, then its decay to
$\gamma\gamma$ through $W$ and $t$ loops may be important. This is
also included in the HIGGS process and may be selected by choosing
\verb|GM| as the jet type for the decay.

      If the Higgs has $M_Z < M_H < 2M_Z$, then decays into one real
and one virtual $Z^0$ are generated if the \verb|Z0 Z0| decay mode is
selected, using the calculation of Keung and Marciano, Phys.\ Rev.\
D30, 248 (1984). Since the calculation assumes that one $Z^0$ is
exactly on shell, it is not reliable within of order the $Z^0$ width
of $M_H = 2M_Z$; Higgs and and $Z^0 Z^0$ masses in this region should
be avoided. The analogous Higgs decays into one real and one virtual
charged W are not included.

      Note that while HIGGS contains the dominant graphs for Higgs
production and graphs for $W$ pair production related by gauge invariance,
it does not contain the processes
\begin{eqnarray*}
q + \bar q &\to& W^+ W^- \\
q + \bar q &\to& Z^0 Z^0
\end{eqnarray*}
which give primarily transverse gauge bosons. These must be generated
with WPAIR.

      If the \verb|MSSMi| or \verb|SUGRA| keywords are used with
HIGGS, then one of the three MSSM neutral Higgs is generated instead
using gluon-gluon and quark-antiquark fusion with the appropriate SUSY
couplings. Since heavy CP even SUSY Higgs are weakly coupled to W
pairs and CP odd ones are completely decoupled, $WW$ fusion and $WW
\to WW$ scattering are not included in the SUSY case. ($WW \to WW$ can
be generated using the Standard Model process with a light Higgs mass,
say 100 GeV.) The MSSM Higgs decays into both Standard Model and SUSY
modes as calculated by ISASUSY are included. For more discussion see
the SUSY subsection below and the writeup for ISASUSY. The user must
select which Higgs to generate using HTYPE; see Section 6 below. If a
mass range is not specified, then the range mass $M_H \pm 5\Gamma_H$
is used by default. (This cannot be done for the Standard Model Higgs
because it is so wide for large masses.) Decay modes may be selected
in the usual way.

\subsubsection{WHiggs} Generates associated production of gauge and
Higgs bosons, i.e.,
$$
q + \bar q \to H + W, H + Z\,,
$$
in the narrow resonance approximation. The desired subprocesses can be
selected with JETTYPEi, and specific decay modes of the $W$ and/or $Z$
can be selected using the WMODEi keywords. Standard Model couplings are
assumed unless SUSY parameters are specified, in which case the SUSY
couplings are used.

\subsubsection{SUSY} Generates pairs of supersymmetric particles from
gluon-quark or quark-antiquark fusion. If the MSSMi or SUGRA parameters
defined in Section 6 below are not specified, then only gluinos and
squarks are generated:
\begin{eqnarray*}
g + g      &\to& \tilde g + \tilde g \\
q + \bar q &\to& \tilde g + \tilde g \\
g + q      &\to& \tilde g + \tilde q \\
g + g      &\to& \tilde q + \tilde{\bar q} \\
q + \bar q &\to& \tilde q + \tilde{\bar q} \\
q + q      &\to& \tilde q + \tilde q
\end{eqnarray*}
Left and right squarks are distinguished but assumed to be degenerate.
Masses can be specified using the \verb|GAUGINO|, \verb|SQUARK|, and
\verb|SLEPTON| parameters described in Section 6. No decay modes are
specified, since these depend strongly on the masses. The user can
either add new modes to the decay table (see Section 9) or use the
\verb|FORCE| or \verb|FORCE1| commands (see Section 6).

      If \verb|MSSMA|, \verb|MSSMB|, and \verb|MSSMC| are specified,
then the ISASUSY package is used to calculate the masses and decay
modes in the minimal supersymmetric extension of the standard model
(MSSM), assuming SUSY grand unification constraints in the neutralino
and chargino mass matrix but allowing some additional flexibility in
the masses. The scalar particle soft masses are input via
\verb|MSSMi|, so that the physical masses will be somewhat different
due to $D$-term contributions and mixings for 3rd generation sparticles.
$\tilde t_1$ and $\tilde t_2$ production and decays are now included.
The lightest SUSY particle is assumed to be the lightest neutralino
$\tilde Z_1$. If the \verb|MSSMi| parameters are specified, then the
following additional processes are included using the MSSM couplings
for the production cross sections:
\begin{eqnarray*}
g + q    &\to& \tilde Z_i + \tilde q, \quad \tilde W_i + \tilde q \\
q + \bar q &\to& \tilde Z_i + \tilde g, \quad \tilde W_i + \tilde g \\
q + \bar q &\to& \tilde W_i + \tilde Z_j \\
q + \bar q &\to& \tilde W_i^+ + \tilde W_j^- \\
q + \bar q &\to& \tilde Z_i + \tilde Z_j \\
q + \bar q &\to& \tilde\ell^+ + \tilde\ell^-, \quad \tilde\nu + \tilde\nu
\end{eqnarray*}
Processes can be selected using the optional parameters described in
Section 6 below.

      Beginning with Version 7.42, matrix elements are taken into
account in the event generator as well as in the calculation of decay
widths for MSSM three-body decays of the form $\tilde A \to \tilde B f
\bar f$, where $\tilde A$ and $\tilde B$ are gluinos, charginos, or
neutralinos. This is implemented by having ISASUSY save the poles and
their couplings when calculating the decay width and then using these
to reconstruct the matrix element. Other three-body decays may be
included in the future. Decays selected with \verb|FORCE| use the
appropriate matrix elements.

      An optional keyword \verb|MSSMD| can be used to specify the second
generation masses, which otherwise are assumed degenerate with the first
generation. An optional keyword \verb|MSSME| can be used to specify
values of the $U(1)$ and $SU(2)$ gaugino masses at the weak scale rather
than using the default grand unification values. The chargino and
neutralino masses and mixings are then computed using these values.

\subsubsection{SUSY Models} The 24 MSSMi parameters describe the MSSM at
the weak scale with the additional assumptions of exact flavor and $CP$
conservation; the general MSSM has 105 parameters. These weak-scale
SUSY-breaking parameters presumably arise from spontaneous SUSY breaking
in a hidden sector that is communicated to the MSSM at some scale $M \gg
M_Z$. There are a number of plausible models in which this symmetry
breaking is simple, so that the MSSM at the high scale involves only a
small number of parameters. These are then related to those at the weak
scale by the renormalization group equations (RGE's).

      Isajet therefore implements in subroutine SUGRA the complete
2-loop RGE's for the gauge couplings, Yukawa couplings, and soft
breaking terms. Contributions from right-handed neutrinos are
optionally included. The RGE's are solved iteratively, running from
the weak scale to the high scale $M$ and back using Runge-Kutta
integration. After each iteration the SUSY masses are recalculated,
and the renormalization group improved 1-loop corrected 
Higgs potential is calculated and minimized. These results are
used to modify the RGE $\beta$-functions appropriately as each
threshold is crossed during the next iteration. Beginning with Version
7.65, the constant parts as well as the logarithms from these
thresholds are included using the results of Pierce {\it et al.},
Nucl.\ Phys.\ {\bf B491}, 3 (1997). The whole process is repeated,
increasing the number of Runge-Kutta steps by a factor of 1.2 for each
iteration, until all the RGE variables except $\mu$ and $B$ differ by
less than 0.3\%. Since $\mu$ and $B$ vary rapidly near the weak scale,
they are only required to differ by less than 5\%. The requirement of
good electroweak symmetry breaking, $\mu^2>0$, is only imposed after
the iterative solution has converged.

\medskip

      A number of different models for SUSY breaking at the high scale
are included in ISAJET. The \verb|SUGRA| parameters must be
specified for the
minimal supergravity framework. This assumes that the gauge couplings
unify at the GUT scale, $M \sim 10^{16}\,{\rm GeV}$, defined by
$\alpha_2=\alpha_1$. SUSY breaking occurs at that scale with universal
soft breaking terms produced by gravitational interactions with a hidden
sector. The parameters of the model are
\begin{itemize}
\item $m_0$: the common scalar mass at the GUT scale;
\item $m_{1/2}$: the common gaugino mass at the GUT scale;
\item $A_0$: the common soft trilinear SUSY breaking parameter at the
GUT scale;
\item $\tan\beta$: the ratio of Higgs vacuum expectation values at the
electroweak scale;
\item $\sgn\mu=\pm1$: the sign of the Higgsino mass term.
\end{itemize}
An attractive feature of this model is that the Higgs are unified with
the other scalars at the GUT scale but $m_{H_u}^2$ is driven negative by
the large top Yukawa coupling $f_t$. Isajet imposes this radiative
symmetry breaking for the SUGRA model but not other possible constraints
such as $b$-$\tau$ unification or limits on proton decay.

      The SUGRA model with exact compling constant unification produces
too large a value of $\alpha_s$ at the weak scale. The default is to use
the experimental value, assuming that threshold effects at the GUT scale
produce this. Exact unification can also be imposed.

      The assumption of universality at the GUT scale is rather
restrictive and may not be valid. A variety of non-universal SUGRA
(NUSUGRA) models can be generated using the \verb|NUSUG1|, \dots,
\verb|NUSUG5| keywords. These might be used to study how well one could
test the minimal SUGRA model. The keyword \verb|SSBCSC| can be used to
specify an alternative scale (i.e., not the coupling constant
unification scale) for the RGE boundary conditions.
A SUGRA model with non-universal Higgs masses $m_{H_u}$ and $m_{H_d}$
which are determined via input of weak scale parameters $\mu$
and $m_A$ can be input using the \verb|NUHM| keyword.

\medskip

      An alternative to the SUGRA model is the Gauge Mediated SUSY
Breaking (GMSB) model of Dine, Nelson, and collaborators. In this model
SUSY breaking is communicated through gauge interactions with messenger
fields at a scale $M_m$ small compared to the Planck scale and are
proportional to gauge couplings times $\Lambda_m$. Since $M_m$ is small
and the masses at it are the same for each generation, there are no
flavor changing neutral currents. The messenger fields should form
complete $SU(5)$ representations to preserve the unification of the
coupling constants. The parameters of the GMSB model, which are
specified by the \verb|GMSB| keyword, are
\begin{itemize}
\item $\Lambda_m = F_m/M_m$: the scale of SUSY breaking, typically
10--$100\,{\rm TeV}$;
\item $M_m > \Lambda_m$: the messenger mass scale; 
\item $N_5$: the equivalent number of $5+\bar5$ messenger fields.
\item $\tan\beta$: the ratio of Higgs vacuum expectation values at the
electroweak scale;
\item $\sgn\mu=\pm1$: the sign of the Higgsino mass term;
\item $C_{\rm grav}\ge1$: the ratio of the gravitino mass to the value it
would have had if the only SUSY breaking scale were $F_m$.
\end{itemize}
In GMSB models the lightest SUSY particle is always the nearly massless
gravitino $\tilde G$. The parameter $C_{\rm grav}$ scales the gravitino
mass and hence the lifetime of the next lightest SUSY particle to decay
into it. The \verb|NOGRAV| keyword can be used to turn off gravitino
decays. 

      A variety of non-minimal GMSB models can be generated using
additional parameters set with the GMSB2 keyword. These additional
parameters are
\begin{itemize}
\item $\slashchar{R}$, an extra factor multiplying the gaugino masses
at the messenger scale. (Models with multiple spurions generally have
$\slashchar{R}<1$.)
\item $\delta M_{H_d}^2$, $\delta M_{H_u}^2$, Higgs mass-squared
shifts relative to the minimal model at the messenger scale. (These
might be expected in models which generate $\mu$ realistically.)
\item $D_Y(M)$, a $U(1)_Y$ messenger scale mass-squared term
($D$-term) proportional to the hypercharge $Y$.
\item $N_{5_1}$, $N_{5_2}$, and $N_{5_3}$, independent numbers of
gauge group messengers. They can be non-integer in general.
\end{itemize}
For discussions of these additional parameters, see S. Dimopoulos, S.
Thomas, and J.D. Wells, hep-ph/9609434, Nucl.\ Phys.\ {\bf B488}, 39
(1997), and S.P. Martin, hep-ph/9608224, Phys.\ Rev.\ {\bf D55}, 3177
(1997).

      Gravitino decays can be included in the general MSSM framework by
specifying a gravitino mass with \verb|MGVTNO|. The default is that such
decays do not occur.

\medskip

     Another alternative SUSY model choice allowed is anomaly-mediated
SUSY breaking, developed by Randall and Sundrum.  In this model, it is
assumed that SUSY breaking takes place in other dimensions, and SUSY
breaking is communicated to the visible sector via the superconformal
anomaly. In this model, the lightest SUSY particle is usually the
neutralino which is nearly pure wino-like. The chargino is nearly mass
degenerate with the lightest neutralino. It can be very long lived, or
decay into a very soft pion plus missing energy.  The model incorporated
in ISAJET, based on work by Ghergetta, Giudice and Wells
(hep-ph/9904378), and by Feng and Moroi (hep-ph/9907319) adds a
universal contribution $m_0^2$ to all scalar masses to avoid problems
with tachyonic scalars. The parameters of the model, which can be set
via the \verb|AMSB| keyword, are
\begin{itemize}
\item	$m_0$: Common scalar mass;
\item 	$m_{3/2}$: Gravitino mass, typically $\simge 10\,{\rm TeV}$ since 
$m_i = (\beta_i/g_i) m_{3/2}$.
\item	$\tan\beta$: Usual ratio of vev's at weak scale;
\item	$\sgn\mu$: Usual sign of $\mu$, $\pm1$.
\end{itemize}
Care should be taken with the chargino decay, since it may have
macroscopic decay lengths, or even decay outside the detector.
A variety of non-minimal AMSB models can be generated by using the
AMSB2 keyword, which allows input of $c_f$ multipliers of 
the $m_0^2$ contribution to 
sfermion masses: $m_{\tilde f}^2=m_{\tilde f}^2(AMSB)+c_f m_0^2$, 
for $f=Q,D,U,L,E,H_d$ and $H_u$.

The mixed modulus-AMSB model, inspired by the KKLT string model of
compactification of type IIB strings with fluxes, 
is also available by stipulating the 
\verb|MMAMSB| keyword. Inputs consist of the mixing parameter $\alpha$,
$m_{3/2}$, $\tan\beta$ and $sign(\mu )$. Also, the modular weights
$n_Q$, $n_D$, $n_U$, $n_L$, $n_E$, $n_{H_d}$, $n_{H_u}$ must be specified,
as well as moduli powers $\ell_1$, $\ell_2$ and $\ell_3$ in the 
gauge kinetic function. These latter quantities are usually all 
taken equal to 1 for gauge fields on a D7 brane. The matter and Higgs field
modular weights can be $0,\ 1$ or $1/2$ depending on whether the 
fields live on a D7 or D3 brane, or their intersection, respectively.
The mixing parameter $\sim -20<\alpha < \sim 20$ while $m_{3/2}:2-50$ TeV.
See hep-ph/0604253 for more information.

A more {\it generalized mirage mediation} model based on arXiv:1610.06205
is also available via the \verb|GNMIRAGE| keyword. This model requires
inputs $\alpha ,m_{3/2},c_m,c_{m3},a_3,\tan\beta ,sgn( \mu ),c_{H_u},c_{H_d}$ 
inputs. Alternatively, the $c_{H_u},c_{H_d}$ inputs can be overwritten 
in lieu of $\mu ,m_A$ inputs by using \verb|NUHM| keyword.

Also, a {\it generalized anomaly-mediation} model which allows for
naturalness in AMSB has been entered in Isajet 7.88 (with keyword
input NAMSB). 
This model allows independent bulk soft term contributions to Higgs multiplets
along with bulk $A_0$ terms (as originally envisioned by 
Randall and Sundrum). Inclusion of these terms allows for $m_h\sim 125$ GeV and
natural AMSB spectra, albeit with light higgsinos as LSP instead of light winos.
The parameter space is given as $m_0(1,2),m_0(3),m_{3/2},A_0,\tan\beta ,\mu, m_A$.

\medskip

     Since neutrinos seem to have mass, the effect of a massive
right-handed neutrino has been included in ISAJET, when calculating the
sparticle mass spectrum. If the keyword \verb|SUGRHN| is used, then the
user must input the 3rd generation neutrino mass (at scale $M_Z$) in
units of GeV, and the intermediate scale right handed neutrino Majorana
mass $M_N$, also in GeV. In addition, one must specify the soft
SUSY-breaking masses $A_n$ and $m_{\tilde\nu_R}$ valid at the GUT scale.
Then the neutrino Yukawa coupling is computed in the simple see-saw
model, and renormalization group evolution includes these effects
between $M_{GUT}$ and $M_N$. Finally, to facilitate modeling of $SO(10)$
SUSY-GUT models, loop corrections to 3rd generation fermion masses have
been included in the ISAJET SUSY models.

\bigskip\bigskip

      The ISASUSY program can also be used independently of the rest of
ISAJET, either to produce a listing of decays or in conjunction with
another event generator. Its physics assumptions are described in more
detail in Section~\ref{SUSY}. ISASUSY accepts soft SUSY breaking
parameters at the weak scale and calculates the masses and decay modes
from them. The ISASUGRA program can also be used independently to solve
the renormalization group equations with SUGRA, NUSUGRA, GMSB, or AMSB
boundary conditions and then to call ISASUSY to calculate the decay
modes. The main programs described in Section~\ref{sugrun} prompt for
interactive input and print the results to a file.

      Generally the MSSM, SUGRA, or GMSB option should be used to study
supersymmetry signatures; the SUGRA or GMSB parameter space is clearly
more manageable. The more general option may be useful to study
alternative SUSY models. It can also be used, e.g., to generate
pointlike color-3 leptoquarks in technicolor models by selecting squark
production and setting the gluino mass to be very large. The MSSM or
SUGRA option may also be used with top pair production to simulate top
decays to SUSY particles.

\subsubsection{$e^+e^-$} An $e^+e^-$ event generator is also included in
ISAJET. The
Standard Model processes included are $e^+e^-$ annihilation through
$\gamma$ and $Z$ to quarks and leptons, and production of $W^+W^-$ and
$Z^0Z^0$ pairs. In contrast to WPAIR and HIGGS for the hadronic
processes, the produced $W$'s and $Z$'s are treated as particles, so
their spins are not properly taken into account in their decays.
(Because the $W$'s and $Z$'s are treated as particles, their decay
modes can be selected using \verb|FORCE| or \verb|FORCE1|, not
\verb|WMODEi|. See Section [6] below.)  Other Standard Model
processes, including $e^+ e^- \to e^+ e^-$ ($t$-channel graph) 
are not included.  Once the primary reaction has been
generated, QCD radiation and hadronization are done as for hadronic
processes. 

The $e^+e^-$ generator can be run assuming no initial state
radiation (the default), or an initial state electron structure function
can be used for bremsstrahlung or the combination bremsstrahlung/beamstrahlung
effect. Bremsstrahlung is implemented using the Fadin-Kuraev
$e^-$ distribution function, and can be turned on using the \verb|EEBREM|
command while stipulating the minimal and maximal subprocess energy.
Beamstrahlung is implemented by invoking the \verb|EEBEAM| keyword.
In this case, in addition the beamstrahlung parameter $\Upsilon$ and
longitudinal beam size $\sigma_z$ (in mm) must be given.
The definition for $\Upsilon$ in terms of other beam parameters can be 
found in the article Phys. Rev. D49, 3209 (1994) by Chen, Barklow and Peskin.
The bremsstrahlung structure function is then convoluted with the 
beamstrahlung distribution (as calculated by M. Peskin) and a spline fit
is created. Since the cross section can contain large spikes, event generation
can be slow if a huge range of subprocess energy is selected for light 
particles; in these scenarios, \verb|NTRIES| must be increased well beyond
the default value.
In Isajet 7.70 and beyond, $e^+e^-\to\gamma\gamma\to f\bar{f}$ 
($f$ is a SM fermion) processes are included, via Peskin's 
photon structure function from brem- and beamstrahlung. These
gamma-gamma induced processes are activated by stipulating the keyword
\verb|GAMGAM| to be \verb|.TRUE.|, when running with \verb|EEBEAM|.
Since the photon structure function is so highly peaked at low $x$, 
it is wise to use \verb|GAMGAM| with only one subprocess at a time, 
a large number for \verb|NTRIES|, and to
use a judicious range of subprocess CM energies in \verb|EEBEAM|.

      $e^+e^-$ annihilation to SUSY particles is included as well with
complete lowest order diagrams, and cascade decays.  The processes
include
\begin{eqnarray*}
e^+ e^- &\to& \tilde q \tilde q \\
e^+ e^- &\to& \tilde\ell \tilde\ell \\
e^+ e^- &\to& \tilde W_i \tilde W_j \\
e^+ e^- &\to& \tilde Z_i \tilde Z_j \\
e^+ e^- &\to& H_L^0+Z^0,H_H^0+Z^0,H_A^0+H_L^0,H_A^0+H_H^0,H^++H^-
\end{eqnarray*}
Note that SUSY Higgs production via $WW$ and $ZZ$ fusion, which can
dominate Higgs production processes at $\sqrt{s} > 500\,\GeV$,
is not included. Spin correlations are neglected, although 
3-body sparticle decay matrix elements are included.

      $e^+e^-$ cross sections with polarized beams are included for
both Standard Model and SUSY processes. The keyword \verb|EPOL| is
used to set $P_L(e^-)$ and $P_L(e^+)$, where
$$
P_L(e) = (n_L-n_R)/(n_L+n_R)
$$
so that $-1 \le P_L \le +1$. Thus, setting \verb|EPOL| to $-.9,0$ will
yield a 95\% right polarized electron beam scattering on an unpolarized
positron beam.

Isajet 7.84 and beyond include an added subroutine ISASEE which calculates
$e^+e^-\rightarrow SUSY$ and $HIGGS$ total cross sections 
depending on collider energy and beam polarization for a given spectrum
generated from ISASUSY or ISASUGRA. The cross sections are output to the
relevant Les Houches Accord (LHA) file. The PRTEESIG flag in the 
Makefile must be enabled for this to work.

\subsubsection{Technicolor} Production of a technirho of arbitrary
mass and width decaying into $W^\pm Z^0$ or $W^+ W^-$ pairs. The cross
section is based on an elastic resonance in the $WW$ cross section
with the effective $W$ approximation plus a $W$ mixing term taken from
EHLQ.  Additional technicolor processes may be added in the future.

\subsubsection{Extra Dimensions} The possibility that there might be
more than four space-time dimensions at a distance scale $R$ much larger
than $G_N^{1/2}$ has recently attracted interest. In these theories,
$$ 
G_N = {1 \over 8\pi R^\delta M_D^{2+\delta}}\,, 
$$
where $\delta$ is the number of extra dimensions and $M_D$ is the
$4+\delta$ Planck scale. Gravity deviates from the standard theory at a
distance $R \sim 10^{22/\delta-19}\,{\rm m}$, so $\delta\ge2$ is
required. If $M_D$ is of order $1\,{\rm TeV}$, then the usual heirarchy
problem is solved, although there is then a new heirarchy problem of why
$R$ is so large.

      In such models the graviton will have many Kaluza-Klein
excitations with a mass splitting of order $1/R$. While any individual
mode is suppressed by the four-dimensional Planck mass, the large number
of modes produces a cross section suppressed only by $1/M_D^2$. The
signature is an invisible massive graviton plus a jet, photon, or other
Standard Model particle. The \verb|EXTRADIM| process implements this
reaction using the cross sections of Giudice, Rattazzi, and Wells,
hep-ph/9811291. The number $\delta$ of extra dimensions, the mass scale
$M_D$, and the logical flag \verb|UVCUT| are specified using the keyword
\verb|EXTRAD|. If \verb|UVCUT| is \verb|TRUE|, the cross section is cut
off above the scale $M_D$; the model is not valid if the results depend
on this flag.

\subsection{Multiparton Hard Scattering}

      All the processes listed in Section~\ref{hard} are either $2\to2$
processes like \verb|TWOJET| or $2\to1$ $s$-channel resonance processes
followed by a 2-body decay like \verb|DRELLYAN|. The QCD parton shower
described in Section~\ref{qcdshower} below generates multi-parton final
states starting from these, but it relies on an approximation which is
valid only if the additional partons are collinear either with the
initial or with the final primary ones. Since the QCD shower uses exact
non-colliear kinematics, it in fact works pretty well in a larger region
of phase space, but it is not exact.

      Non-collinear multiparton final states are interesting both in
their own right and as backgrounds for other signatures. Both the matrix
elements and the phase space for multiparton processes are complicated;
they have been incorporated into ISAJET for the first time in
Version~7.45. To calculate the matrix elements we have used the MadGraph
package by Stelzer and Long, Comput.\ Phys.\ Commun.\ {\bf81}, 357
(1994), hep-ph/9401258. This automatically generates the amplitude using
\verb|HELAS|, a formalism by Murayama, Watanabe, and Hagiwarak
KEK-91-11, that calculates the amplitude for any Feynman diagram in
terms of spinnors, vertices, and propagators. The MadGraph code has been
edited to incorporate summations over quark flavors. To do the phase
space integration, we have used a simple recursive algorithm to generate
$n$-body phase space. We have included limits on the total mass of the
final state using the \verb|MTOT| keyword. Limits on the $p_T$ and
rapidity of each final parton can be set via the \verb|PT| and \verb|Y|
keyworks, while limits on the mass of any pair of final partons can be
set via the \verb|MIJTOT| keyword. These limits are sufficient to shield
the infrared and collinear singularities and to render the result
finite. However, the parton shower populates all regions of phase space,
so careful thought is needed to combine the parton-shower based and
multiparton based results.

      While the multiparton formalism is rather general, it still takes
a substantial amount of effort to implement any particular process. So
far only one process has been implemented.

\subsubsection{$Z + {\rm 2\ jets}$} The \verb|ZJJ| process generates a
$Z$ boson plus two jets, including the $q\bar{q} \to Z q \bar{q}$, $gg
\to Z q\bar{q}$, $q\bar{q} \to Zgg$, $qq \to Zqq$, and $gq \to Z gq$
processes. The $Z$ is defined to be jet 1; it is treated in the narrow
resonance approximation and is decayed isotropically. The quarks,
antiquarks, and gluons are defined to be jets 2 and 3 and are
symmetrized in the usual way.

\subsection{QCD Radiative Corrections\label{qcdshower}}

      After the primary hard scattering is generated, QCD radiative
corrections are added to allow the possibility of many jets. This is
essential to get the correct event structure, especially at high
energy.

      Consider the emission of one extra gluon from an initial or a
final quark line,
$$
q(p) \to q(p_1) + g(p_2)
$$
From QCD perturbation theory, for small $p^2$ the cross section is
given by the lowest order cross section multiplied by a factor
$$
\sigma = \sigma_0  \alpha_s(p^2)/(2\pi p^2) P(z)
$$
where $z=p_1/p$ and $P(z)$ is an Altarelli-Parisi function. The same
form holds for the other allowed branchings,
\begin{eqnarray*}
g(p) &\to& g(p_1) + g(p_2) \\
g(p) &\to& q(p_1) + \bar q(p_2)
\end{eqnarray*}
These factors represent the collinear singularities of perturbation
theory, and they produce the leading log QCD scaling violations for the
structure functions and the jet fragmentation functions. They also
determine the shape of a QCD jet, since the jet $M^2$ is of order
$\alpha_s p_t^2$ and hence small.

      The branching approximation consists of keeping just these
factors which dominate in the collinear limit but using exact,
non-collinear kinematics. Thus higher order QCD is reduced to a
classical cascade process, which is easy to implement in a Monte Carlo
program. To avoid infrared and collinear singularities, each parton in
the cascade is required to have a mass (spacelike or timelike) greater
than some cutoff $t_c$. The assumption is that all physics at lower
scales is incorporated in the nonperturbative model for hadronization.
In ISAJET the cutoff is taken to be a rather large value,
$(6\,\GeV)^2$, because independent fragmentation is used for the jet 
fragmentation; a low cutoff would give too many hadrons from
overlapping partons. It turns out that the branching approximation not
only incorporates the correct scaling violations and jet structure but
also reproduces the exact three-jet cross section within factors of
order 2 over all of phase space.

      This approximation was introduced for final state radiation by
Fox and Wolfram. The QCD cascade is determined by the probability for
going from mass $t_0$ to mass $t_1$ emitting no resolvable radiation.
For a resolution cutoff $z_c < z < 1-z_c$, this is given by a simple
expression,
$$      
P(t_0,t_1)=\left(\alpha_s(t_0)/\alpha_s(t_1)\right)^{2\gamma(z_c)/b_0}
$$
where
$$
\gamma(z_c)=\int_{z_c}^{1-z_c} dz\,P(z),\qquad
b_0=(33-2n_f)/(12\pi)
$$
Clearly if $P(t_0,t_1)$ is the integral probability, then $dP/dt_1$ is
the probability for the first radiation to occur at $t_1$. It is
straightforward to generate this distribution and then iteratively to
correct it to get a cutoff at fixed $t_c$ rather than at fixed $z_c$.

      For the initial state it is necessary to take account of the
spacelike kinematics and of the structure functions. Sjostrand has
shown how to do this by starting at the hard scattering and evolving
backwards, forcing the ordering of the spacelike masses $t$. The
probability that a given step does not radiate can be derived from the
Altarelli-Parisi equations for the structure functions. It has a form
somewhat similar to $P(t_0,t_1)$ but involving a ratio of the structure
functions for the new and old partons. It is possible to find a bound
for this ratio in each case and so to generate a new $t$ and $z$ as for
the final state. Then branchings for which the ratio is small are
rejected in the usual Monte Carlo fashion. This ratio suppresses the
radiation of very energetic partons. It also forces the branching $g
\to t + \bar t$ for a $t$ quark if the $t$ structure function vanishes
at small momentum transfer.

      At low energies, the branching of an initial heavy quark into a
gluon sometimes fails; these events are discarded and a warning is
printed.

      Since $t_c$ is quite large, the radiation of soft gluons is cut
off. To compensate for this, equal and opposite transverse boosts are
made to the jet system and to the beam jets after fragmentation with a
mean value
$$
\langle p_t^2\rangle = (.1\,\GeV) \sqrt{Q^2}
$$
The dependence on $Q^2$ is the same as the cutoff used for DRELLYAN and
the coefficient is adjusted to fit the $p_t$ distribution for the $W$.

      Radiation of gluons from gluinos and scalar quarks is also
included in the same approximation, but the production of gluino or
scalar quark pairs from gluons is ignored. Very little radiation is
expected for heavy particles produced near threshold.

      Radiation of photons, $W$'s, and $Z$'s from final state quarks is
treated in the same approximation as QCD radiation except that the
coupling constant is fixed. Initial state electroweak radiation is not
included; it seems rather unimportant. The $W^+$'s, $W^-$'s and $Z$'s
are decayed into the modes allowed by the \verb|WPMODE|, \verb|WMMODE|,
and \verb|Z0MODE| commands respectively. {\it Warning:} The branching
ratios implied by these commands are not included in the cross section
because an arbitrary number of $W$'s and $Z$'s can in principle be
radiated.

\subsection{Jet Fragmentation:}

      Quarks and gluons are fragmented into hadrons using the
independent fragmentation ansatz of Field and Feynman. For a quark
$q$, a new quark-antiquark pair $q_1 \bar q_1$ is generated with
$$
u : d : s = .43 : .43 : .14
$$
A meson $q \bar q_1$ is formed carrying a fraction $z$ of the momentum,
$$
E' + p_z' = z (E + p_z)
$$
and having a transverse momentum $p_t$ with $\langle p_t \rangle =
0.35\,\GeV$. Baryons are included by generating a diquark with
probability 0.10 instead of a quark; adjacent diquarks are not
allowed, so no exotic mesons are formed. For light quarks $z$ is
generated with the splitting function
$$
f(z) = 1-a + a(b+1)(1-z)^b, \qquad
a = 0.96, b = 3
$$
while for heavy quarks the Peterson form
$$
f(z) = x (1-x)^2 / ( (1-x)^2 + \epsilon x )^2
$$
is used with $\epsilon = .80 / m_c^2$ for $c$ and $\epsilon = .50 /
m_q^2$ for $q = b, t, y, x$. These values of $\epsilon$ have been
determined by fitting PEP, PETRA, and LEP data with ISAJET and should
not be compared with values from other fits. Hadrons with longitudinal
momentum less than zero are discarded. The procedure is then iterated
for the new quark $q_1$ until all the momentum is used. A gluon is
fragmented like a randomly selected $u$, $d$, or $s$ quark or
antiquark. 

      In the fragmentation of gluinos and scalar quarks, supersymmetric
hadrons are not distinguished from partons. This should not matter
except possibly for very light masses. The Peterson form for $f(x)$ is
used with the same value of epsilon as for heavy quarks, $\epsilon =
0.5 / m^2$.

      Independent fragmentation correctly describes the fast hadrons in
a jet, but it fails to conserve energy or flavor exactly. Energy
conservation is imposed after the event is generated by boosting the
hadrons to the appropriate rest frame, rescaling all of the
three-momenta, and recalculating the energies.

\subsection{Beam Jets}

      There is now experimental evidence that beam jets are different in
minimum bias events and in hard scattering events. ISAJET therefore uses
similar a algorithm but different parameters in the two cases.

      The standard models of particle production are based on pulling
pairs of particles out of the vacuum by the QCD confining field,
leading naturally to only short-range rapidity correlations and to
essentially Poisson multiplicity fluctuations. The minimum bias data
exhibit KNO scaling and long-range correlations. A natural explanation
of this was given by the model of Abramovskii, Kanchelli and Gribov.
In their model the basic amplitude is a single cut Pomeron with
Poisson fluctuations around an average multiplicity $\langle n
\rangle$, but unitarity then produces graphs giving $K$ cut Pomerons
with multiplicity $K\langle n \rangle$.

      A simplified version of the AKG model is used in ISAJET. The
number of cut Pomerons is chosen with a distribution adjusted to fit the
data. For a minimum bias event this distribution is
$$
P(K) = ( 1 + 4 K^2 ) \exp{-1.8 K}
$$
while for hard scattering
$$
P(1) \to 0.1 P(1),\quad  P(2) \to 0.2 P(2),\quad  P(3) \to 0.5 P(3)
$$
For each side of each event an $x_0$ for the leading baryon is selected
with a distribution varying from flat for $K = 1$ to like that for
mesons for large K:
$$
f(x) = N(K) (1- x_0)^c(K),\qquad c(K) = 1/K + ( 1 - 1/K ) b(s)
$$
The $x_i$ for the cut Pomerons are generated uniformly and then
rescaled to $1-x_0$. Each cut Pomeron is then hadronized in its own
center of mass using a modified independent fragmentation model with
an energy dependent splitting function to reproduce the rise in
$dN/dy$:
$$
f(x) = 1 - a  +  a(b(s) + 1)^ b(s),\qquad 
b(s) = b_0 + b_1  \log(s)
$$
The energy dependence is put into $f(x)$ rather than $P(K)$ because in
the AKG scheme the single particle distribution comes only from the
single chain. The probabilities for different flavors are taken to be
$$
u : d : s = .46 : .46 : .08
$$
to reproduce the experimental $K/\pi$ ratio.

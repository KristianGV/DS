%%%%% APPENDIX: Relic density - numerical details %%%%%

\section{Relic density -- numerical integration of the density equation}
\label{sec:numdens}

Let us write the evolution equation for the density,
\begin{equation} \label{eq:Boltzmann4}
  \frac{dY}{dx} = - \sqrt{\frac{\pi}{45G}} \frac{g_{*}^{1/2}m_1}{x^2}
  \langle \sigma_{\rm{eff}} v \rangle \left( Y^2 -
  Y_{\rm{eq}}^2 \right) 
\end{equation}
as
\begin{equation} 
  \frac{dY}{dx} = \lambda (Y^2 - q^2),
  \label{eq:evol}
\end{equation}
where $\lambda$ contains the annihilation rate and $q$ represents the
thermal-equilibrium density.

This equation is stiff and an explicit method, like Euler or Runge-Kutta, fails
to converge. To obtain a numerical solution, we use an adaptive implicit
trapezoidal method which we explain in the following.  Basically we discretize
the equation first with a trapezoidal then with an Euler method, and adapt the
step size according to the difference in the updated function values.

For simplicity we denote the right hand wide of eq.~(\ref{eq:evol}) as $f(x)$.
We further write $f_{i} = f(x_i)$ and similarly for the other functions
$\lambda(x)$ and $q(x)$. Given $Y_{i} = Y(x_i)$ we find $Y_{i+1} = Y(x_{i+1}) $
with $x_{i+1} = x_{i} + h$ as follows.

First we discretize the evolution equation as
\begin{equation}
  Y_{i+1} - Y_{i} = h \frac{ f_i + f_{i+1}}{2} .
\end{equation}
We insert
\begin{eqnarray}
  f_{i} &=& \lambda_{i} \left( Y_{i}^2 - q_{i}^2 \right), \\
  f_{i+1} &=& \lambda_{i+1} \left( Y_{i+1}^2 - q_{i+1}^2 \right) ,
\end{eqnarray}
and solve the resulting quadratic equation for $Y_{i+1}$ to obtain
\begin{equation}
  \label{eq:y_one}
  Y_{i+1} = \frac{c}{1+\sqrt{1+uc}} ,
\end{equation}
where
\begin{eqnarray}
c &=& 2 Y_{i} + u \left[ (q^2_{i+1} + \rho q^2_{i}) - \rho Y^2_{i} \right] , \\
u &=& h \lambda_{i+1} ,\\
\rho &=& \lambda_i / \lambda_{i+1} .
\end{eqnarray}
In the expression for $c$ we have explicitly indicated the order of evaluation
which we found avoids round-off errors. If in eq.~(\ref{eq:y_one}) $1+uc$ is
negative, we simply reduce the step $h$ to $h/2$ and try again.


Secondly we discretize the evolution equation as
\begin{equation}
  Y_{i+1} - Y_{i} = h f_{i+1} .
\end{equation}
We insert the expression for $f_{i+1}$ 
and solve the quadratic equation for $Y_{i+1}$ to obtain
\begin{equation}
  \label{eq:y_two}
  Y'_{i+1} = \frac{1}{2} \, \frac{c'}{1+\sqrt{1+uc'}} ,
\end{equation}
where
\begin{equation}
  c' = 4 \left( Y_{i} + u q^2_{i+1} \right) .
\end{equation}
Again if in eq.~(\ref{eq:y_two}) $1+uc'<0$, we reduce the step $h$ to $h/2$ and
try again.

We then adapt the step size according to the relative difference of $Y_{i+1} $
and $Y'_{i+1}$,
\begin{equation}
d = \left| \frac{ Y_{i+1} - Y'_{i+1} }{ Y_{i+1} } \right| .
\end{equation}
If the difference is larger than a prefixed $\epsilon$, set at 0.01, we reduce
the step size $h$ to $hs/\sqrt{\epsilon}$ but never to less than $h/10$.  $s$
is a safety factor set to 0.9. If $d<\epsilon$, we increase the step size by a
factor $s/\sqrt{\epsilon}$ but never by more than a factor of 5.  We do not
allow the step size to become smaller than $h_{\rm min} = 10^{-9}$. Error code
5 is reported if this happens.  Error code 4 occurs when $x_{i+1}$ is
numerically equal to $x_{i}$ because of round-off. Error code 6 occurs when the
number of steps exceeds a maximum of 100000.  Finally the initial step size is
taken to be 0.01.


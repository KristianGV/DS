%%%%%%%%%%%%%%%%%%%%%%%%%%%%%%%%%%%%%%%%%%%%%%%%%%%%%%%%%%%%%%%%%%%%

\section{Gamma rays from the halo -- theory}
\label{sec:cr_gamma}

Among the yields of decay or pair annihilations of halo dark matter particles,
the role played by gamma-rays could be a major one. Unlike the cases
involving charged particles, for gamma-rays it is straightforward
to relate the distribution of sources and the expected flux
at the earth. Most flux estimated can be obtained just by summing
over the contributions along 
lines of sight (or better, geodesics): gamma-rays have a low 
enough cross section on gas and dust and therefore the Galaxy is 
essentially transparent to them (except perhaps in the innermost part, 
very close to the region where a massive black hole is inferred); 
absorption by starlight and infrared background becomes effecient
only for very far away sources (redshift larger than about 1).

It follows that in case the gamma-ray signal is detectable, 
this might be the only chance for mapping the fine structure of a dark 
halo, with a much better resolution for inhomogeneities (clumps) with 
respect to what is achievable through dynamical measurements or lensing effects. 
This is especially true for annihilating dark matter. 
Turning the latter argument around, if the fine structure of the Galactic 
halo is clumpy, or if a large density enhancement is present towards the 
Galactic center, as seen in N-body simulations of dark matter halos,
this dark matter detection technique may be more promising than indicated 
by the estimates in which smooth non-singular halo 
scenarios are considered (recall that the fluxes per unit volume are 
proportional to the square of the dark matter density locally in space).

%A further reason to examine in details this detection methods is that
%we are approaching what will probably be the golden age for
%gamma-ray observations, with a several new experiments that are going 
%to map the gamma-ray sky. These experiments will have unprecedented 
%sensitivities and cover an energy range, namely 10 GeV -- few hundred GeV, 
%in which very scarce data are available at the time being and which may 
%turn out to be the most interesting for dark matter detection.
%The hypothesis of a gamma-ray signal from WIMP annihilations 
%will be tested for both by the upcoming space experiments 
%(GLAST, AMS, AGILE) and by the new generation of ground-based 
%air cherenkov telescopes (ACTs) being built (Magic, Hess, Veritas).

Several targets have been discussed as sources of gamma-rays from the
annihilation of dark matter particles. An obvious source is the dark
halo of our own galaxy~\cite{Turner:1986vr,Ipser:1987ru,Freese:1989tg,Berezinsky:1994wva} and in particular the Galactic center,
as the dark matter density profile is expected, in most models, to be 
picked towards it, possibly with huge enhancements close to te central 
black hole. The Galactic center is an ideal target for both ground-
and space-based gamma-ray telescopes. As satellite experiments have 
much wider field of views and will provide a full sky coverage,
they can in principle test the hypothesis of gamma-rays emitted in clumps of dark 
matter which may be present in the 
halo~\cite{Lake:1990du,Silk:1992bh,CalcaneoRoldan:2000yt,Bergstrom:1998zs,Bergstrom:1998jj,Bergstrom:2000bk}. 
Another possibility which has been considered is the case of 
gamma-ray fluxes from external nearby galaxies~\cite{Baltz:1999ra}. 
Furthermore, it has 
been proposed to search for an extragalactic flux originated by all 
cosmological annihilations of dark matter 
particles~\cite{Cline:1990mr,Gao:1991rz,Bergstrom:2001jj}.
\ds\ is suitable to compute the gamma-ray flux from all these (and possibly 
other) sources. 

\section{Continuous gamma yields and line signals}

The bulk of the gamma-ray yield from DM annihilation or decay arises from quark 
jets that when they hadronize/decay give rise to neutral pions which decay to
gamma rays. At loop level, it could also be possible to produce monochromatic gamma rays, 
which could be a smoking gun for dark matter searches. The advantage with the gamma-ray lines 
is the distinctive spectral signature, which has no plausible astrophysical counterpart. 

Compared to the monochromatic flux, the gamma-ray flux produced in 
$\pi^0$ decays is much larger but has less distinctive features.
The photon spectrum in the process of a pion decaying into $2\gamma$ is, 
independent of the pion energy, peaked at half of the $\pi^0$ mass, 
about 70~MeV, and symmetric with respect to this peak if plotted in 
logarithmic variables. Of course, this is true both for pions produced by DM 
and, e.g., for those generated by cosmic ray protons interacting with the interstellar medium.

When considered together with to the cosmic-ray induced Galactic gamma-ray
background, the DM induced signal looks like a component analogous
to the secondary flux due to nucleon nucleon interactions: it is
drowned into the Bremsstrahlung component at low energy, while it may 
be the dominant contribution at energies above 1 GeV or so. 
In fact, if the exotic component is indeed significant
an option to disentangle it would be to search for a break in the
energy spectrum at about the WIMP mass, where the line feature
might be present as well: while the maximal energy for a photon emitted 
in WIMP pair annihilations is equal to the WIMP mass,
the component from cosmic ray protons extends to much higher energies,
essentially with the same spectral index as for the proton spectrum
(the role played by the third main background component, 
inverse Compton emission, has still to settled at the time being and
may worsen the problem of discrimination against background).

Besides this (weak) spectral feature, another way to disentangle
the dark matter signal may be to exploit a directional signature:
data with a wide angular coverage should be analyzed to search for 
a gamma-ray flux component following the shape and density profile 
of the dark halo, including eventual contributions from clumps.


\section{Fluxes}

Given a density distribution of DM particles, we can define a source function that tells 
us how many particles that are produced per volume element per time unit (and per energy interval for differential yields) from either annihilation or decay. 
The DM signal ultimately only depends on the local 
injection rate of some stable (cosmic ray) particle $f$, per volume and energy,
\be
\label{psource}
\frac{d\mathcal{Q}(E_f,\mathbf{x})}{dE_f}=\sum_n \rho_\chi^n(\mathbf{x}) 
\left\langle\mathcal{S}_n(E_f)\right\rangle\,.
\ee
Here, $\rho_\chi$ is the DM density (of the respective component, in case of multi-component DM) 
and the ensemble average $\langle ...\rangle$ is taken 
over the DM velocities; in principle, it  depends on the spatial location $\mathbf{x}$, but in many 
applications of interest this can be neglected. 

The particle source terms $\mathcal{S}_n(E_f)$ encode the full information about 
the DM particle physics model. For a typical WIMP DM candidate, e.g., only the annihilation part ($n=2$)
contributes,
\be
 \mathcal{S}_2(E_f)=\frac{1}{N_\chi m_\chi^2}\sum_i \sigma_i v \frac{dN_i}{dE_f} \,,
 \label{eq:Sann}
\ee
where $\sigma_i$ is the annihilation cross section of two DM particles into final state $i$ and 
$dN_i/dE_f$ is the resulting number of stable particles of type $f$ per such 
annihilation and unit energy. $N_\chi$ is a symmetry factor that depends on the nature of DM; if 
DM is (not) its own antiparticle we have $N_\chi=2$ ($N_\chi=4$). 
The right-hand side of the above expression must be further integrated over $f(v)$, the velocity distribution of 
the \emph{relative} velocity of the two dark matter particles, but in practice it is typically sufficient to evaluate it 
for $v=0$.

For decaying DM, on the other hand, we have
\be
 \mathcal{S}_1(E_f)=\frac{1}{m_\chi}\sum_i \Gamma_i \frac{dN_i}{dE_f} \,,
 \label{eq:Sdec}
\ee
where $\Gamma_i$ denotes the partial decay widths. Let us stress, however, that 
Eq.~(\ref{psource}) is much more general in that it encapsulates also DM that is {\it both} 
annihilating and decaying, multi-component SM, as well as DM models with an internal $Z^n$ 
symmetry \cite{Belanger:2012zr,Belanger:2014bga,Ko:2014nha,Choi:2015bya}. 


For a telescope pointing in the direction $\psi$, the expected DM-induced differential flux in gamma 
rays or neutrinos -- i.e.~the expected number of particles per unit area, time and energy -- from a 
sky-region $\Delta \psi$  is thus given by a line-of-sight integral
\be
 \frac{d\Phi}{dE}= \frac1{4\pi}\int_{\Delta\psi} d\Omega \int_{\rm l.o.s.}\!\!\!\!d\ell\, \frac{d\mathcal{Q}}{dE}\,.
\ee
For an effectively point-like source at distance $d$, this line-of-sight integral simplifies to
\be
\int_{\Delta\psi} d\Omega \int_{\rm l.o.s.}\!\!\!\!d\ell\, \frac{d\mathcal{Q}}{dE}
\longrightarrow \frac{1}{d^2} \int dV \frac{d\mathcal{Q}}{dE}\,,
\ee
where the integral is over the extention of the source (much smaller
than $d$).

For decaying DM, the above line-of-sight integral always factorizes into the particle source term
$S_1$ given in Eq.~(\ref{eq:Sdec}) and a term that only depends on the DM distribution,
\be
\label{eq:phiann}
\frac{d\Phi^{\rm dec}}{dEd\Omega}= \frac1{4\pi}J^{\rm dec} S_1\,, \qquad J^{\rm dec} \equiv \int_{\rm l.o.s.}\!\!\!\!d\ell\, \rho
\ee
For annihilating DM, the corresponding factorization strictly speaking  {\it only} holds {\it if} the annihilation rate is independent of velocity:
\be
\label{eq:phidec}
\frac{d\Phi^{\rm ann}}{dEd\Omega}= \frac1{4\pi}J^{\rm ann} S_2\,, \qquad J^{\rm ann} \equiv \int_{\rm l.o.s.}\!\!\!\!d\ell\, \rho^2\,.
\ee
While notable exceptions exist (in particular for resonances \cite{Arina:2014fna}, $p$-wave annihilation \cite{xxx} and  
Sommerfeld-enhanced annihilation \cite{xxx}), this is a commonly encountered situation and hence of general interest.

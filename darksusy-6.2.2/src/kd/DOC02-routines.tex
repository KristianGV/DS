%%%%%%%%%%%%%%%%%%%%%%%%%%%%%%%%%%%%%%%%%%%%%%%%%%%%%%%%%%%%%%%%%%%%
\section{Kinetic decoupling -- routines}

Before using any of the routines provided in {\tt src/kd} for the first time, one has to 
call \ftb{dskdset} in order to make some necessary initializations; in particular, a call
to this routine ensures that the relevant tables for the relativistic degrees of freedom in the
early universe are correctly read in. Typically, the subroutines of greatest interest will be 
\ftb{dskdtkd} and \ftb{dskdmcut}, and there should be no need to call any of the other routines directly.

\ftb{dskdtkd} numerically solves the Boltzmann equation (\ref{dydx}) and determines
$T_{\rm kd}$ as given in Eq.~(\ref{eq:tkddef}). Here, special care is taken to 
accurately handle potential resonances in the scattering amplitude; to this end, 
\ftb{dskdboltz\_init} identifies the location of all relevant resonances and passes this
information to \ftb{dskdgammarate} where the integral of Eq.~(\ref{fT}) is performed.

Finally, \ftb{dskdmcut} returns the mass cutoff in the power spectrum, with an input parameter 
determining whether it is $M_{\rm fs}$ or $M_{\rm ao}$; the default call results in 
$M_{\rm cut}=\max\left[M_{\rm fs},M_{\rm ao}\right]$, i.e.~the mass of 
the smallest protohalos.

There are three interface functions that a particle physics module must provide for the inetic
decoupling routines in \code{src/} to work: \code{dskdm2} returns the full scattering matrix element squared, evaluated 
at $t=0$ or averaged over $t$, while \code{dskdm2simp} returns only the leading contribution
for small $\omega$ (expressed as a simple power-law in $\omega$, in which case there exists an
analytic solution for $T_\mathrm{kd}$ \cite{Bringmann:2006mu}). Lastly, the particle physics
module must provide a routine \code{dskdparticles} which, in analogy to the routine  \code{dsrdparticles}
for the case of chemical decoupling
discussed above, sets the location of resonances in $\left|\mathcal{M}\right|^2$.
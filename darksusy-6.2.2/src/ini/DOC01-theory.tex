%%%%%%%%%%%%%%%%%%%%%%%%%%%%%%%%%%%%%%%%%%%%%%%%%%%%%%%%%%%%%%%%%%%%%%
\section{Initialisation routines}

This directory contains general initialisation routines that need to be called to prepare calculations with 
\ds, but which are independent of the particle physics. Most importantly, it contains the
subroutine \code{dsinit}, which should be called at the beginning of any program using
\ds. In particular, this routine calls various more specific initialisation routines that are relevant
only for specific applications (such as the relic density calculation) and therefore reside
in the respective directory in \code{src/}. It also calls \code{dsinit\_module}, which initializes
the specific particle module that the user has chosen to link to when compiling the main program.
Another subroutine called by \code{dsinit} is \code{ini/dsreadnuclides}. It reads in names and properties 
of nuclei and stores them in common blocks, to which both direct detection and solar/earth capture
routines need access.

Lastly, the directory \code{/ini} also contains two files \code{dsdirname.c.in} and \code{dsvername.c.in}.
These are needed to provide a system-wide reference to the installation directory and version of
\ds, respectively. These refer to include files that are set at configure time, \code{dsdir.h} and \code{dsver.h}. The user should never have to modify these manually as they are determined at the configure stage.



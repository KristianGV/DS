%%%%%%%%%%%%%%%%%%%%%%%%%%%%%%%%%%%%%%%%%%%%%%%%%%%%%%%%%%%%%%%%%%%%%%
\section{Direct detection -- theory}

Specific choices of nuclear structure functions can be selected by calling \code{dsddset('sf',} \code{label)}, where the 
character variable \code{label} indicates the set of structure functions. For the default option ('\code{best}'),
e.g., the code automatically picks the best currently available structure function (depending on the nucleus). 
This mean, in order Fourier-Bessel, Sum-of-Gaussians, Fermi, Lewin-Smith.
The function returning the value of $\tilde\sigma_{\chi T}$ is to be provided by an interface function 
\code{dsddsigma(v,Er,A,Z,sigij, ierr)} residing in the particle physics module, where on input \code{v=$v$}, 
\code{Er=$E_R$}, 
\code{A=$A$}, \code{Z=$Z$} and on output the 27$\times$27 array \code{sigij} contains the (partial) equivalent 
cross sections  $\tilde\sigma_{ij}$ in cm$^2$ and the integer \code{ierr} contains a possible error code. 
The order of the entries in \code{sigij} corresponds to that of the independent nonrelativistic operators 
$\mathcal{O}_i$; for the first 11 entries, we use the same operators and convention as in 
Ref.~\cite{Fitzpatrick:2012ix}, while for the last 16 entries  we add the additional operators discussed in 
Ref.~\cite{GondoloScopel}. In particular, \code{sigij(1,1)} is the 
usual spin-independent cross section and \code{sigij(4,4)} is the usual spin-dependent cross section.
In addition, the direct detection module in \ds\ provides utility functions that can be used in the computation of the 
cross section. For example, the subroutine \code{dsddgg2sigma(v,} \code{er,A,Z,gg,sigij,ierr)} computes the (partial) 
equivalent  scattering cross sections $\tilde\sigma_{ij}$ for nucleus $(A,Z)$ at relative velocity $v$ and recoil energy 
$E_R$ starting from values of the $G_i^N$ constants in \code{gg}, with nuclear structure functions set by the 
previous call to \code{dsddset}. The actual nuclear recoil event rate as given in 
\begin{equation}
\frac{dR}{dE_R} = \sum_T c_T \frac{\rho_\chi^0}{m_T  m_\chi} \int_{v>v_{\rm min}} \frac{d\sigma_{\chi T}}{dE_R}  \frac{f({\bf v},t)}{v} d^3 v\,,
\label{eq:dRdE} 
\end{equation}
finally, is computed by the function \code{dsdddrde}.
The latter two functions are independent of the specific particle physics implementation and hence 
are contained in the core library.
In the above expression, the sum  runs over nuclear species in the detector, $c_T$ being the detector mass fraction in nuclear species 
$i$. $m_T$ is the nuclear (target) mass, and $\mu_{\chi T} = m_\chi m_T /(m_\chi + m_T)$ is the reduced DM--nucleus 
mass. 
Furthermore, $\rho_\chi^0$ is the local DM density, ${\bf v}$ the DM velocity relative to the detector, $v=|{\bf v}|$, and 
$f({\bf v},t)$ is the (3D) DM velocity distribution.  In order to impart a recoil energy $E_R$ to the nucleus, the DM
particle needs a minimal speed of $v_{\rm min}=\sqrt{M_TE_R/2\mu^2_{\chi T}}$. 


%%%%%%%%%%%%%%%%%%%%%%%%%%%%%%%%%%%%%%%%%%%%%%%%%%%%%%%%%%%%%%%%%%%%%%
\section{Dark matter self-interactions (si) -- theory}
\label{sec:si}

The CDM paradigm, resting on cold and collisionless dark matter, describes cosmological structure 
formation with remarkable accuracy at scales larger than about one Mpc. At smaller cosmological scales, 
on the other hand, this  paradigm is less well tested, and currently still allows for DM self-interactions  
as strong as the interaction between nucleons --  and thus much stronger than current limits on
DM interacting with SM particles. The possibility that DM could be relatively strongly interacting with itself \cite{Spergel:1999mh}
and thereby leave imprints on cosmological observables related to structure formation
has therefore seen significant interest in recent years, both from an astrophysical and a model-building
perspective. Self-interacting DM (SIDM) could even address \cite{Loeb:2010gj,Vogelsberger:2012ku,Peter:2012jh,
Zavala:2012us,Aarssen:2012fx,Elbert:2014bma,Kamada:2016euw,Robertson:2017mgj} the most pressing 
potential small-scale problems of $\Lambda$CDM cosmology 
\cite{Bullock:2017xww}, referred to as `core-vs.-cusp'' \cite{Flores:1994gz,Moore:1994yx}, `too-big-to-fail'
\cite{BoylanKolchin:2011de,BoylanKolchin:2011dk}, `diversity'  \cite{Oman:2015xda,Oman:2016zjn}
and `missing satellites' problems \cite{Moore:1999nt,Klypin:1999uc} (the latter problem is addressed for 
SIDM with late kinetic decoupling, which is a natural combination in many models \cite{Bringmann:2016ilk}).
Those specific discrepancies with the $\Lambda$CDM paradigm may of course well turn out to be either
due to  poorly modelled baryonic effects or observational uncertainties -- but these examples serve as 
evidence that SIDM {\it can} leave observable imprints even in the absence of interactions with the SM.
An unambiguous detection of such signatures, which are not expected in for example traditional
WIMP models of DM, would significantly narrow down the range of possible particle explanations
for the nature of DM.

The traditional reference quantity for  the impact of DM self-interactions on the halo 
structure is the momentum transfer cross section,
\be
\label{eq:sigmat}
 \sigma_T \equiv \int d\Omega\left(1-\cos\theta\right)\frac{d\sigma}{d\Omega}\,,
\ee
where $\sigma$ is the standard cross section for DM-DM scattering. 
This has the advantage of regulating large forward-scattering amplitudes, which should
not affect the DM distribution (the same goes for backward scattering which, however,
is treated symmetrically in this prescription. See Ref.~\cite{Tulin:2013teo,Kahlhoefer:2017umn} 
for a more detailed discussion).
The size of this quantity that is of cosmological relevance is very roughly given by
\be
  \sigma_T/m_\chi \sim 1\,\mathrm{cm}^2/\mathrm{g}\,.
\ee
Cross sections in this ballpark, in other words, may leave observable imprints and possibly address 
the various $\Lambda$CD; small-scale structure problems mentioned above, while much smaller cross 
sections have no impact on the structure of DM halos. Much larger values of $\sigma_T/m_\chi$ are 
ruled out, on the other hand, in particular from observations of galaxy clusters. For a detailed 
review that summarizes both various constraints on DM self-interactions and
models that have been discussed in the literature, we refer to Ref.~\cite{Tulin:2017ara}.

The details of the DM self-interactions depend on the underlying particle theory. A simple contact 
interaction term in the Lagrangian, for example, would lead to a constant (velocity-independent)
$\sigma_T$. Another well-motivated option is that the interaction is mediated by a massive particle
with mass $m_{\rm med}$, which leads to a Yukawa potential between the two DM particles:
\be
V(r)=\pm\frac{\alpha_\chi}{r} e^{-m_{\rm med} r}\,.
\ee
Here, $\alpha_\chi=g_\chi^2/(4\pi)$ describes the coupling strength between mediator and DM particles,
and the different signs refer to attractive (-) and repulsive (+) potentials. Scalar mediators only generate
attractive potentials, while for vector mediators this depends on the particles involved in the scattering: 
for (e.g.~Dirac) DM scattering with anti-DM, the force is attractive, otherwise it is repulsive. An interesting 
phenomenological aspect for the scattering of nonrelativistic particles in such a potential is the resulting 
strong velocity-dependence 
of $\sigma_T$; this allows to achieve large scattering cross sections for the relatively small velocities 
of $\sim$30\,km/s typically encountered
in dwarf galaxies (where one observes potential discrepancies with the $\Lambda$CDM expectations)
while evading the strong bounds at cluster scales for velocities of $\sim$\,1000km/s.
The transfer cross section resulting from scattering in a Yukawa potential has been extensively 
studied, and depends on the scattering regime ($v_{\rm rel}$ is the relative velocity between the
DM particles):

\begin{itemize}
	\item In the \textbf{{Born} regime}  ($\alpha_\chi m_\chi \lesssim m_{\rm med}$), $\sigma_T$ 
	can be calculated perturbatively, leading (for both attractive and repulsive potentials) to~\cite{Feng:2009hw}
	\begin{equation}
	\sigma_T^{\rm Born} = \frac{8\pi \alpha_\chi^2}{m^2_\chi v^4_{\rm rel}} \left( \ln \left[1+\frac{m^2_\chi v^2_{\rm rel}}{m^2_{\rm med}} \right]-\frac{m^2_\chi v^2_{\rm rel}}{m^2_{\rm med}+m^2_\chi v^2_{\rm rel}}\right)\,.
	\label{sigmatborn}
	\end{equation}

	\item In the 
	\textbf{{classical} regime} ($m_\chi v_{\rm rel} \gtrsim m_{\rm med}$), a large numbers of partial waves
	contributes such that non-perturbative effects are important. In principle, this can be computed by numerically 
	solving the Schr\"odinger equation for each case and then summing all contributions  \cite{Tulin:2013teo}.
         A computationally much more efficient method is to use parameterizations \cite{Cyr-Racine:2015ihg} of 
         numerical results from the Plasma literature that have been obtained for screened Coulomb 
         scattering \cite{Khrapak:2003kjw,PhysRevE.70.056405}. For an attractive potential, these are given by
\smallskip         
	\begin{eqnarray}
	\sigma_T^{\mathbf{-}}= \frac{\pi}{m_{\rm med}^2}\times 
\left\{
    \begin{array}{ll}        
    2\, \beta^2 \ln[1+\beta^{-2}] & \textrm{for } \beta \lesssim 10^{-2} \\[0.5ex]
     \frac{7\beta^{1.8} +1960 (\beta/10)^{10.3}}{1+1.4\beta +0.006 \beta^4 +160(\beta/10)^{10}} & \textrm{for } 10^{-2} \lesssim \beta \lesssim 10^2\\[1ex]
                     0.81\,(1+\ln \beta -(2\ln\beta)^{-1})^2 & \textrm{for }  \beta \gtrsim 10^2  
 \end{array}
 \right.
 \label{sigmatclassicalattractive}
	\end{eqnarray}
\smallskip         
	where $\beta\equiv 2\alpha_\chi m_{\rm med}/(m_\chi v^2_{\rm rel})$\,, while for a repulsive potential we have 
\smallskip         
	\begin{eqnarray}
	\sigma_T^{\mathbf{+}}= \frac{\pi}{m_{\rm med}^2}\times 
\left\{
    \begin{array}{ll}        
  2\beta^2 \ln [1+\beta^{-2}] & \textrm{for } \beta \lesssim 10^{-2} \\[0.5ex]
    \frac{8\beta^{1.8}}{1+5\beta^{0.9} +0.85 \beta^{1.6}} & \textrm{for } 10^{-2} \lesssim \beta \lesssim 10^4\\[1ex]
	(\ln 2\beta -\ln\ln2\beta)^2 & \textrm{for } \beta \gtrsim 10^4
 \end{array}
 \right.
 \label{sigmatclassicalrepulsive}
	\end{eqnarray}
\smallskip         

	
	\item Finally, there is the \textbf{{resonant regime}} (for $m_\chi v_{\rm rel} \lesssim m_{\rm med}$ or 
	$\alpha_\chi m_\chi \gtrsim m_{\rm med}$). Full numerical solutions can be obtained by solving the 
	Schr\"odinger equation explicitly in this regime. These are well described by the following analytic expressions
	that result from approximating the Yukawa potential with a Hulth\'en potential \cite{Tulin:2013teo}
	\begin{equation}
	\sigma_T^{\rm Hulth{\'e}n}= \frac{16 \pi}{m_\chi v^2_{\rm rel}} \sin^2\left({\rm Arg}\left[ \frac{i\Gamma [i \Theta v_{\rm rel}]}{\Gamma[\lambda_+]\,\Gamma[\lambda_-]}\right] \right)\,,
	\label{eq:hulthen}
	\end{equation}

	where $\Theta\equiv \frac{m_\chi}{\sqrt{2\zeta(3)} m_{\rm med}}$
	%\sim \frac{m_\chi}{1.55 m_{\rm med}}$ 
	and $\lambda_{\pm}\equiv 1+i\Theta  v_{\rm rel}/2\pm \sqrt{\alpha_\chi \Theta - \Theta^2 v^2_{\rm rel}/4}$\, for an attractive potential and $\lambda_{\pm}\equiv 1+i\Theta  v_{\rm rel}/2\pm i\sqrt{\alpha_\chi \Theta + \Theta^2 v^2_{\rm rel}/4}$\, for a repulsive potential.
	
\end{itemize}








